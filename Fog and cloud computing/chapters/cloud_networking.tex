\chapter{Cloud networking}
\section{Data centers networks}
Unquestionably, networking is the core of \emph{cloud computing}, in fact, it
has been made feasible by the continuous evolution of internet connectivity.
Historically, networks have been classified into three layers:
\begin{itemize}
    \item \emph{Tier 1}: core part of network infrastructure that allows all
    other networks on the internet to communicate each other;
    \item \emph{Tier 2}: portions of \emph{tier 1} infrastructure which is
    bought by internet service providers which then sold access to it to customers;
    \item \emph{Tier 3}: part of the network infrastructure that allows final
    users to gain access to other networks;
\end{itemize}
\begin{note}
    \emph{Tier 2} networks communicated each other as peers through special nodes
    called Internet Exchange Points (IXP).
\end{note}

\begin{figure}[h!]
    \centering
    \img{network-tiers.png}{0.4}
    \caption{Internet infrastructure}
\end{figure}

\noindent
The transformation of the internet experience, with the shift towards web only
services (e.g. web applications instead of desktop applications), resulted in
a significant increase in volumes of shared data and required bandwidth. To
address this increase in bandwidth demand, cloud providers introduced Content
Delivery Networks (CDN) whose function is to bring service providers closer
to the final users. A secondo step was made with the spread of IoT and
decentralized computing paradigms such as \emph{fog computing}.

\begin{figure}[ht!]
    \centering
    \subfloat[Traditional infrastructure]{\img{traditional-network.png}{0.48}}
    \hfill
    \subfloat[Modern infrastructure]{\img{modern-network.png}{0.48}}\\
    \subfloat[Today's infrastructure]{\img{todays-network.png}{0.48}}\\
\end{figure}

\subsection{Characteristics of data center networks}
Data centers are huge structures with tons of servers, so they demand high
bandwidth with the outside and a very low RTT within their premises. Also,
each data center is a signle administrative domain, meaning that its
administrators have full control over its intern network infrastructure, 
endpoints and protocols. This allows them to deviate from standards and adopt
each kind of custom solutions, as long as the services they host are reachable.

This liberty becames relavant as soon as we realize that most of the traffic
in a data center is created by machine-to-machine communications, leaving
user-to-machine traffic a small percentage of the total amount. 

\begin{figure}[h!]
    \centering
    \img{data-center-traffic}{0.5}
    \caption{Traffic within a data center}
\end{figure}

\noindent
This means that the performance of the networks inside the data centers, which
are called \emph{interconnection networks}, are critical.

\paragraph{Interconnetion networks}
An \emph{interconnection network} is composed of node and link or communication
channels. Nodes can be servers, memory units or even processors. Then, each node
has an interface connecting it to the network, and the number of link the node
is connected defines its degree. The interconnection fabric is made of
switches\footnotemark and links. Switches receive packes, and look inside them
to determine the way toward their final destination. An $n$-way switch is a
switch with $n$ ports that can be connected to $n$ links. The interconnection
fabric determines whether the \emph{interconnection network} is \emph{blocking}
or not. It is \emph{non-blocking} if any permutation of source and destination
nodes can connect each other at any time. It is \emph{blocking} if this is not
true.

\footnotetext{We're refering to a generic switching device without saying if
is a layer 2 or layer 3 one}

\emph{Interconnection networks} can be distinguished by three parameters:
\begin{itemize}
    \item \emph{Topology}: defines the way nodes are interconnected;
    \item \emph{Routing}: defines how a message gets from source to destination;
    \item \emph{Flow control}: negotiates how the buffer space is allocated;
\end{itemize}
The \emph{topology} also determines the \emph{network diameter} and
\emph{bisection width}.

\begin{definition}[Network diameter]
    Network diameter is defined as the average distance between pairs of
    nodes.
\end{definition}
\begin{definition}[Bisection width]
    Bisection width is defined as the minimum number of links that have to be
    cut to partition the interconnection network into two halves.
\end{definition}

\noindent
When an \emph{interconnection network} is partitioned into two networks of
them same size, the bisection bandwidth measures the communication bandwidth
between the two. We talk about full bisection bandwidth when one half of nodes
can communicate simultaneously with the other half.

\paragraph{Topologies}
There are two main types of \emph{topologies}:
\begin{itemize}
    \item \emph{Static networks}: servers are connected through direct connections;
    \item \emph{Switched networks}: servers are connected through switches;
\end{itemize}

\begin{figure}[h!]
\centering
\subfloat[\emph{Bus topology}]{\resizebox*{0.48\textwidth}{!}{\begin{graph}
    \tikzset{
        rect/.style={rectangle, draw, inner sep=5mm},
    }
    \node[rect] (0) {Memory};
    \node[rect] (1) [right of=0, xshift=20mm] {Memory};
    \node[rect] (2) [right of=1, xshift=20mm] {Memory};
    \node[rect] (3) [right of=2, xshift=20mm] {Memory};

    \node[rect] (4) [below left of=0, yshift=-20mm] {Cache};
    \node[rect] (5) [right of=4, xshift=20mm] {Cache};
    \node[rect] (6) [right of=5, xshift=20mm] {Cache};
    \node[rect] (7) [right of=6, xshift=20mm] {Cache};

    \node[rect] (8) [below of=4, yshift=4.5mm, minimum width=20.9mm,
        minimum height=18mm] {Proc};
    \node[rect] (9) [below of=5, yshift=4.5mm, minimum width=20.9mm,
        minimum height=18mm] {Proc};
    \node[rect] (10) [below of=6, yshift=4.5mm, minimum width=20.9mm,
        minimum height=18mm] {Proc};
    \node[rect] (11) [below of=7, yshift=4.5mm, minimum width=20.9mm,
        minimum height=18mm] {Proc};

    \node[empty] (a) [below of=0, yshift=3mm] {};
    \node[empty] (b) [right of=a, xshift=20mm] {};
    \node[empty] (c) [right of=b, xshift=20mm] {};
    \node[empty] (d) [right of=c, xshift=20mm] {};
    \node[empty] (e) [right of=d, xshift=10mm] {};
    \node[empty] (f) [left of=a, xshift=-10mm] {};
    
    \node[empty] (g) [above of=8, yshift=12.6mm] {};
    \node[empty] (h) [right of=g, xshift=20mm] {};
    \node[empty] (i) [right of=h, xshift=20mm] {};
    \node[empty] (j) [right of=i, xshift=20mm] {};

    \draw[-]    (0) -- (a)
                (1) -- (b)
                (2) -- (c)
                (3) -- (d);

    \draw[-]    (4) -- (g)
                (5) -- (h)
                (6) -- (i)
                (7) -- (j);
    
    \draw[-, line width=1.3pt] (f) -- (e);

    \node[empty] (z) [below of=j, yshift=-45.5mm] {};
\end{graph}}}
\hfill
\subfloat[\emph{Hypercube topology}]{\resizebox*{0.48\textwidth}{!}{\begin{graph}
    \tikzset{
        rect/.style={rectangle, draw, fill, minimum size=3mm, inner sep=0},
        point/.style={circle, draw=gray, fill=gray, minimum size=2mm, inner sep=0},
        every path/.style={draw=gray}
    }
    
    \foreach \x in {0,3}
        \foreach \y in {0,-3}
        {
            \node[point] (a\x\y) at (\x, \y) {};
            \node[rect] (r\x\y) [below right of=a\x\y, shift={(-6mm, 6mm)}] {};
            \node[point] (b\x\y) [below left of=a\x\y, shift={(2mm, 2mm)}] {};
            \node[rect] (r1\x\y) [below right of=b\x\y, shift={(-6mm, 6mm)}] {};

            \draw[-]    (a\x\y) -- (b\x\y)
                        (a\x\y) -- (r\x\y)
                        (b\x\y) -- (r1\x\y);
        }

    \draw[-] (a00) -- (a30) -- (a3-3) -- (a0-3) -- (a00);
    \draw[-] (b00) -- (b30) -- (b3-3) -- (b0-3) -- (b00);

    \foreach \x in {6,9}
        \foreach \y in {0,-3}
        {
            \node[point] (c\x\y) at (\x, \y) {};
            \node[rect] (r2\x\y) [below right of=c\x\y, shift={(-6mm, 6mm)}] {};
            \node[point] (d\x\y) [below left of=c\x\y, shift={(2mm, 2mm)}] {};
            \node[rect] (r3\x\y) [below right of=d\x\y, shift={(-6mm, 6mm)}] {};

            \draw[-]    (c\x\y) -- (d\x\y)
                        (c\x\y) -- (r2\x\y)
                        (d\x\y) -- (r3\x\y);
        }

    \draw[-] (c60) -- (c90) -- (c9-3) -- (c6-3) -- (c60);
    \draw[-] (d60) -- (d90) -- (d9-3) -- (d6-3) -- (d60);
    \draw[-, bend left=25]
        (a00) edge (c60)
        (a30) edge (c90)
        (b00) edge (d60)
        (b30) edge (d90);
    \draw[-, bend right=25]
        (a0-3) edge (c6-3)
        (a3-3) edge (c9-3)
        (b0-3) edge (d6-3)
        (b3-3) edge (d9-3);
\end{graph}}}
\end{figure}
\newpage
\begin{figure}[ht!]
\ContinuedFloat
\centering
\subfloat[\emph{2D-mesh topology}]{\resizebox*{0.48\textwidth}{!}{\begin{graph}
    \tikzset{
        rect/.style={rectangle, draw, fill, minimum size=3mm, inner sep=0},
        point/.style={circle, draw=gray, fill=gray, minimum size=2mm, inner sep=0},
        every path/.style={draw=gray}
    }
    
    \foreach \x in {0, 3, 6, 9}
        \foreach \y in {0,-3, -6, -9}
        {
            \node[point] (a\x\y) at (\x, \y) {};
            \node[rect] (r\x\y) [below right of=a\x\y, shift={(-6mm, 6mm)}] {};

            \draw[-] (a\x\y) -- (r\x\y);
        }

    \foreach \x in {0, 3, 6}
        \foreach \y in {1, ..., 4}
        {
            \FPeval{\res}{clip(\x+3)}
            \ifthenelse{\y = 1}{
                \draw[-] (a\x0) -- (a\res0);
                \draw[-] (a\x0) -- (a\x-3);
            }{
                \ifthenelse{\y = 2}{
                    \draw[-] (a\x-3) -- (a\res-3);
                    \draw[-] (a\x-3) -- (a\x-6);
                }{
                    \ifthenelse{\y = 3}{
                        \draw[-] (a\x-6) -- (a\res-6);
                        \draw[-] (a\x-6) -- (a\x-9);
                    }{
                        \draw[-] (a\x-9) -- (a\res-9);
                    }
                }
            }
        }
    \draw[-]    (a90) -- (a9-3) -- (a9-6) -- (a9-9);
\end{graph}}}
\hfill
\subfloat[\emph{Torus topology}]{\resizebox*{0.48\textwidth}{!}{\begin{graph}
    \tikzset{
        rect/.style={rectangle, draw, fill, minimum size=3mm, inner sep=0},
        point/.style={circle, draw=gray, fill=gray, minimum size=2mm, inner sep=0},
        curve/.style={rounded rectangle, draw, minimum height=15mm,
            minimum width=105mm, shift={(15mm, 7.5mm)}},
        rcurve/.style={rounded rectangle, draw, minimum height=15mm,
            minimum width=105mm, rotate=-90, shift={(15mm, -7.5mm)}},
        every path/.style={draw=gray}
    }
    
    \foreach \x in {0, 3, 6, 9}
        \foreach \y in {0,-3, -6, -9}
        {
            \node[point] (a\x\y) at (\x, \y) {};
            \node[rect] (r\x\y) [below right of=a\x\y, shift={(-6mm, 6mm)}] {};

            \draw[-] (a\x\y) -- (r\x\y);
        }

    \foreach \x in {0, 3, 6}
        \foreach \y in {1, ..., 4}
        {
            \FPeval{\res}{clip(\x+3)}
            \ifthenelse{\y = 1}{
                \draw[-] (a\x0) -- (a\res0);
                \draw[-] (a\x0) -- (a\x-3);
            }{
                \ifthenelse{\y = 2}{
                    \draw[-] (a\x-3) -- (a\res-3);
                    \draw[-] (a\x-3) -- (a\x-6);
                }{
                    \ifthenelse{\y = 3}{
                        \draw[-] (a\x-6) -- (a\res-6);
                        \draw[-] (a\x-6) -- (a\x-9);
                    }{
                        \draw[-] (a\x-9) -- (a\res-9);
                    }
                }
            }
        }
    \draw[-]    (a90) -- (a9-3) -- (a9-6) -- (a9-9);

    \node[curve] (r1) at (a30) {};
    \node[curve] (r2) at (a3-3) {};
    \node[curve] (r3) at (a3-6) {};
    \node[curve] (r4) at (a3-9) {};

    \node[rcurve] (r5) at (a0-3) {};
    \node[rcurve] (r6) at (a3-3) {};
    \node[rcurve] (r7) at (a6-3) {};
    \node[rcurve] (r8) at (a9-3) {};

    \node[empty] (z) [below of=a0-9, yshift=-3.5mm] {};
\end{graph}}}
\caption{\emph{Static networks topologies}}
\end{figure}

\begin{figure}[h!]
    \centering
    \subfloat[\emph{Crossbar switch topology}]{\img{crossbar-switch.png}{0.48}}
    \hfill
    \subfloat[\emph{Omega switch topology}]{\img{omega-switch.png}{0.48}}      
    \caption{\emph{Switched network topologie}}
\end{figure}

\noindent
Up to this point, we can say that an \emph{interconnection network} must be
scalable, provide a high bandwidth with low latency e must guarantee what is
called \emph{location transparent communication}, meaning that every server
should communicate with the others with similar speed and latency. In simple
terms, the position of a server in the data center shouldn't have an impact on
its networking performance. This latter requirement, translates in the
impossibility of adopting a hierarchical organization of servers.

Costs, also comes in hand when designing an \emph{interconnection network}. Both
servers and network apparatus costs are relevant, but talking about networking,
the choice of routers and switches often requires a compromised between latency
and costs. For example, based on the number of ports of a router we can have
low-radix and high-ridix routers. The former has a few ports, while the latter
has more; thus bandwidth is divided into a smaller number of wide ports and
a larger number of narrow ports respectively.

\bigskip\noindent
In general, data centers networks are designed around the following schema:
\begin{figure}[h!]
    \centering
    \img{general-schema.png}{0.6}
    \caption{Generale \emph{interconnetion network} schema}
\end{figure}

\noindent
Layer 2 and layer 3 has pros and cons that have to balanced to guarantee ease
of configuration, administration and problem-solving.

Layer 2, that is the ethernet switching part of the infrastructure, is easy to
configure, thanks to autoconfiguration protocols such as DHCP, and allow the
addition of servers in a plug \& play way, but it's important to limit broadcast
domains to avoid link saturarion due to broadcast frames, and dealing with the
Spanning Tree Protocol might be a struggle.

On the other hand, IP routing on layer 3 makes scalability easy thanks to
hieranchical addressing and allows to obtain multipath routing with equal-cost
multipath\footnote{We will see more about this later in this section}. However,
its confiuration in more complex and migration requires changing the IP addresses.

\begin{note}
    At the end of the day, the entire \emph{interconnection network} will look
    like a giant switch that allows all the servers to communicate one another.
\end{note}

\subsection{Topologies more in depth}
In the \emph{butterfly network topology} data always travel through the most
efficient route, but two packets attempting to reach the same port at the same
time would couse a collision, so this is a \emph{blocking topology}.
\begin{note}
    The name comes from the pattern of inverted triangles created by the
    interconnections, which look like butterfly wings.
\end{note}
\emph{Clos networks} is a \emph{non-blocking topology} that consists of two
\emph{butterfly networks} in which the last stage of the output is fused to the
first stage of the output. This way, all packet sent, overshoot their destination
and then hop back to it. However, this overshoot is mostly unnecessary and
increases the latency because each packet takes twice as many hops as it need.
A solution to this, is the \emph{folded clos topology} which is obtained by
making input and output nodes share the same switch devices. Such networks are
also called \emph{fat trees} and the \emph{topology} is also called
\emph{fat tree topology}.

\begin{figure}[ht!]
    \centering
    \subfloat[\emph{Clos topology} made of two \emph{3-stage butterfly
    networs} and radix-2 routers]{\img{clos-network.png}{0.48}}
    \hspace{1.5cm}
    \subfloat[\emph{Folded clos topology}]{\img{folded-clos-network.png}{0.3}}
    \caption{\emph{Clos} VS \emph{folded clos topology}}
\end{figure}

\paragraph{Fat tree topology}
In a \emph{fat tree} network, servers are placed at the leafs, while internal
and root nodes are switches. To increase bandwidth, additional links are also
added to near-root nodes. The advantage of such \emph{topology} is that all the
switching elements are identical and this allows costs reductions. Another,
already named, characteristic is the presence of multiple equal-cost path that
both allow for load splitting and \emph{location trasparent communication.}

\begin{figure}[h!]
    \centering
    \img{fat-tree.png}{0.6}
    \caption{Tipical \emph{fat tree topology}}
\end{figure}

\noindent
A \emph{fat tree} made of $k$ pods has two layers and $k/2$ switches for each
pod. Each switch at the lower layer is directly connected to $k/2$ servers
and $k/2$ switches of the upper layer.

\begin{note}
    From the lower to the upper, layers are also called \emph{edge},
    \emph{aggregation} and \emph{core} respectively.
\end{note}

\noindent
Similarly, each \emph{aggregation layer} switch is connected to $k/2$ \emph{core
layer} switches which also have $k/2$ more connections to other aggregation
switches, one for each pod. This configuration results in an infrastructure
that has $4$ servers, actually, $4$ racks with $4$ on-top-of-rack switches,
$k(k+1)$ switches and a total of $(k/2)^2$ paths connecting each pairs of
servers.

When it comes to IP assignment, switches are numbered left-to-right and
bottom-to-top as $base.pod.switch.1$ (in the previous image, $base=87$).
\emph{Core layer} switches instead, are numbered as $base.k.i.j$ where $k$ is
the number of pods and $i,j$ denotates the coordinates of the switch in the
grid, starting from top-left. Finally, servers IPs are in the form
$base.pod.switch.serverID$ and $serverID$ is assigned from left-to-right.

\subsection{Open issues}
Existing \emph{topologies}, including \emph{Fat trees}, still fail to address
workload balancing and TCP-related issues.

\begin{note}
    TCP retransmission and congestiona avoidance policies causes a degradation
    of throughput, so new protocols are currently being developed to displace
    TCP (e.g. QUIC).
\end{note}

\noindent
When it comes to workload, we distinguish between small and large flows. As the
names suggests, the first denotates small flows of data which require low
latency, while the second regards bigger flows which should benefit a higher
throughput.

\begin{note}
    Small and large flows are also called mice and elephants flow respectively.
\end{note}

\begin{eg}[Traffic balancing problems]
Let's assume to be in the following situation and to have many flows going from
$S$ to $D$:

\begin{figure}[h!]
    \centering
    \img{tb-1.png}{0.4}
\end{figure}

\noindent
We want to assign each flow to a path in the most efficient way. We could just
use the Equal-Cost Multipath routing (ECMP) to randomly assign a path to each
flow. This would be a simple approch, but it would also be totally agnostic to
available resources; thus we could experience long-lasting collisions between
elephant flows.

\begin{figure}[h!]
    \centering
    \img{tb-2.png}{0.4}
\end{figure}

\noindent
The problem complicates when we have more source and destination nodes to
connect. Ideally, we would like paths to not intersect each other, like in the
above picture:

\begin{figure}[h!]
    \centering
    \img{tb-3.png}{0.4}
\end{figure}

\noindent
Wrong choices might cause collision in both upward and downward paths:
\begin{figure}[ht!]
    \centering
    \subfloat[Upward path congested]{\img{tb-4.png}{0.4}}
    \subfloat[Downward path congested]{\img{tb-5.png}{0.4}}
\end{figure}
\end{eg}

\paragraph{Proposed solutions}
Up to now, various solutions have been proposed, but none of them solves complety
all the issues. For example, an idea was to detect elephant flows by putting
a threshold on link capacity (e.g. 10\%), compute non-conflicting path for
each flow over the limit and using plain ECMP for mice flows.

Other proposals where to reroute flows based on link utilization on queue
occupancy. However, the first favors mice flows vice versa, the second tends to
promote elephant flows. We could even try to spread all packets evenly, but that
would make congestion control hard to implement.

\section{Networking in virtualised environments}