\chapter{Could ecosystem}
\section{Some definitions}
\begin{figure}[h!]
    \centering
    \img{cloud-computing-5points.png}{0.7}
    \caption{Five key aspects of \emph{cloud computing}}
\end{figure}
\begin{figure}[h!]
    \centering
    \img{cloud-computing-nist.png}{1}
    \caption{\emph{NIST} reference model for \emph{cloud computing}}
\end{figure}
\begin{note}
    A carrier is someone who provides access to a cloud service, such as
    Telecom, while a broker is a subject that handles the delivery of cloud
    services to users, such as a portal to the cloud (e.g. Booking.com).
\end{note}

\noindent Before analysing some key aspects of \emph{cloud computing}, some
definitions are required.

\subsection{Virtualization}
\begin{definition}[Virtualization]
    Virtualization allows the abstraction of computing resources by hiding
    their physical characteristics from the way systems, applications and users
    interact with them.
\end{definition}

\begin{figure}[h!]
    \centering
    \subfloat[Non-virtualized system]{\img{non-virtualized-system.png}{0.38}}
    \hspace{1.5cm}
    \subfloat[Virtualized system]{\img{virtualized-system.png}{0.38}}
    \caption{General architecture of virtualized systems}
\end{figure}

\subsection{Single-tenancy VS multi-tenancy}
\begin{definition}[Single-tenancy]
    With single-tenancy each user has its own software instance.
\end{definition}
\begin{definition}[Multi-tenancy]
    With multi-tenancy a single isntance of a software can serve multiple users.
\end{definition}\noindent
As a consequence of these definitions, we can say that with \emph{single-tenancy}
each user requires a dedicated set of resources to fulfill its needs, while
\emph{multi-tenancy} allows sharing resources managment and costs among all of
them.

Actually, in a \emph{multi-tenancy} enviroment, a group of users who share
common access with specific privileges to a software instance, is called
\emph{tenent}. An instance includes, among others, data, configurations and
users management.

\newpage
\begin{figure}[ht!]
    \centering
    \subfloat[\emph{Single-tenancy}]{\img{single-tenancy.png}{0.33}}
    \hspace{2mm}
    \subfloat[\emph{Single-tenancy} for database and \emph{multi-tenancy} for
    the application]{\img{partial-multi-tenancy.png}{0.33}}
    \hspace{2mm}
    \subfloat[\emph{Multi-tenancy}]{\img{full-multi-tenancy.png}{0.3}}
    \caption{\emph{Single-tenancy} VS \emph{multi-tenancy}}
\end{figure}

\subsection{Elasticity and resource provisioning}
Resource provisioning is the ability of adding or removing resources at a fine
grain (e.g. one server at a time) with a short lead time (e.g. minutes). This
allows a close matching of resources and workloads and, together with the
\emph{pay-as-you-go} model, brings elasticity to the users, who no longer need
to worry about sudden spikes of resource usage. The key advantage of elasticity
is that it reduces problems resulting from \emph{underprovisioning} and
\emph{overprovisioning}.

\emph{Overprovisioning} happens when current workload is using much less resources
than the ones that have been allocated, thus resulting in a waste.
Simmetrically, \emph{underprovisioning} means that the available resources are
insufficient to serve requests, thus resulting in bad performances and
possible loss of clients.

When users relies on proprietary resources (e.g. company's private servers) the
amount of allocated resources must be determined by the quantity that is
required to meet the highest predicted pick. Since it's difficult to predict
picks, most of the time there will be redundant resources.

Another advantage of the way \emph{cloud computing} provides resources is on costs,
because with a \emph{cloud approch} there isn't any inital cost for buying and
setting up the infrastructure.

\section{Delivery models}
Going back to the key aspects of \emph{cloud computing}, delivery models define
the kind of product that is provided. The main types of model are three:
\begin{itemize}
    \item \emph{Software-as-a-Service} (\emph{SaaS}): an application is provided
    to the users through the web;
    \item \emph{Platform-as-a-Service} (\emph{PaaS}): APIs and deployment
    environments are provided to developers;
    \item \emph{Infrastructure-as-a-Service} (\emph{IaaS}): computing resources
    are provided to system administrators;
\end{itemize}

\subsection{Software as a service}
Applications are supplied by service providers and users have no control over their
capabilities and underlying cloud infrastructure. This model isn't suitable for
real-time applications or applications for which data isn't allowed to be stored
externally.

\paragraph{Examples} Google Drive, Google Docs, Spotify

\subsection{Platform as a Service}
\emph{PaaS} allows developers to deploy applications (consumer-created or acquired
from others) using tools and programming languages supported by the service
provider. Developers have control over the deployed applications and, possibly,
over the app hosting environment. However, they still don't have access to the
underlying insfrastructure (e.g. network devices, OSs, storage).


This model isn't indicated for portable applications, apps in which proprietary
programming languages are used or which require hardware and software customization.

\paragraph{Examples} Google App Engine, Heroku

\subsection{Infrastructure as a Service}
Services provided by this model include: server hosting, storage, computing
hardware, operating systems, virtual instances, load balancing, internet access
and bandwidth provisioning.

System administrators can manage OSs, storage, deployed applications and may
even have little control over network components such as firewalls. They're
able to deploy arbitrary software including operating systems, but there's
still an underlying instrastructure that can't be accessed.

\paragraph{Examples} Amazon EC2

\begin{note}
    Everything can be deployed as a service, for example databases or hardware,
    thus \emph{Database-as-a-Service} and \emph{Hardware-as-a-Service} may
    exist.
\end{note}

\section{Deployment models}
Deployment models describe the way cloud infrastructures may be accessed
and by whom and who is responsible for their maintainance. In particular,
there are four types of environment.

\subsection{Public cloud}
\begin{itemize}
    \item \emph{Consumer}: general users or large industrial groups;
    \item \emph{Service provider}: there's an organization that settles down
    and manages the infrastructure;
    \item \emph{Resource location}: all resources are within the premises of the
    cloud provider;
    \item \emph{Multi-tenancy model}: different consumers are served by the same
    instances;
\end{itemize}

\subsection{Private cloud}
\begin{itemize}
    \item \emph{Consumer}: a specific organization;
    \item \emph{Service provider}: the same organization that uses it or a third
    party one;
    \item \emph{Resource location}: it can either be on-premises if the organization
    doesn't want to remotely host data, on off-premises if it relyes on a third
    party private cloud;
\end{itemize}

\subsection{Community cloud}
\begin{itemize}
    \item \emph{Consumer}: a community composed by one or more organizations
    which share common concerns such as their mission, policies and security
    considerations;
    \item \emph{Service provider}: either the organizations or a third party;
    \item \emph{Resource location}: either on-premises or off-promises;
\end{itemize}

\subsection{Hybrid cloud}
It's the composition of more deployment models which remain unique
entities, but are bound together by standardised or proprietary technologies
that enable data and applications portability.

For example an organization might use a \emph{public cloud} for some aspects
of its business and a \emph{private} one for its sensitive data.