\chapter{Introduction}
\begin{definition}[Data science]
    Data science is the science of learning from data, and it employs various
    techniques such as statistical methodologies, machine learning and data
    mining.
\end{definition}\noindent
\emph{Data science} relies on large amount of data that is constantly increasing
in quantity, variety and veracity (i.e. data is more and more accurate and
conform to the studied reality). Finally, since data is fast in production, it
needs to be collected and manipulated as fast.

Because of these characteristics, we are now facing many challenges in storing,
sharing, analysing, transfering and securing data. To address these problematics,
distributed and scalable systems are required. This resulted in the proliferation
of large data centers that, by storing tons of servers, have centraliesed data
manipulation and storing. This came in hand with a reduction in plants, IT
assets, operating and energy costs.

\emph{Cloud computing} allowed all of this to be possible and furthermore,
transformed what was a product into a service that can suit specific users
needs. For example, companies that maintain data centers can provide storage,
computational power, network access and many other commodities to their
customers as on-demand services for which they pay-as-they-go, meaning that
they can pay only for the resources they actually use.

To be considered convinient, a cloud service must satisfy some requirements:
\begin{itemize}
    \item\emph{Connectivity}: it must be possible to move data through the
    network;
    \item\emph{Interactivity}: users need to have an interface through which
    monitor their products, the resources they're using and made configurations;
    \item\emph{Reliability}: users mustn't be affected by maintainer's failures
    (i.e. providers must prevent and handle them);
    \item\emph{Performance}: services must be better than what customers already
    have;
    \item\emph{Pay-as-you-go}: there mustn't be upfront fees and users must only
    pay for what they use;
    \item\emph{Programmability}: it must be easy for users to develop and maintain
    their products;
    \item\emph{Data management}: providers must be able to handle large amount
    of data;
    \item\emph{Efficiency}: the plants must be efficient on costs and power usage;
    \item\emph{Scalability and elasticity}: providers must be flexible and give
    rapid response to users needs;
\end{itemize}

\begin{definition}[Cloud computing]
    Cloud computing is a model for enabling ubiquitous, convenient,
    on-demand network access to a shared pool of configurable computing
    resources (e.g. networks, servers, storage, applications and services) that
    can be rapidly provisioned and released with minimal management effort
    or service provider interaction.
\end{definition}

\noindent
So, \emph{cloud computing} relies on 5 key points:
\begin{enumerate}
    \item\emph{Shared or pooled resources}: resources are retrieved from a
    common pool;
    \item\emph{Broad network access}: it must be available from anywhere
    through internet connection and must be accessible using any platform;
    \item\emph{On-demand automated reservation}: customers can reserve
    resources as needed without requiring human interaction with cloud service
    provider;
    \item\emph{Rapid elasticity}: resources can be rapidly and automatically
    scaled up and down to satisfy customers demands;
    \item\emph{Pay by use}: services are metered like a utility, so users must
    pay only for the services they're using, and they must also be able to
    cancel them at anytime;
\end{enumerate}\noindent
Sure, centralizing too much can be a bad idea (e.g. if an entire data center
goes down, tons of services may be unable for a long time and for everyone),
and many operations that require just a \q{small} portion of data might be
computed outside a data center and nearer to the source of that data.
From this idea, originated the concept of \emph{fog computing}.

\begin{definition}[Fog computing]
    Fog computing is an evoluting of cloud computing in which computation is
    decentralized by subdividing it into multiple nodes that act indipendently.
    Groups of nodes refere to an aggregation node that handle them and
    more aggregation nodes are then connected to a central point that provides,
    among others, an interface for users.
\end{definition}
\begin{note}
    A \emph{fog node} is an active component that performs some operations
    and not hust a passive data collector such as a sensor.
\end{note}

\noindent
This may allow reducing resources required by a single data center, since it
might store and handle only the results of manipulations already performed by
\emph{fog nodes} or \emph{aggregation nodes}.