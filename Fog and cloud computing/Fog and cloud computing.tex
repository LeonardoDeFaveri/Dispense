\documentclass[12pt, a4paper]{report}

\usepackage[T1]{fontenc}
\usepackage[utf8]{inputenc}
\usepackage{geometry}
 \geometry{
 a4paper,
 total={170mm,257mm},
 left=20mm,
 top=20mm,
}

\usepackage{titlesec}
\titleformat
{\chapter}
[display]{\bfseries\Large\itshape}
{Chapter Nr.\thechapter}
{0.5ex}
{
    \rule{\textwidth}{1pt}
    \vspace{1ex}
	\centering
}
[\vspace{-0.5ex}\rule{\textwidth}{0.3pt}]

\renewcommand{\contentsname}{Indice}

\usepackage[dvipsnames]{xcolor}
\usepackage{hyperref}
\hypersetup{
  colorlinks,
  linkcolor={red!90!black},
  citecolor={blue!50!black},
  urlcolor={blue!80!black}
  pdftitle={Dispense di algoritmi e strutture dati},
  pdfpagemode=FullScreen,
}

\usepackage{bold-extra}
\usepackage{multirow}
\usepackage{longtable}
\usepackage{enumitem}
\usepackage{diagbox}
\usepackage{graphicx}
\usepackage{graphbox}

\usepackage[export]{adjustbox}
\usepackage{subfig}
\usepackage{caption}
\graphicspath{{./images/}}% Imposta il percorso relativo per le immagini
\renewcommand{\figurename}{Image}% Cambia il testo delle caption per le immagini
\renewcommand{\tablename}{Table}% Cambia il testo delle caption per le tabelle

\usepackage{amssymb,amsmath,amsthm}
\theoremstyle{remark}
\newtheorem*{note}{\textbf{Note}}

\newtheoremstyle{def}
{\topsep}{\topsep}%
{\em}{}%
{\bfseries}{}
{\newline}
{%
  \rule{\textwidth}{0.4pt}\\*%
  \thmname{#1}~\thmnumber{#2}\thmnote{\ -\ #3}\label{def:#2}.\\*[-1.5ex]
  \rule{\textwidth}{0.4pt}
}

\newtheoremstyle{eg}
{\topsep}{\topsep}%
{\em}{}%
{\bfseries}{}
{\newline}
{%
  \rule{\textwidth}{0.4pt}\\*%
  \thmname{#1}~\thmnumber{#2}\thmnote{\ -\ #3}\label{eg:#2}.\\*[-1.5ex]
  \rule{\textwidth}{0.4pt}
}

\newtheoremstyle{prob}
{\topsep}{\topsep}%
{\em}{}%
{\bfseries}{}
{\newline}
{%
  \rule{\textwidth}{0.4pt}\\*%
  \thmname{#1}~\thmnumber{#2}\thmnote{\ -\ #3}\label{prob:#2}.\\*[-1.5ex]
  \rule{\textwidth}{0.4pt}
}

\theoremstyle{def}
\newtheorem{definition}{Definition}
\theoremstyle{prob}
\newtheorem{problem}{Problem}
\theoremstyle{eg}
\newtheorem{eg}{Example}

\newcommand{\codecolor}{Sepia!90!Black}

\newtheoremstyle{codestyle}
{\topsep}{\topsep}%
{\tt\color{\codecolor}}{}%
{\bfseries\color{black}}{}
{\newline}
{%
  \rule{\textwidth}{0.4pt}\\*%
  \thmname{#1}~\thmnumber{#2}\thmnote{\ -\ #3}\label{code:#2}.\\*[-1.5ex]
  \rule{\textwidth}{0.4pt}
}
\theoremstyle{codestyle}
\newtheorem{codeblock}{Frammento}
\newenvironment{code}[1]
{\begin{codeblock}[#1]\hbadness=10000}{\end{codeblock}}
\newenvironment{minicode}[1]
{\begin{codeblock}[#1]\hbadness=10000\begin{minipage}[t]{\textwidth}}
{\end{minipage}\end{codeblock}}
\newenvironment{codecont}{\noindent\tt\color{\codecolor}\hbadness=10000}{}

\newcommand{\hitem}[1]{\refstepcounter{enumii}\hypertarget{#1:\arabic{enumii}}{\item}}

\newcommand{\com}[1]{{\color{ForestGreen}\% #1}\\}
\newcommand{\bc}[1]{{\bf#1}}
\newcommand{\nl}{\par\smallskip\noindent}

\newcommand{\customindent}{19pt}
\newcommand{\lineheight}{\baselineskip}
\newcommand{\rmbreak}{\par\vspace{-\lineheight}}
\newcommand{\fstart}{\hspace{-\customindent}\par\rmbreak}

% Livelli di indentazione progressivi
% Sorgente                  Risultato
% \ind if (true) then\\     if (true) then
%   print "Hello"               print "Hello"
\newcommand{\ind}{\noindent\rmbreak\noindent\hangindent=\customindent}
\newcommand{\indf}{\rmbreak\hangindent=38pt}
\newcommand{\indff}{\rmbreak\hspace*{19pt}\hangindent=57pt}
\newcommand{\indfff}{\rmbreak\hspace*{38pt}\hangindent=76pt}
\newcommand{\indffff}{\rmbreak\hspace*{57pt}\hangindent=95pt}
\newcommand{\indfffff}{\rmbreak\hspace*{76pt}\hangindent=114pt}

\newcommand{\rmindent}{\rmbreak\hspace*{0pt}\hangindent=\customindent}

\usepackage{ifthen}
\usepackage{calc}
\usepackage{tikz}
\usetikzlibrary{shapes.geometric, shapes.misc}
\usetikzlibrary{positioning}
\usetikzlibrary{backgrounds}
\newenvironment{graph}{\begin{tikzpicture}[
  baseline={(0,0)},
  node distance={20mm},
  main/.style={draw, circle, minimum size=10mm},
  empty/.style={inner sep=0em},
  every path/.style={line width=0.8pt}
]}{\end{tikzpicture}}

% Comando per posizionare un nodo rispetto a un altro (#3)
% in base a un angolo (#1) e a una distanza (#2)
% \node[empty] (1) [pos=angolo:distanza from nodo] {};
\tikzset{
  pos/.style args={#1:#2 from #3}{
    at=(#3.#1), anchor=#1+180, shift=(#1:#2)
  }
}

\usepackage[title]{appendix}
\usepackage{bm}

\newcommand{\q}[1]{``#1''}
\newcommand{\Mod}[1]{\ \mathrm{mod}\ #1}

\newcommand{\img}[2]{\includegraphics[width=#2\textwidth]{#1}}

\title{Handouts of fog and cloud computing}
\author{Leonardo De Faveri}
\date{A.A. 2021/2022}

\begin{document}
\maketitle
\tableofcontents

\chapter{Introduction}
\begin{definition}[Data science]
    Data science is the science of learning from data, and it employs various
    techniques such as statistical methodologies, machine learning and data
    mining.
\end{definition}\noindent
\emph{Data science} relies on large amount of data that is constantly increasing
in quantity, variety and veracity (i.e. data is more and more accurate and
conform to the studied reality). Finally, since data is fast in production, it
needs to be collected and manipulated as fast.

Because of these characteristics, we are now facing many challenges in storing,
sharing, analysing, transfering and securing data. To address these problematics,
distributed and scalable systems are required. This resulted in the proliferation
of large data centers that, by storing tons of servers, have centraliesed data
manipulation and storing. This came in hand with a reduction in plants, IT
assets, operating and energy costs.

\emph{Cloud computing} allowed all of this to be possible and furthermore,
transformed what was a product into a service that can suit specific users
needs. For example, companies that maintain data centers can provide storage,
computational power, network access and many other commodities to their
customers as on-demand services for which they pay-as-they-go, meaning that
they can pay only for the resources they actually use.

To be considered convinient, a cloud service must satisfy some requirements:
\begin{itemize}
    \item\emph{Connectivity}: it must be possible to move data through the
    network;
    \item\emph{Interactivity}: users need to have an interface through which
    monitor their products, the resources they're using and made configurations;
    \item\emph{Reliability}: users mustn't be affected by maintainer's failures
    (i.e. providers must prevent and handle them);
    \item\emph{Performance}: services must be better than what customers already
    have;
    \item\emph{Pay-as-you-go}: there mustn't be upfront fees and users must only
    pay for what they use;
    \item\emph{Programmability}: it must be easy for users to develop and maintain
    their products;
    \item\emph{Data management}: providers must be able to handle large amount
    of data;
    \item\emph{Efficiency}: the plants must be efficient on costs and power usage;
    \item\emph{Scalability and elasticity}: providers must be flexible and give
    rapid response to users needs;
\end{itemize}

\begin{definition}[Cloud computing]
    Cloud computing is a model for enabling ubiquitous, convenient,
    on-demand network access to a shared pool of configurable computing
    resources (e.g. networks, servers, storage, applications and services) that
    can be rapidly provisioned and released with minimal management effort
    or service provider interaction.
\end{definition}

\noindent
So, \emph{cloud computing} relies on 5 key points:
\begin{enumerate}
    \item\emph{Shared or pooled resources}: resources are retrieved from a
    common pool;
    \item\emph{Broad network access}: it must be available from anywhere
    through internet connection and must be accessible using any platform;
    \item\emph{On-demand automated reservation}: customers can reserve
    resources as needed without requiring human interaction with cloud service
    provider;
    \item\emph{Rapid elasticity}: resources can be rapidly and automatically
    scaled up and down to satisfy customers demands;
    \item\emph{Pay by use}: services are metered like a utility, so users must
    pay only for the services they're using, and they must also be able to
    cancel them at anytime;
\end{enumerate}\noindent
Sure, centralizing too much can be a bad idea (e.g. if an entire data center
goes down, tons of services may be unable for a long time and for everyone),
and many operations that require just a \q{small} portion of data might be
computed outside a data center and nearer to the source of that data.
From this idea, originated the concept of \emph{fog computing}.

\begin{definition}[Fog computing]
    Fog computing is an evoluting of cloud computing in which computation is
    decentralized by subdividing it into multiple nodes thast act indipendently.
    Groups of nodes referes to an aggregation node that handle them and
    more aggregation nodes are then connected to a central point that provides,
    among others, an interfaced for users.
\end{definition}
\begin{note}
    A \emph{fog node} is an active component that performs some operations
    and not hust a passive data collector such as a sensor.
\end{note}

\noindent
This may allow reducing resources required by a single data center, since it
might store and handle only the results of manipulations already performed by
\emph{fog nodes} or \emph{aggregation nodes}.
\chapter{Could ecosystem}
\section{Some definitions}
\begin{figure}[h!]
    \centering
    \img{cloud-computing-5points.png}{0.7}
    \caption{Five key aspects of \emph{cloud computing}}
\end{figure}
\begin{figure}[h!]
    \centering
    \img{cloud-computing-nist.png}{1}
    \caption{\emph{NIST} reference model for \emph{cloud computing}}
\end{figure}
\begin{note}
    A carrier is someone who provides access to a cloud service, such as
    Telecom, while a broker is a subject that handles the delivery of cloud
    services to users, such as a portal to the cloud (e.g. Booking.com).
\end{note}

\noindent Before analysing some key aspects of \emph{cloud computing}, some
definitions are required.

\subsection{Virtualization}
\begin{definition}[Virtualization]
    Virtualization allows the abstraction of computing resources by hiding
    their physical characteristics from the way systems, applications and users
    interact with them.
\end{definition}

\begin{figure}[h!]
    \centering
    \subfloat[Non-virtualized system]{\img{non-virtualized-system.png}{0.38}}
    \hspace{1.5cm}
    \subfloat[Virtualized system]{\img{virtualized-system.png}{0.38}}
    \caption{General architecture of virtualized systems}
\end{figure}

\subsection{Single-tenancy VS multi-tenancy}
\begin{definition}[Single-tenancy]
    With single-tenancy each user has its own software instance.
\end{definition}
\begin{definition}[Multi-tenancy]
    With multi-tenancy a single isntance of a software can serve multiple users.
\end{definition}\noindent
As a consequence of these definitions, we can say that with \emph{single-tenancy}
each user requires a dedicated set of resources to fulfill its needs, while
\emph{multi-tenancy} allows sharing resources managment and costs among all of
them.

Actually, in a \emph{multi-tenancy} enviroment, a group of users who share
common access with specific privileges to the software isntances, is called
\emph{tenent}. An instance includes, among others, data, configurations and
users management.

\newpage
\begin{figure}[ht!]
    \centering
    \subfloat[\emph{Single-tenancy}]{\img{single-tenancy.png}{0.33}}
    \hspace{2mm}
    \subfloat[\emph{Single-tenancy} for database and \emph{multi-tenancy} for
    the application]{\img{partial-multi-tenancy.png}{0.33}}
    \hspace{2mm}
    \subfloat[\emph{Multi-tenancy}]{\img{full-multi-tenancy.png}{0.3}}
    \caption{\emph{Single-tenancy} VS \emph{multi-tenancy}}
\end{figure}

\subsection{Elasticity and resource provisioning}
Resource provisioning is the ability of adding or removing resources at a fine
grain (e.g. one server at a time) with a short lead time (e.g. minutes). This
allows a close matching of resources and workloads and, together with the
\emph{pay-as-you-go} model, brings elasticity to the users, who no longer need
to worry about sudden spikes of resource usage. The key advantage of elasticity
is that it reduces the problems resulting from \emph{underprovisioning} and
\emph{overprovisioning}.

\emph{Overprovisioning} happens when the workload is using much less resources
than the ones that have been allocated, thus resulting in a waste.
Simmetrically, \emph{underprovisioning} means that the available resources are
insufficient to serve the requests, thus resulting in bad performance and
possible loss of clients.

When users relies on proprietary resources (e.g. company's private servers) the
amount of allocated resources must be determined by the quantity that is
required to meet the highest predicted pick. Since it's difficult to predict
picks, most of the time there will be redundant resources.

Another advantage of the way \emph{cloud computing} provides resources is on costs,
because with a \emph{cloud approch} there isn't any inital cost for buying and
setting up the infrastructure.

\section{Delivery models}
Going back to the key aspects of \emph{cloud computing}, delivery models define
the kind of product that is provided. The main types of model are three:
\begin{itemize}
    \item \emph{Software-as-a-Service} (\emph{SaaS}): an application is provided
    to the users through the web;
    \item \emph{Platform-as-a-Service} (\emph{PaaS}): APIs and deployment
    environments are provided to developers;
    \item \emph{Infrastructure-as-a-Service} (\emph{IaaS}): computing resources
    are provided to system administrators;
\end{itemize}

\subsection{Software as a service}
Applications are supplied by service providers and users have no control over their
capabilities and underlying cloud infrastructure. This model isn't suitable for
real-time applications or applications for which data isn't allowed to be stored
externally.

\paragraph{Examples} Google Drive, Google Docs, Spotify

\subsection{Platform as a Service}
\emph{PaaS} allows developers to deploy applications (consumer-created or acquired
from others) using tools and programming languages supported by the service
provider. Developers have control over the deployed applications and, possibly,
over the app hosting environment. However, they still don't have access to the
underlying insfrastructure (e.g. network devices, OSs, storage).


This model isn't indicated for portable applications, apps in which proprietary
programming languages are used or which require hardware and software customization.

\paragraph{Examples} Google App Engine, Heroku

\subsection{Infrastructure as a Service}
Services provided by this model includes: server hosting, storage, computing
hardware, operating systems, virtual instances, load balancing, internet access
and bandwidth provisioning.

System administrators can manage OSs, storage, deployed applications and may
even have little control over network components such as firewalls. So, they're
able to deploy arbitrary software including operating systems. However, there
is still a below instrastructure that can't be accessed.

\paragraph{Examples} Amazon EC2

\begin{note}
    Everything can be deployed as a service, for example databases or hardware,
    thus \emph{Database-as-a-Service} and \emph{Hardware-as-a-Service} may
    exist.
\end{note}

\section{Deployment models}
Deployment models describe the way the cloud infrastructure may be accessed
and by whom and who is the owner. In particular, there are four types of
environment.

\subsection{Public cloud}
\begin{itemize}
    \item \emph{Consumer}: general users or large industrial groups;
    \item \emph{Service provider}: there's an organization that settles down
    and manages the infrastructure;
    \item \emph{Resource location}: all resources are within the premises of the
    cloud provider;
    \item \emph{Multi-tenancy model}: different consumers are served by the same
    instances;
\end{itemize}

\subsection{Private cloud}
\begin{itemize}
    \item \emph{Consumer}: a specific organization;
    \item \emph{Service provider}: the same organization that uses it or a third
    party one;
    \item \emph{Resource location}: it can either be on-premises if the organization
    doesn't want to remotely host data, on off-premises if it relyes on a third
    party private cloud;
\end{itemize}

\subsection{Community cloud}
\begin{itemize}
    \item \emph{Consumer}: a community composed by one or more organizations
    which share common concerns such as their mission, policies and security
    considerations;
    \item \emph{Service provider}: either the organizations or a third party;
    \item \emph{Resource location}: either on-premises or off-promises;
\end{itemize}

\subsection{Hybrid cloud}
It's the composition of more deployment models which remain unique
entities, but are bound together by standardised or proprietary technologies
that enables data and applications portability.

For example an organization might use public cloud for some aspects of its
business and a private one for its sensitive data.
\chapter{Virtualization}
\section{Introduction}
Before virtualization tooked place, companies used to have varius servers but
most of the time they were found to be idle. The problem was, that due to OSs
failures, they couldn't run flawlessly more than one application at a time. In
particular, OSs couldn't provide:
\begin{itemize}
    \item\emph{Full isolation of configurations and shared components}: for
    example an app requiring version 1.0 of some library, created conflicts
    with another app requiring a different version for the same library;
    \item\emph{Temporal isolation for performance predictability}: it could
    happen that one app used a lot of resources causing the degradation of
    performances for another app;
    \item\emph{Strong spacial isolation for security and reliability}: if
    some app crashed it might have compromised others;
\end{itemize}
All of this lead to companies needing to have a lot of different servers running
even if they were massively underutilized and were consuming a lot of power.

Computing virtualization established because it offered a flexible way to
share hardware resources between different operating systems. This came in hand
with both advantages and disadvantages.

\paragraph{Advantages}
\begin{itemize}
    \item\emph{Isolation}: critical applications can run in different and
    easily isolated OSs. Also, different services can run in the same host,
    into different \emph{virtual machines} that could even use different CPU cores;
    \item\emph{Consolidation}: different OSs can run at the same time on the
    same hardware, thus saving resources and minimysing costs and energy
    consumption;
    \item\emph{Fleibility and agility}: the system admin has complete control
    over \emph{virtual machines} execution, and it can pause and restart them.
    Moreover, it might migrate one to a different host, or duplicated it to address a
    workload peak. Finally, it's easy to recover from a disaster (e.g. restarting
    a VM from a safe snapshot) or spawn a new \emph{virtual machine};
\end{itemize}
\paragraph{Disadvantages}
\begin{itemize}
    \item\emph{Additional overhead}: since each \emph{virtual machine} needs it
    own OS, more hardware resources are required;
    \item\emph{Increased difficulty in handling different hardware}: it might
    be difficult for the virtualization manager to grant some application access
    to special components;
\end{itemize}
Virtualization can be used for both server and dektop virtualization, but its
main usage is in server virtualization. In fact, since more OSs can
run on the same physical machine using a configurable amount of resources, it is
no longer necessary for system admins to buy machines with specific physical
characteristics. Instead, they can just buy \emph{COTS} (Common Off The Shelf)
hardware and, on top of which, create different \emph{virtual machines} with their
required specifications. This is convinient because companies can buy tons of
equivalent servers, put them into a datacenter and virtualise their resources.
Also, buying in large volumes often results in a lower individual price.

\subsection{Some definitions}
Before diving into more technical aspects of virtualization, let's give some
definitions.

\begin{definition}[Layering]
    Layering is a common approch to manage system complexity which allows
    to minimize the interactions among the subsystems of a complex system. The
    description of those subsystems is also simplified because each of them is
    abstracted through its interface to the others. Finally, layering allows
    to manage each subsystem individually.
\end{definition}\noindent
For example, a computer can be divided into two main layers: hardware and software,
and software can then be divided into OS, libraries and applications.
The interfaces between software layers are ISA, ABI and API.

\begin{definition}[Virtual machine (VM)] Virtual machines are software emulation
    of a physical machine that executes both OS and applications as if they were
    executed on a physical machine.
\end{definition}\noindent
When taking about \emph{virtual machine} we need to distinguish between two actors:
\begin{enumerate}
    \item\emph{Host OS}: it's the OS that is running on the physical machine and
    that is handling virtualization;
    \item\emph{Guest OS}: it's the OS running on a \emph{virtual machine}, and it
    shouldn't be aware of being running in a virtualised enviroment;
\end{enumerate}

\begin{definition}[Hypervisor]
    The hypervisor is the software in charge of the virtualization process,
    meaning that it has to virtualise the hardware resources and this is done by:
    \begin{itemize}
        \item Assigning, when possible, a specific set of resources to each
        virtual machine while garanteeing that each of them doesn't get access
        outside its boundaries;
        \item Arbitring access to shared resources that cannot be partitioned;
    \end{itemize}
\end{definition}
\begin{note}
    The \emph{hypervisors} is often refered to also as \emph{Virtual Machine
    Manager} (\emph{VMM}).
\end{note}
\noindent
The \emph{hypervisor} is often implemented as a Linux-based stripped-off OS
(i.e. an OS with minimum functionality) to make it more efficient
and more easily securable. The \emph{hypervisor} exports also a set of
\q{standard} devices to hosted OSs (i.e. the most common pieces of hardware
that are supported from most OSs).

The \emph{hypervisor} must provide a \q{virtual hardware} to \emph{guest
OSs} with the exact characteristics specified in a given hardware profile.
Also, the real hardware may be different from the virtualised one because it
depends on the devices that are exposed by the \emph{hypervisor}.

To allow \emph{guest OSs} to run in a virtualised enviroment, CPU, memory and
I/O need to be virtualised correctly.

\section{CPU virtualization}
The \emph{VMM} assigns one or more CPU cores to the VMs so that they can run
their OS. The ISA of the virtualised hardware will usually be the same
of the physical one, but it is not mandatory. Basically, if they're different
there will be an emulation process that will translate messagges between them.
However, since the emulation process works by doing a binary translation between
the two different ISAs, it is too slow to be generally convinient.

Going back to the \emph{VMM}, it must satisfy three characteristics:
\begin{itemize}
    \item The execution enviroment exported by it must be identical to the
    physical one, so that OSs can run unmodified;
    \item It must have complete control over real system resources, so that
    any \emph{guest OS} can access only those components it has been grant access
    to;
    \item It must run the virtualised systems efficiently;
\end{itemize}

\subsection{Some definitions}
System based on x86 or x64 architecture are usually model into a \emph{privileg
ring} structure. In particular, there are four privileg levels with decreasing
privileges as you move away from the center. In fact, \emph{ring 0} is dedicated
to the OS kerner, and \emph{ring 3} to generic applications.

\begin{figure}[h!]
    \centering
    \img{privileged-ring-model.png}{0.4}
    \caption{\emph{Privilege-ring model}}
\end{figure}

\noindent
Virtualization can use \q{ring de-priviledging}, a technique that runs
\emph{guest OSs} in level greater than \emph{ring 0} so that they have
limited privileges and therefore can't interfere with each other or with the
\emph{VMM}. However, there
are two possible modes:
\begin{itemize}
    \item\emph{0/1/3}: the \emph{VMM} runs at \emph{ring 0}, \emph{guest OSs}
    at \emph{ring 1} and applications at \emph{ring 3}. Since, in x86 architeture,
    some privileges with respect to memory accesses are granted to \emph{ring 0-2},
    \emph{guest OS} might still interfere with the \emph{VMM};
    \item\emph{0/3/3}: the \emph{VMM} runs at \emph{ring 0}, \emph{guest OSs}
    and applications at \emph{ring 3}. This solves the previous problem, but
    \emph{guest OSs} are no longer protected by malicious applications;
\end{itemize}

\begin{definition}[Privileged instruction]
    A privileged instruction is a CPU instruction that needs to be executed in
    a privileged hardware context.
\end{definition}
\begin{definition}[Sensitive instruction]
    A sensitive instruction is a CPU instruction that can leak information about
    the physical state of the processor.
\end{definition}
\begin{note}
    \emph{Sensitive instructions} are, for example, those that can read the
    register in which is stored the currect CPU privileg level.
\end{note}\noindent
To be virtualizable all CPU's \emph{sensitive instructions} must be
\emph{privileged}.

\begin{definition}[Trap]
    A trap is an event that triggers the switch from an unprivileged context to
    a priviliged one.
\end{definition}\noindent
If a \emph{privileged instruction} is called while the CPU isn't running in
kernel mode (the mode associated to \emph{ring 0}), a \emph{trap} is generated,
so the CPU jumps to the \emph{Hardware Exception Handler Vector} (HEHV) and
executes said instruction in kernel mode.

Situations in which a \emph{trap} can occur can be put in one of three buckets:
\begin{enumerate}
    \item\emph{Exceptions}: invoked when unexpected error or system mulfunction
    occur (e.g. \emph{privileged instruction} executed in user mode);
    \item\emph{System call}: invoked by applications in user mode (e.g.
    application asking OS for system I/O);
    \item\emph{Hardware interrupts}: invoked by hardware events in any mode
    (e.g. hardware clock timer triggers events);
\end{enumerate}

\paragraph{System call invocation and hardware interrupts}
In traditional OSs, when an application invokes a \emph{system call}, the CPU
will trap to interrupt handler vector in OS, then will switch to kernel mode
and execute OS instructions.

Similarly, when an \emph{hardware interrupt} verifies, CPU execution will stop
and it will jump to interrupt handler in OS.

\begin{figure}[h!]
    \centering
    \img{trap-handling-traditional-OSs.png}{0.25}
    \caption{\emph{Trap handling} in traditional OSs}
\end{figure}

\noindent
Diving deeper into \emph{trap handling}, when a \emph{trap} is generated,
userland code (i.e. code outside the kernel) generates a \emph{software
interrupt} (e.g. thorugh the instruction \texttt{INT xx}). Hence, the generic
interrupt routing of the OS is started, and it determines where to jump in the
OS code to serve that interrupt. Finally, kernel jumps to the identified code,
serves the interrupst and then returns control back to the caller (i.e.
instruction \texttt{IRET}). All of this requires to load and parse the content
of several memory locations, so it's rather slow.

A more modern way to serve interrupts uses \texttt{SYSENTER} and
\texttt{SYSEXIT} instructions (\texttt{SYSCALL} and \texttt{SYSRETURN} in
x64 systems) to speed up the process. Practically, userland code writes the
address of the targeted kernel routine in a specific register, then
\texttt{SYSENTER} is called and the kernel jumps to the selected routing
reading the address from the register without additional accesses to memory.

\paragraph{Going back to virtualization}
Said this, we can go back to virtualization and talk about the three types
of virtualization that exists:
\begin{enumerate}
    \item\emph{Full virtualization}: \emph{guest OSs} can run unmodified;
    \item\emph{Paravirtualization}: \emph{guest OSs} are aware of being running
    in a \emph{virtual machine}, so they need to be modified;
    \item\emph{Hardware assisted virtualization}: the \emph{hypervisor} exploits
    some functionalities provided by modern CPU chips;
\end{enumerate}

\subsection{Trap \& emulate paradigm}
This paradigm allows \emph{full virtualization} and provides that \emph{guest
OSs} run in an unprivileged enviroment, hence when a \emph{privileged
instruction} has to be executed, a \emph{trap} is launched by the CPU. Then,
that \emph{trap} is intercepted by the \emph{VMM} that emulates the effect of
the \emph{privileged instruction} for the caller (ofcourse only if it's
legitimate) and, at the end, gives control back to \emph{guest OS}.

Actually, when the \emph{VMM} intercepts a \emph{trap} it behaves differently
based on the event that caused it. If it was causes by an application, then the
\emph{VMM} passes it direcctly to the \emph{guest OS}. On the other hand, if it
was caused by the \emph{guest OS} itself, the \emph{MVV} handles it by
modifying the state of the \emph{virtual machine}.

\paragraph{System call handling}
\mbox{}

\bigskip\noindent
\begin{minipage}[t]{0.48\textwidth}
    When a \emph{system call} happens, CPU traps it to interrupt handler vector
    of the \emph{VMM}. This then jumps back to the \emph{OS}. All of this,
    results in extra context switch operations and performance deteriorates
    further if the \emph{guest OS} isn't able to handle the interrupt
    routing. So, the time spent to execute a single \emph{system call} might
    be 10 times greater than what it would have been required by the \emph{host
    OS}.
\end{minipage}
\begin{minipage}[t]{0.48\textwidth}
    \centering
    \strut\vspace*{-\baselineskip}\newline
    \img{trap-handling-trap-and-emulate1.png}{0.8}
\end{minipage}

\newpage
\paragraph{Privileged instruction handling}
\mbox{}

\bigskip\noindent
\begin{minipage}[t]{0.48\textwidth}
    When a \emph{privileged instruction} is executed it will be trapped to
    \emph{VMM} who will emulate it. After that, \emph{VMM} will give control back
    to the caller.
\end{minipage}\hfill
\begin{minipage}[t]{0.48\textwidth}
    \centering
    \strut\vspace*{-\baselineskip}\newline
    \img{trap-handling-trap-and-emulate2.png}{0.8}
\end{minipage}

\paragraph{Hardware interrupt handling}
\mbox{}

\bigskip\noindent
\begin{minipage}[t]{0.48\textwidth}
    When a \emph{hardware interrupt} is launched, CPU will trap it to interrupt
    handler of \emph{VMM} that then will jump to the corresponding interrupt
    handler of the \emph{guest OS}.
\end{minipage}\hfill
\begin{minipage}[t]{0.48\textwidth}
    \centering
    \strut\vspace*{-\baselineskip}\newline
    \img{trap-handling-trap-and-emulate3.png}{0.8}
\end{minipage}

\paragraph{Problems with systems virtualization}
In this paradigm, each time a \emph{privileged instruction} is executed in an
unprivileged context, a \emph{trap} has to be generated and detected by the
\emph{VMM}. These actions are time-consuming, as we have just seen.

However, this process isn't necessary for all architetures, but unfortunatly,
it is in x86 and x64 architectures. Moreover, these architectures presents
some \emph{sensitive instructions} (e.g. \texttt{POPA}, \texttt{POPF}) that
don't trap when executed in an unprivileged context, hence, these architectures
are said to be \q{non-virtualizable}.

\paragraph{Possible solutions}
Therefore, we have some \emph{sensitive instructions} that don't trap and,
consequently, the \emph{VMM} cannot emulate the correcct behavior during the
execution.

To address this problem we can change virtualization paradigm or introduce
some code into the \emph{VMM} that parses the instruction stream to detect
all \emph{sensitive instructions} dynamically. Then, we can both use
interpretation and binary translation. Interpretation in an old and slow approch
in which emulating a single ASM instruction originates an overhead of one order
of magnitude at least. Binary translation, on the other hand, introduces a
lower performance overhead.

\paragraph{Dynamic binary translation}
The idea behind this approch is to dynamically translate \q{non-virtualizable}
ISA to a virtualizable one during run time. In particular, dynamic means that
translation is done on-the-fly at execution time and interleaved with normal
code execution. Binary means that \emph{VMM} translates binary code instead of
source code and this is more efficient.

The pros of this technique is that it still allows \emph{full virtualization}
without needing specific hardware support, but virtualization overhead is still
too high and several instructions or execution patterns (e.g. \emph{system call})
are significantly slower than real execution.

\begin{note}
    We could use caching techniques to recognize significant instruction patterns
    and increase translation speed.
\end{note}
\begin{note}
    Original VMware \emph{VMM} combined \emph{Trap \& emualte} with a system
    level \emph{Dynamic Binary Translation}. \emph{Guest OSs} run at \emph{ring 1}
    and \emph{VMM} inspected dynamically code to swap non-trappable portions of
    code with \q{safe} instructions.
\end{note}

\subsection{Paravirtualization}
The idea that drives this paradigm is to let \emph{guest OSs} know that
they're running in a virtualised enviroment and that, in some case, they'll
have to leave control to a \emph{VMM}. So, \emph{guest OSs} are explicitly
modified to be virtualizable, changing the interface provided to make it
easier to implement.

In particular, \emph{system calls} and \q{non-virtualizable instructions} are
replaced with specific \emph{hypervisor calls} (\emph{hypercalls}). Hence,
they won't trap anymore and all the \emph{trap \& emulate} process is also
removed. Ofcourse, modifications don't affect the \emph{ABI}, so applications
can be executed without further changes.

\emph{Guest OSs} are explicitly deprivileged meaning that they know they're being
executed at \emph{ring 1}. This allows the introduction in \emph{guest OSs}
kernel of efficient mechanisms that ease the communication with the
\emph{hypervisor}:
\begin{itemize}
    \item\emph{Guesto-to-Hypervisor}: \emph{privileged instructions} are replaced
    with synchronous paravirtualised equivalent \emph{hypercalls};
    \item\emph{Hypervistor-to-Guest}: \emph{hypervisor} can notify certain events
    asynchronously to the \emph{guest};
\end{itemize}

\bigskip\noindent
Talking about pros and cons in a paravirtualized enviroment, only modifiable
OSs can be used, because it's necessary to access their source code.

Performances are surely higher than those of \emph{Trap \& Emulate paradigm}
because neither emulation nor translation are required, hence \emph{MVV}
implementation is also simpler and faster.

\subsection{Hardware assisted virtualization}
Up to this point, we still have two unsolved issues: complex implementati of
\emph{MVV} and necessity to provide full virtualization for most of x86 and x64
systems (most of them weren't still modifiable).

\emph{Hardware assisted virtualization} aims at providing a solution to those
problems by proposing an efficient \emph{Trap \& Emulate} approach to
virtualization thanks to an additional hardware support.

This is based on the idea of avoiding \emph{sensitive instructions}, either
because they can be \q{promoted} to \emph{privileged} or bacause the \emph{MVV}
can dynamically configure which instructions have to be trapped. There are some
instruction then, that cause \emph{virtual machines} to exit unconditionally
(e.g. \texttt{INVD} instraction for CPU internal cache invalidation) and therefore
can never be executed in a virtualised non-root enviroment. Finally, all
events and some other instructions can be configured to operate conditionally
using \emph{virtual machine} execution control fields.

To do all of this, processors are provided with an additional running mode
named \emph{Virtual Machine eXtensions} (\emph{VMX}). When this mode is enabled,
CPU will activate two different running modes called \emph{operating levels}
that are: \emph{non-root mode} and \emph{root mode}. These modes still works
with the usual \emph{privileg ring} structure.

\begin{figure}[h!]
    \centering
    \img{hav.png}{0.4}
    \caption{\emph{Hardware assisted virtualization} modes}
\end{figure}\noindent
The \emph{VMM} runs al \emph{ring 0} in \emph{root-mode}, while \emph{guest OSs}
run at \emph{ring 0} in \emph{non-root mode}. Ofcourse applications still run
at \emph{ring 3} in \emph{non-root mode}.

\paragraph{VMX instructions}
If sytem code tries to execute instructions violating isolation of the \emph{VMM}
or that must be emulated via software, hardware traps it and switches back to
the \emph{VMM}. CPU enters \emph{non-root mode} via the new \texttt{VMLAUNCH}
and \texttt{VMRESUME} instructions, and it returns to \emph{root mode} for a
number of reasons, collectively called \emph{VM exits}.

\emph{VM exits} should return control to the \emph{VMM}, which should complete
the emulation of the action that the guest code was trying to execute, then
give control back to the \emph{guest} by re-entering \emph{non-root mode}. All
the new \emph{virtual machine} instructions are only allowed in \emph{root mode}.

For example, while in \emph{non-root mode}, \texttt{INT xx} instruction may cause
a swich from \emph{non-root user mode} to \emph{non-root kernel mode}, and
\texttt{IRET} may return from \emph{non-root kernel mode} back to \emph{non-root
user mode}.

\bigskip\noindent
So, when a trapping condition is triggered, \emph{VMM} takes control of the
execution and emulates the correct behaviour. Transition between \emph{root}
and \emph{non-root mode} is realized through:
\begin{itemize}
    \item\emph{VM entry}: from \emph{VMM} to \emph{guest};
    \item\emph{VM exit}: from \emph{guest} to \emph{VMM};
\end{itemize}\noindent
When this happens, registers and address spaces are swapped in a single atomic
operation and, as we would expect, this remains the main source of overhead.

\paragraph{Virtual machine control structure}
To mantain \emph{virtual machines} state and control information \emph{VMM}
uses a particular structure called \emph{Virtual Machine Control Structure}
(\emph{VMCS} or \emph{VMCB}). It concretely represents the control panel of
the \emph{virtual machine}, storing information about \emph{guest} state,
\emph{host} processor and control data (e.g. trapping condition). It also
mirrors all registers modifications needed to set a certain configuration in
\emph{guest OSs}. \emph{VMCS} introduced dedicated instructions to modify it:
\texttt{VMWRITE} and \texttt{VMREAD}.

\begin{figure}[h!]
    \centering
    \img{vmcs.png}{0.4}
    \caption{\emph{Virtual Machine Control Structure}}
\end{figure}

\section{Memory virtualization}
\subsection{Memory management in general}
Modern operating systems use a \emph{memory paging} technique to access, as
contiguous, dispersed locations in the physical memory. In particular, the main
memory (RAM) is divided into frames of fixed size. The OS assigns each process
one or more pages that are the same size as the frames, so the address space of
processes spans across multiple frames which aren't necessarily contiguous, but
pages are, so processes can behave as if their address space were unitary.

Therefore, there is a difference between virtual or logical address that referes
to pages, and physical addresses that referes to physical memory. Processes
use only virtual addresses, so when they need to access memory, logical address
have to be translated into physical ones. This translation is done by the
\emph{Memory Management Unit} (\emph{MMU}), a unit that resides in the CPU.

Operating systems maintain a page table for each process in which every line
holds information about one page. In particular, each row associates the
\emph{Logical Page Number} (\emph{LPN}) to the physical one called \emph{PPN}.
When a logical address is accessed the \emph{MMU} walks all these page tables
to determine the corresponding \emph{PPN}, and thus, determining the frame physical
addres too.

\begin{figure}[h!]
    \centering
    \img{page-tables.png}{0.6}
    \caption{Memory paging}
\end{figure}

\noindent
In the case of big page tables, the \emph{MMU} can use a \emph{Translation
Lookaside Buffer} (\emph{TLB}) that works as a cache for recently used page
translations. The \emph{TLB} works as a fully associative memory in which
\emph{LPNs} are used as keys to get the corresponding \emph{PPNs}.
\begin{note}
    \emph{TLB} works similarly to a hash map.
\end{note}

\noindent
There are mainly three reasons for which modern operating systems choose to use
\emph{memory paging}:
\begin{enumerate}
    \item \emph{Simplicity}: every process gets the illusion af a whole address
    space;
    \item \emph{Isolation}: every process address space is strictly separated
    from others;
    \item \emph{Optimization}: it is possible to exploit \emph{swapping} to allow
    operating systems to handle more pages than what the physical memory alone
    could hold;
\end{enumerate}

\subsection{Memory management in virtualised environments}
In a virtualised environment, \emph{Guest OSs} don't have direct access to memory,
so what they percive as physical addresses are in fact virtual ones. This means
that page tables of processes running in the VM associates \emph{Logical Page
Numbers} of the processs to \emph{Physical Page Numbers} of the virtual machine,
which in turn are associated to \emph{Physical Page Numbers} of the physical
machine. For this reason, translating a logical address of a VM's process would
require two steps:
\[\begin{array}{lrcl}
    1. & \emph{Guest Logical Address} & \to & \emph{Guest Physical Address}\\
    2. & \emph{Guest Physical Address} & \to & \emph{Machine Physical Address}
\end{array}\]
To avoid this, a \q{shadow page table} is introduced. This table stores and keeps
track of the mapping between \emph{Guest Logical Addresses} and \emph{Machine
Physical Addresses}. It is invisible from the \emph{guest} point of view
because is maintained by the \emph{VMM}, who also exposes it to the \emph{MMU}.

Going into more details, the association between \emph{Physical Page Numbers}
and \emph{Machine Page Numbers} (\emph{MPNs}) is maintained by the \emph{VMM}
in internal data structures, while the association between \emph{LPNs} and
\emph{MPNs} is stored by the \emph{VMM} in the \q{shadow page table} that is
exposed to the \emph{MMU}.

\begin{figure}[h!]
    \centering
    \img{shadow-page-tables.png}{0.6}
    \caption{Memory paging in virtualised environments}
\end{figure}

\noindent
It is still possible to cache most recently used translation between \emph{LPN}
and \emph{MPN} in a \emph{TLB}.

\paragraph{Shadow page table creation}
Each \emph{Guest OS} maintains the associations between \emph{LPNs} and
\emph{PPNs} as seen before, but when it tries to access a physical address,
since it isn't running directly on the hardware, the request is trapped by the
\emph{VMM}. Then, the \emph{VMM}, who already knows what \emph{MPN} is bound to
that \emph{PPN}, saves the original \emph{LPN} is the shadow table bounding it
to the correct \emph{MPN}.

\paragraph{Problems with shadow page table}
Ofcource the \emph{VMM} is in charge of keeping the shadow page table
synchronized with the \emph{Guest OS} page tables. So, an extra overhead is
introduced, and it becomes a problem if some applications force \emph{Guest OS} to
update them frequently (e.g. some apps might couse many page faults).

\subsection{Hardware assisted memory virtualization}
To avoid that extra overhead, hardware manufacturer implemented a type of hardware
that allows the mapping between \emph{LPNs} and \emph{PPNs}, maintained by
\emph{Guest OS}, to coexist together with the mapping between \emph{PPNs} and
\emph{MPNs}, created by the \emph{VMM}, in the same page table.

In particular, the translation to \emph{MPNs} is put in an additional nested level
of page tables. Both the traditional and the nested tables are exposed to the
hardware, so that, when a logical address is accessed, the hardware walks the
guest page tables as in the case of native execution (no virtualization), but for
every \emph{PPN} accessed during the process, the hardware also walks the
nested page tables to determine the corresponding \emph{MPN}.

\begin{figure}[h!]
    \centering
    \img{page-tables-HW.png}{0.6}
    \caption{Hardware assisted memory virtualization}
\end{figure}

\noindent
This approch removes the need for the \emph{VMM} to keep synchronized additional
tables, thus removing the previously discussed overhead. However, since the
hardware has to waltk through two tables to translate every address, the cost of
every translation is increased. For this reason, \emph{TLB} becomes critical to
guarantee good performance and, for memory intensive tasks, having larger pages
might increase the \emph{TLB} \q{hit ratio}.

\paragraph{Tagged TLBs}
An additional optimization that can be implemented to increase \emph{TLB} hit
ratio is represented by \emph{tagged TLBs}. Adding a tag means adding to each
\emph{TLB} line a Virtual Processor ID, that is an identifier for each virtual
processor. This prevents wrong access to other virtual processors cache lines,
thus allowing multiple virtual processors to coexist on the \emph{TLB} at the
same time.

Previously in fact, \emph{TLB} needed to be flushed on each \emph{VM exit} and
\emph{VM entry} beacuse virtual machines addresses, both \emph{LPN} and \emph{PPN},
aren't globally unique and keeping them in the \emph{TLB} would have created
conflicts and accesses to memory areas of other VMs.

\section{I/O virtualization}
There are various techniques to implement I/O virtualization: \emph{device
emulation}, \emph{paravirtualization} and \emph{direct assignment}. Unlike CPU and
memory, I/O devices might be assigned to just one or some VMs.

\subsection{Device emulation}
With \emph{device emulation} the \emph{VMM} proposes to the \emph{Guest OS} an
emulated device which implements in software an hardware specification.
\emph{Guest OS} uses that device without knowing that it is being emulated and
to do so, it uses the same drivers used with an equivalent physical device.

This is a simple approch that doesn't require \emph{Guest OSs} to install
dedicated drivers, and a single physical device could be multiplexed into
multiple emulated devices. Ofcourse, the \emph{VMM} has to remap the communication
with the physical device. Then, I/O operations are generally slower than the
physical ones and with higher latency, especially in case of devices with high
I/O (e.g. NIC, disk). Also, since CPU has to emulate each request, its workload
might increase substantially.

\subsection{Paravirtualized devices}
Unlike \emph{CPU paravirtualization} that required kernel modifications, to
paravirtualize I/O we just need to write new drivers that can than be added as
external modules to the OS.

Paravirtualized drivers are a convenient solution that also allows further
optimization such as \q{memory balooning}. When creating a virtual machine, the
\emph{VMM} defines its memory size and allocates it statically. To obtain a
dynamic and more efficient use of memory, the \emph{VMM} can exploit memory
balooning paravirtualized drivers installed by \emph{Guest OSs}. Those drivers
proive the \emph{VMM} with information about current memory occupation of the
guests, allowing it to change the amont of memory allocated to those VMs and
providing it to others.

\subsection{Direct assignment}
With \emph{direct assignment} a device is exclusively assigned to one VM that
can directly communicate with it without needing any driver apart from the
traditional ones of the device. The device is totally handled by the \emph{Guest
OS}; hence it can be multiplexed over several virtual machines.

Despite seeming very simple, this approch is very complex indeed, because it
raises critical issues on memory usage. Direct memory access (DMA) has to be
performed on the physical address space of the \emph{Guest OS}, but the device
doesn't know the mapping between \emph{Guest} and \emph{Host} physical addresses.
This could potentially lead to memory corruption and to avoid it the \emph{VMM}
has to intercept the I/O operations and perform the correct translation. The
problem is that this is slow and can introduce a significant overhead in I/O
operations.

\paragraph{Hardware assisted direct assignment}
As seen before, hardware manufacturer can implement technologies to ease the
virtualization process.

\begin{figure}[ht!]
    \centering
    \subfloat[\emph{IOMMU} extension]{\img{iommu.png}{0.2}}
    \hspace{2cm}
    \subfloat[\emph{SR-IOV} implementation model]{\img{sr-iov.png}{0.23}}
    \caption{\emph{Hardware assisted direct assignment} possibilities}
\end{figure}

\noindent
One solution to the memory problem discussed above
is the introduction of an \emph{Input Output Memory Management Unit}
(\emph{IOMMU}), an extension that can boost and make direct access to memory
easier to be implemented in \emph{VMMs}. Like to \emph{MMU}, the \emph{IOMMU}
remaps the addresses accessed by the hardware according to the same table used
to map \emph{PPNs} to \emph{MPNs}, allowing direct memory access cycles to
safely access the correct memory locations.

As for the networks card instead, the \emph{PCI-e} standard defines
\emph{Single Root Input/Output Virtualization} (\emph{SR-IOV}) as a mechanism
to allow several directly assigned devices to be shared among VMs. \emph{SR-IOV}
defines the possibility for the devices to present several virtual devices,
\q{virtual functions} to the OS. The \emph{VMM} will directly assign each
virtual function to each VM and the hardware will handle multiplexing by
itself.

\section{Hypervisors architectures}
\emph{Hypervisors} can be based upon two architectures that pursue two
distinc objectives: performance the first and easiness of deployment
and utilization the second. Ofcourse hybrid implementations exist.

\subsection{Type 1 architecture}
The \emph{hypervisor} runs directly on bare metal, so there isn't any extra
layer between the hardware and it. Normally, it's able to provide the best
performace. However, the \emph{hypervisor} needs to be implemented as a
stripped-off OS with basic functionalities and, thus, there might be problems
with drivers. As we already said, \emph{hypervisors} need to have basic
functionatilties to be less prone to bugs and attacks.

\paragraph{Examples} Microsoft Hyper-V

\subsection{Type 2 architecture}
The \emph{hypervisor} runs on top of an OS as a privileged process, so it's
easier to install but less performing.

\paragraph{Examples} VirtualBox

\begin{note}
    Systems with dual boot where a normal OS resides together with a \emph{type
    1 hypervisor} are an example of how an hybrid approch works.
\end{note}

\subsection{Hybrid architecture}
\emph{Hybrid hypervisors} are implemented as a component of the OS kernel.
So, the \emph{Host OS} is itself the \emph{hypervisor}, but also works as a
normal OS. This makes this kind of \emph{hypervisors} easy to install and deploy
bacause drivers and support comes from the mainstream OS. Performance can also
be very good.

\paragraph{Examples} KVM

\begin{figure}[hb!]
    \centering
    \img{vmm-architectures.png}{0.45}
    \caption{\emph{Hypervisor} architectures in comparison}
\end{figure}

\section{OS-level virtualization}
Going back to \emph{full virtualization} we can summarize its pros and cons as
it follow:

\noindent
\begin{minipage}[t]{0.48\textwidth}
    \textbf{Advantages}
    \begin{itemize}
        \item Compatible with existing applications;
        \item Supports different OSs;
        \item Each VM can have its own execution environment;
        \item The isolation backed by hardware is execellent;
    \end{itemize}
\end{minipage}
\hfill
\begin{minipage}[t]{0.48\textwidth}
    \textbf{Disadvantages}
    \begin{itemize}
        \item Running each \emph{Guest OS} requires additional overhead;
        \item It's necessary to configure and keep updated each instance of
        \emph{Guest OS};
        \item OS booting time (e.g. seconds or more) might not be acceptable;
    \end{itemize}
\end{minipage}

\bigskip\noindent
Before cloud took place, datacenters stored lots of servers which were meant to
run different OSs to meet different requirements (e.g. desktop environments for
real users, support for specific peripherals). However, with the spreading of
cloud computing, hardware became a commodity and interaction with users is
provided by web applications instead of desktop environment. Hence, we could
achive great operational efficiency if we reduced the number of OSs to just one:
Linux.

\begin{note}
    From now on, we will only discuss mechanism and approches used by Linux-based
    operating systems.
\end{note}

\subsection{Lightweigth virtualization}
In this context the idea of \emph{lightweight virtualization} was born. It aims
to the creation of a system in which all the advantages of \emph{full
virtualization} are guaranteed, but resource consumption is much less concerning.

\emph{Lightweigth virtualization} is therefore appropriate when there is no need
for a classical VM or when its overhead is unacceptable. Also, when we'd like to
have an isolated environment that is quick to deploy, migrate and dispose with
little or no overhead, or when we want to scale both vertically (i.e. many
isolated environments on the same machine) and horizontally (i.e. deploy the
same environment on many machines), \emph{lightweight virtualization} can be a
good solution.

\bigskip\noindent
Going deeper into \emph{lightweigth virtualization} characteristics, with it,
we use \emph{OS-level} or \emph{application level virtualization} instead of
\emph{full virtualization}. In particular, with \emph{OS-level virtualization},
the \emph{hypervisor} is the Linux kernel itself.

As already mentioned, classical VMs are replaced by isolated environments (i.e.
virtual private servers, jails, containers) and each one of them features a
given extent of resources management and isolation that usually is less than
what can be guaranteed by classical VMs. Finally, applications can be executed
inside these environments.

A good \emph{lightwiegth virtualization} implementation must provide a
fine-grained control of resources of the physical machine, allowing sysadmins to
partition and control resources among different isoltated environments. Another
requirement is on security and isolation, meaning that each environment should
be assigned to one app or user, and it should prevent a misbehaviour in one
environment from affecting others. Finally, it should be possible to manage an
entire datacenter as a unique entity, such as with cloud toolkits; even better,
the capability to integrate \emph{lightweigth virtualization} with a cloud
toolkit in order to have the flexibility to deploy VMs, containers and such upon
requests, should be provided.

\paragraph{Why is process isolation so important?}
Community recognized the need to implement strong process isolation in Linux
kernel because servers running multiple services want to be sure that possible
intrusions on some services don't affect others. Also, it must be safe to run
arbitrary or unknown software on a server (e.g. students code, hakaton, testing
environment).

\bigskip\noindent
How can all of this be done without adopting techniques such as hardware
virtualization that generates too much overhead?

In theory many possibilities exists, but practically only a few answers the
questions: Linux containers (LXC) and LXC-based software. Other technologies
used are Linux \emph{cgroups} and \emph{namespaces}, that were created to
strengthen processes isolation without thinking to virtualization, but can be
leveraged to create a form of \emph{lightweigth virtualization} with minimun
overhead.

\subsection{Linux cgroups}
\chapter{Cloud networking}
\section{Data centers networks}
Unquestionably, networking is the core of \emph{cloud computing}, in fact, it
has been made feasible by the continuous evolution of internet connectivity.
Historically, networks have been classified into three layers:
\begin{itemize}
    \item \emph{Tier 1}: core part of network infrastructure that allows all
    other networks on the internet to communicate each other;
    \item \emph{Tier 2}: portions of \emph{tier 1} infrastructure which is
    bought by internet service providers which then sold access to it to customers;
    \item \emph{Tier 3}: part of the network infrastructure that allows final
    users to gain access to other networks;
\end{itemize}
\begin{note}
    \emph{Tier 2} networks communicated each other as peers through special nodes
    called Internet Exchange Points (IXP).
\end{note}

\begin{figure}[h!]
    \centering
    \img{network-tiers.png}{0.4}
    \caption{Internet infrastructure}
\end{figure}

\noindent
The transformation of the internet experience, with the shift towards web only
services (e.g. web applications instead of desktop applications), resulted in
a significant increase in volumes of shared data and required bandwidth. To
address this increase in bandwidth demand, cloud providers introduced Content
Delivery Networks (CDN) whose function is to bring service providers closer
to the final users. A secondo step was made with the spread of IoT and
decentralized computing paradigms such as \emph{fog computing}.

\begin{figure}[ht!]
    \centering
    \subfloat[Traditional infrastructure]{\img{traditional-network.png}{0.48}}
    \hfill
    \subfloat[Modern infrastructure]{\img{modern-network.png}{0.48}}\\
    \subfloat[Today's infrastructure]{\img{todays-network.png}{0.48}}\\
\end{figure}

\subsection{Characteristics of data center networks}
Data centers are huge structures with tons of servers, so they demand high
bandwidth with the outside and a very low RTT within their premises. Also,
each data center is a signle administrative domain, meaning that its
administrators have full control over its intern network infrastructure, 
endpoints and protocols. This allows them to deviate from standards and adopt
each kind of custom solutions, as long as the services they host are reachable.

This liberty becames relavant as soon as we realize that most of the traffic
in a data center is created by machine-to-machine communications, leaving
user-to-machine traffic a small percentage of the total amount. 

\begin{figure}[h!]
    \centering
    \img{data-center-traffic}{0.5}
    \caption{Traffic within a data center}
\end{figure}

\noindent
This means that the performance of the networks inside the data centers, which
are called \emph{interconnection networks}, are critical.

\paragraph{Interconnetion networks}
An \emph{interconnection network} is composed of node and link or communication
channels. Nodes can be servers, memory units or even processors. Then, each node
has an interface connecting it to the network, and the number of link the node
is connected defines its degree. The interconnection fabric is made of
switches\footnotemark and links. Switches receive packes, and look inside them
to determine the way toward their final destination. An $n$-way switch is a
switch with $n$ ports that can be connected to $n$ links. The interconnection
fabric determines whether the \emph{interconnection network} is \emph{blocking}
or not. It is \emph{non-blocking} if any permutation of source and destination
nodes can connect each other at any time. It is \emph{blocking} if this is not
true.

\footnotetext{We're refering to a generic switching device without saying if
is a layer 2 or layer 3 one}

\emph{Interconnection networks} can be distinguished by three parameters:
\begin{itemize}
    \item \emph{Topology}: defines the way nodes are interconnected;
    \item \emph{Routing}: defines how a message gets from source to destination;
    \item \emph{Flow control}: negotiates how the buffer space is allocated;
\end{itemize}
The \emph{topology} also determines the \emph{network diameter} and
\emph{bisection width}.

\begin{definition}[Network diameter]
    Network diameter is defined as the average distance between pairs of
    nodes.
\end{definition}
\begin{definition}[Bisection width]
    Bisection width is defined as the minimum number of links that have to be
    cut to partition the interconnection network into two halves.
\end{definition}

\noindent
When an \emph{interconnection network} is partitioned into two networks of
them same size, the bisection bandwidth measures the communication bandwidth
between the two. We talk about full bisection bandwidth when one half of nodes
can communicate simultaneously with the other half.

\paragraph{Topologies}
There are two main types of \emph{topologies}:
\begin{itemize}
    \item \emph{Static networks}: servers are connected through direct connections;
    \item \emph{Switched networks}: servers are connected through switches;
\end{itemize}

\begin{figure}[h!]
\centering
\subfloat[\emph{Bus topology}]{\resizebox*{0.48\textwidth}{!}{\begin{graph}
    \tikzset{
        rect/.style={rectangle, draw, inner sep=5mm},
    }
    \node[rect] (0) {Memory};
    \node[rect] (1) [right of=0, xshift=20mm] {Memory};
    \node[rect] (2) [right of=1, xshift=20mm] {Memory};
    \node[rect] (3) [right of=2, xshift=20mm] {Memory};

    \node[rect] (4) [below left of=0, yshift=-20mm] {Cache};
    \node[rect] (5) [right of=4, xshift=20mm] {Cache};
    \node[rect] (6) [right of=5, xshift=20mm] {Cache};
    \node[rect] (7) [right of=6, xshift=20mm] {Cache};

    \node[rect] (8) [below of=4, yshift=4.5mm, minimum width=20.9mm,
        minimum height=18mm] {Proc};
    \node[rect] (9) [below of=5, yshift=4.5mm, minimum width=20.9mm,
        minimum height=18mm] {Proc};
    \node[rect] (10) [below of=6, yshift=4.5mm, minimum width=20.9mm,
        minimum height=18mm] {Proc};
    \node[rect] (11) [below of=7, yshift=4.5mm, minimum width=20.9mm,
        minimum height=18mm] {Proc};

    \node[empty] (a) [below of=0, yshift=3mm] {};
    \node[empty] (b) [right of=a, xshift=20mm] {};
    \node[empty] (c) [right of=b, xshift=20mm] {};
    \node[empty] (d) [right of=c, xshift=20mm] {};
    \node[empty] (e) [right of=d, xshift=10mm] {};
    \node[empty] (f) [left of=a, xshift=-10mm] {};
    
    \node[empty] (g) [above of=8, yshift=12.6mm] {};
    \node[empty] (h) [right of=g, xshift=20mm] {};
    \node[empty] (i) [right of=h, xshift=20mm] {};
    \node[empty] (j) [right of=i, xshift=20mm] {};

    \draw[-]    (0) -- (a)
                (1) -- (b)
                (2) -- (c)
                (3) -- (d);

    \draw[-]    (4) -- (g)
                (5) -- (h)
                (6) -- (i)
                (7) -- (j);
    
    \draw[-, line width=1.3pt] (f) -- (e);

    \node[empty] (z) [below of=j, yshift=-45.5mm] {};
\end{graph}}}
\hfill
\subfloat[\emph{Hypercube topology}]{\resizebox*{0.48\textwidth}{!}{\begin{graph}
    \tikzset{
        rect/.style={rectangle, draw, fill, minimum size=3mm, inner sep=0},
        point/.style={circle, draw=gray, fill=gray, minimum size=2mm, inner sep=0},
        every path/.style={draw=gray}
    }
    
    \foreach \x in {0,3}
        \foreach \y in {0,-3}
        {
            \node[point] (a\x\y) at (\x, \y) {};
            \node[rect] (r\x\y) [below right of=a\x\y, shift={(-6mm, 6mm)}] {};
            \node[point] (b\x\y) [below left of=a\x\y, shift={(2mm, 2mm)}] {};
            \node[rect] (r1\x\y) [below right of=b\x\y, shift={(-6mm, 6mm)}] {};

            \draw[-]    (a\x\y) -- (b\x\y)
                        (a\x\y) -- (r\x\y)
                        (b\x\y) -- (r1\x\y);
        }

    \draw[-] (a00) -- (a30) -- (a3-3) -- (a0-3) -- (a00);
    \draw[-] (b00) -- (b30) -- (b3-3) -- (b0-3) -- (b00);

    \foreach \x in {6,9}
        \foreach \y in {0,-3}
        {
            \node[point] (c\x\y) at (\x, \y) {};
            \node[rect] (r2\x\y) [below right of=c\x\y, shift={(-6mm, 6mm)}] {};
            \node[point] (d\x\y) [below left of=c\x\y, shift={(2mm, 2mm)}] {};
            \node[rect] (r3\x\y) [below right of=d\x\y, shift={(-6mm, 6mm)}] {};

            \draw[-]    (c\x\y) -- (d\x\y)
                        (c\x\y) -- (r2\x\y)
                        (d\x\y) -- (r3\x\y);
        }

    \draw[-] (c60) -- (c90) -- (c9-3) -- (c6-3) -- (c60);
    \draw[-] (d60) -- (d90) -- (d9-3) -- (d6-3) -- (d60);
    \draw[-, bend left=25]
        (a00) edge (c60)
        (a30) edge (c90)
        (b00) edge (d60)
        (b30) edge (d90);
    \draw[-, bend right=25]
        (a0-3) edge (c6-3)
        (a3-3) edge (c9-3)
        (b0-3) edge (d6-3)
        (b3-3) edge (d9-3);
\end{graph}}}
\end{figure}
\newpage
\begin{figure}[ht!]
\ContinuedFloat
\centering
\subfloat[\emph{2D-mesh topology}]{\resizebox*{0.48\textwidth}{!}{\begin{graph}
    \tikzset{
        rect/.style={rectangle, draw, fill, minimum size=3mm, inner sep=0},
        point/.style={circle, draw=gray, fill=gray, minimum size=2mm, inner sep=0},
        every path/.style={draw=gray}
    }
    
    \foreach \x in {0, 3, 6, 9}
        \foreach \y in {0,-3, -6, -9}
        {
            \node[point] (a\x\y) at (\x, \y) {};
            \node[rect] (r\x\y) [below right of=a\x\y, shift={(-6mm, 6mm)}] {};

            \draw[-] (a\x\y) -- (r\x\y);
        }

    \foreach \x in {0, 3, 6}
        \foreach \y in {1, ..., 4}
        {
            \FPeval{\res}{clip(\x+3)}
            \ifthenelse{\y = 1}{
                \draw[-] (a\x0) -- (a\res0);
                \draw[-] (a\x0) -- (a\x-3);
            }{
                \ifthenelse{\y = 2}{
                    \draw[-] (a\x-3) -- (a\res-3);
                    \draw[-] (a\x-3) -- (a\x-6);
                }{
                    \ifthenelse{\y = 3}{
                        \draw[-] (a\x-6) -- (a\res-6);
                        \draw[-] (a\x-6) -- (a\x-9);
                    }{
                        \draw[-] (a\x-9) -- (a\res-9);
                    }
                }
            }
        }
    \draw[-]    (a90) -- (a9-3) -- (a9-6) -- (a9-9);
\end{graph}}}
\hfill
\subfloat[\emph{Torus topology}]{\resizebox*{0.48\textwidth}{!}{\begin{graph}
    \tikzset{
        rect/.style={rectangle, draw, fill, minimum size=3mm, inner sep=0},
        point/.style={circle, draw=gray, fill=gray, minimum size=2mm, inner sep=0},
        curve/.style={rounded rectangle, draw, minimum height=15mm,
            minimum width=105mm, shift={(15mm, 7.5mm)}},
        rcurve/.style={rounded rectangle, draw, minimum height=15mm,
            minimum width=105mm, rotate=-90, shift={(15mm, -7.5mm)}},
        every path/.style={draw=gray}
    }
    
    \foreach \x in {0, 3, 6, 9}
        \foreach \y in {0,-3, -6, -9}
        {
            \node[point] (a\x\y) at (\x, \y) {};
            \node[rect] (r\x\y) [below right of=a\x\y, shift={(-6mm, 6mm)}] {};

            \draw[-] (a\x\y) -- (r\x\y);
        }

    \foreach \x in {0, 3, 6}
        \foreach \y in {1, ..., 4}
        {
            \FPeval{\res}{clip(\x+3)}
            \ifthenelse{\y = 1}{
                \draw[-] (a\x0) -- (a\res0);
                \draw[-] (a\x0) -- (a\x-3);
            }{
                \ifthenelse{\y = 2}{
                    \draw[-] (a\x-3) -- (a\res-3);
                    \draw[-] (a\x-3) -- (a\x-6);
                }{
                    \ifthenelse{\y = 3}{
                        \draw[-] (a\x-6) -- (a\res-6);
                        \draw[-] (a\x-6) -- (a\x-9);
                    }{
                        \draw[-] (a\x-9) -- (a\res-9);
                    }
                }
            }
        }
    \draw[-]    (a90) -- (a9-3) -- (a9-6) -- (a9-9);

    \node[curve] (r1) at (a30) {};
    \node[curve] (r2) at (a3-3) {};
    \node[curve] (r3) at (a3-6) {};
    \node[curve] (r4) at (a3-9) {};

    \node[rcurve] (r5) at (a0-3) {};
    \node[rcurve] (r6) at (a3-3) {};
    \node[rcurve] (r7) at (a6-3) {};
    \node[rcurve] (r8) at (a9-3) {};

    \node[empty] (z) [below of=a0-9, yshift=-3.5mm] {};
\end{graph}}}
\caption{\emph{Static networks topologies}}
\end{figure}

\begin{figure}[h!]
    \centering
    \subfloat[\emph{Crossbar switch topology}]{\img{crossbar-switch.png}{0.48}}
    \hfill
    \subfloat[\emph{Omega switch topology}]{\img{omega-switch.png}{0.48}}      
    \caption{\emph{Switched network topologie}}
\end{figure}

\noindent
Up to this point, we can say that an \emph{interconnection network} must be
scalable, provide a high bandwidth with low latency e must guarantee what is
called \emph{location transparent communication}, meaning that every server
should communicate with the others with similar speed and latency. In simple
terms, the position of a server in the data center shouldn't have an impact on
its networking performance. This latter requirement, translates in the
impossibility of adopting a hierarchical organization of servers.

Costs, also comes in hand when designing an \emph{interconnection network}. Both
servers and network apparatus costs are relevant, but talking about networking,
the choice of routers and switches often requires a compromised between latency
and costs. For example, based on the number of ports of a router we can have
low-radix and high-ridix routers. The former has a few ports, while the latter
has more; thus bandwidth is divided into a smaller number of wide ports and
a larger number of narrow ports respectively.

\bigskip\noindent
In general, data centers networks are designed around the following schema:
\begin{figure}[h!]
    \centering
    \img{general-schema.png}{0.6}
    \caption{Generale \emph{interconnetion network} schema}
\end{figure}

\noindent
Layer 2 and layer 3 has pros and cons that have to balanced to guarantee ease
of configuration, administration and problem-solving.

Layer 2, that is the ethernet switching part of the infrastructure, is easy to
configure, thanks to autoconfiguration protocols such as DHCP, and allow the
addition of servers in a plug \& play way, but it's important to limit broadcast
domains to avoid link saturarion due to broadcast frames, and dealing with the
Spanning Tree Protocol might be a struggle.

On the other hand, IP routing on layer 3 makes scalability easy thanks to
hieranchical addressing and allows to obtain multipath routing with equal-cost
multipath\footnote{We will see more about this later in this section}. However,
its confiuration in more complex and migration requires changing the IP addresses.

\begin{note}
    At the end of the day, the entire \emph{interconnection network} will look
    like a giant switch that allows all the servers to communicate one another.
\end{note}

\subsection{Topologies more in depth}
\chapter{Cloud storage}
After going through CPU and network virtualization, it's time to talk about
proper ways of handling storaged data. In the age of \emph{cloud computing}
there's a huge amount of data to store, access and secure, so novel approches
to system design are necessary. To achieve the goal of cloud computation,
effective data replication and appropriate storage management strategies are
critical.

In the past, storage system were desing according to a
\emph{performance-at-any-cost} philosophy, but now there's been a shift towards
\emph{reliability-at-the-lowest-possible-cost}. For example, data replication
allows concurrent access to data from multiple processors and decreases the
chances of data loss, but maintaining consistency among multiple copies of data
records increases the data management software complexity and could negatively
affect the storage system performance if data is frequently updated.

So, in this chapter we're going to explain some modern strategies adopted
to manage cloud storage while also respecting performance and realibility
requirements.

\section{Preliminary definitions}
Before proceding, we need to fix some concepts.

\begin{definition}[Storage model]
    A storage model describes the layout of a data structure in a physical storage.
\end{definition}
\begin{definition}[Data model]
    A data model captures the most important logical aspects of a data structure
    in a database
\end{definition}

\begin{definition}[Read-write coherence]
    The result of reading of a memory cell, should be the same as the most recent
    writing done on that cell.
\end{definition}
\begin{definition}[Before-or-after atomicity]
    The result of every read or write operation is the same as if that
    operation has been performed completely before or after another read or
    write operation.
\end{definition}

\noindent
\emph{Read-write coherenct} and \emph{Before-or-after atomicity} are two highly
desirable properties of any \emph{storage model}.

\subsection{Types of storage}
There are three main types of storage:
\begin{enumerate}
    \item \emph{Block storage}: data is managed as blocks within sectors and
    tracks. Can be used when storage has access to raw and unformatted hardware
    and is usedfull when both speed and efficiency are important;
    \item \emph{File storage}: data is organized as structured files which are
    managed through a file system. However, this doesn't work very well with
    large amounts of data or high-demands for a particular piece of data;
    \item \emph{Object storage}: data is managed as objects. Typically, each
    object has a gloabal unique identifier and holds both the data itself and
    some metadata. It's possible to access whole objects or blobs of data, but
    random access to data within an object is more difficult;
\end{enumerate}

\noindent
\emph{Block storage} is often used for database because it's the ideal way of
storing relational information. On the other hand, \emph{Object storage} isn't
suitable for databases, but is preferable to store content that can grow without
bounds, for example, backups and archives. Finally, the \emph{File storage} model
can be easily used to create \emph{distributed file system} vian network protocols.

Examples of \emph{Block storage}, \emph{File storage} and \emph{Object storage}
are, respectivelly, Cinder, Google file system and OpenStack Swift.

\section{Block storage}
We'll give a quick look to this model by scratching the surface of \emph{OpenStack
Cinder}. \emph{OpenStack Cinder} implements services and libraries to provide
on-demand, self-service access to block storage resources. It provides API to
interact with storage backends which is also exposed to the cloud. End users can
manage their storage without knowing how it's organized and for both physical and
virtual environments.

\begin{figure}[h!]
    \centering
    \includegraphics[width=0.7\textwidth]{images/block-model-cinder.png}
    \caption{\emph{OpenStack Cinder basic architecture}}
\end{figure}

\noindent
\emph{Cinder} APIs allow users to create and delete volumes and snaphots (for
backup pourposes). Attach or detach volumes for the available storage, clone and
extend them, and so on.

\section{Distributed file system}
Again, we'll start by giving some definitions.

\begin{definition}[File]
    A file is a linear array of cells stored on a persistent storage device. It
    is seen by application as a collection of logical records, and it's stored
    on a physical device as a set of physical records, or blocks, of size
    dictated by the physical media.
\end{definition}
\begin{definition}[File pointer]
    A file pointer is a cell used as starting point for a read or write operation.
\end{definition}

\noindent
A file has a logical organisation reflects the data model from the perspective of
the application, and a physical orgnisation that reflects the storage model and
describes the manner the file is stored on a given storage media.

\begin{definition}[File system]
    A file system is a collection of directories and each directory provides
    information about a set of files. A file system controls how data is stored
    and retrieved.
\end{definition}

\subsection{Unix File System}
The \emph{Unix File System} (\emph{UFS}) has both a layered and hierarchical
design. The first allows it to separate the physical file structure from the
logical one, and this becomes usefull when dealing with files stored both locally
and remotely. The hierarchical design, on the other hand, allows grouping of
files directories, supports multiple levels of directories and collections of
directories and files; thus, supporting a good degree of scalability.

The metadata that \emph{UFS} uses includes file owner, access rights, creation
time, time of the last modification, file size, the structure of the file and
the persistent storage device cells where data is stored. All of these are
memorized in a structure called \emph{inode}. Each \emph{inode} holds information
about individual files and directories and is kept on persistent media together
with the actual data.

\bigskip\noindent
Going back to \emph{UFS} layering, there are two layers:
\begin{itemize}
    \item \emph{Lower layer}: is about the physical organisation and is furderly
    separated into three sub-layers:
    \begin{itemize}
        \item \emph{Block layer}: responsible for locating individual blocks on
        the physical device;
        \item \emph{File layer}: reflects the organisation of blocks into files;
        \item \emph{Inode layer}: provides metadata for files and directories;
    \end{itemize}
    \item \emph{Upper layer}: is about the logical orgnisation;
\end{itemize}

\newpage
\begin{figure}[ht!]
    \centering
    \includegraphics[width=0.6\textwidth]{images/ufs-layering.png}
    \caption{\emph{UFS} layering}
\end{figure}

\subsection{Network File System}
The \emph{Network File System} (\emph{NFS}) is designed to provide the same
semantics as the \emph{UFS} to ensure compatiblity with existing applications.
\emph{NFS} is based on the client-server paradigm and, in particular, clients
runs on the local host and interact with the server via remote procedure calls.

Each remote file is identified by a 32-bit file handler rather than a file
descriptor like in \emph{UFS}. A file handler is obtained combining a file
system identication, an \emph{inode} number and a generation number.

\begin{figure}[h!]
    \centering
    \includegraphics[width=0.6\textwidth]{images/nfs-design.png}
    \caption{\emph{NFS} design}
\end{figure}

\noindent
As the image shows, instead of an \emph{inode layer} the \emph{NFS} has a
\emph{vnode layer} which implements file operation in a uniform manner,
regardless of whether the file is local or remote. Also, an operation targeting
a local file is directed to the local file system without involving the \emph{NFS}.

When a client interacts with the server, the client-side \emph{NFS} packages the
relevant information about target, then sends those packeged information to the
server-side \emph{NFS}. The latter then passes the received information to the
\emph{vnode layer} of the remote host which finally directs it to the remote
file system.

\newpage
\begin{table}[ht!]
    \centering
    \begin{tabular}{|l|l|c|l|p{0.5\textwidth}|}
        \hline
        \textbf{API} & \textbf{Description} && \textbf{API} & \textbf{Description}\\
        \hline
        \texttt{fd} & File descriptor && \texttt{dfh} & The directory when the file
        handle can be found\\
        \hline
        \texttt{fh} & File handle && \texttt{count} & Number of bytes to be transferred\\
        \hline
        \texttt{fname} & File name && \texttt{buf} & Buffer used to transfer data\\
        \hline
        \texttt{dname} & Directory name && \texttt{device} & The device in which the
        file system is located\\
        \hline
    \end{tabular}
    \caption{\emph{NFS} APIs}
\end{table}

\begin{figure}[h!]
    \centering
    \includegraphics[width=0.6\textwidth]{images/nfs-remote-calls.png}
    \caption{\emph{UFS} APIs and corresponding \emph{NFS} remote procedure calls}
\end{figure}

\paragraph{Common design choices for distributed file system}
The vast majority of distributed file systems has been implemented in similar
ways. That's not so surprising if we think that they all need to solve the same
problems and that the possible or best solutions to choose from are limited.

Tipically every distributed file system is implented in a way that once a file
is closed, the server will have the newest version of it on persisten storage.
Another concern is about what to do when writing on a file. The possible
solutions are two:
\begin{enumerate}
    \item \emph{Write-through}: block is written to the disk as soon as it is
    available on the cache. This approch increases reliability, but takes more
    time to complete each write operation;
    \item \emph{Delay in write-back}: a block is first written to cache and
    writing on the disk is delayed for a time in the order of tens of seconds.
    This speeds up writings and avoids useless writing when data is discarded
    before the time to save it to the disk. However, data can be lost in case
    of system failures;
\end{enumerate}

\noindent
Finally, how should the system act when multiple clients tries to access the
same file at the same time. The approches are two:
\begin{enumerate}
    \item \emph{Sequential write-sharing}: a file cannot be opened simultaneously
    for reading and writing by several clients;
    \item \emph{Concurrent write-sharing}: multiple clients can modify the file
    at the same time;
\end{enumerate}

\subsection{General Parallel File System}


\end{document}