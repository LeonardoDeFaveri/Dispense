\chapter{Analisi ammortizzata}
Dopo aver parlato a lungo di \emph{strutture dati}, vediamo ora una tecnica che
ci permette di effettuare una stima dall'alto del costo che è necessario pagare
per eseguire delle operazioni su quelle strutture.
\begin{definition}[Analisi ammortizzata]
    L'analisi ammortizzata è una tecnica per l'analisi di complessità che valuta
    il tempo richiesto per eseguire, nel caso pessimo, una sequenza di operazioni
    su una struttura dati.
\end{definition}\noindent
In generale, esistono operazioni più o meno costose, ma se le operazioni più
costose sono poco frequenti, il loro costo può essere ammortizzato da quelle
meno costose.

È importante esplicitare la profonda differenza tra l'analisi del caso medio
e l'\emph{analisi ammortizzata}, effettuata sul caso pessimo. Nella prima,
l'analisi è di tipo probabilistico ed è effettuata sulle singole operazioni,
mentre, la seconda, è un'analisi deterministica effettuata su una sequenza di
operazioni, e in particolare, sulla sequenza più costosa possibile.

\section{Metodi per l'analisi ammortizzata}
Esistono diverse tecniche per realizzare un'\emph{analisi ammortizzata}.

\begin{definition}[Metodo dell'aggregazione]
    Col metodo dell'aggregazione si calcola la complessità $T(n)$ necessaria
    per eseguire $n$ operazioni in sequenza nel caso pessimo, quindi, si calcola
    il costo ammortizzato come $T(n)/n$ cioè come media su $n$ operazioni.
\end{definition}

\noindent
Definiamo meglio i termini presenti nella definizione:
\begin{itemize}
    \item \emph{Sequenza}: è una sequenza di operazioni che permettono alla
    \emph{struttura dati} di evolvere;
    \item \emph{Caso pessimo}: è la \emph{sequenza} con \emph{complessità} più
    alta tra tutte quelle possibili;
    \item \emph{Aggregazione}: la \emph{complessità} $T(n)$ è ottenuta mediante
    una sommatoria dei costi delle singole operazioni;
\end{itemize}

\newpage
\begin{definition}[Metodo degli accantonamenti]
    Col metodo degli accantonamenti si assegna ad ogni operazione un costo
    ammortizzato che può anche essere diverso dal costo effettivo. In particolare,
    le operazioni meno costose vengono caricate di un costo aggiuntivo detto
    credito, che viene poi usato per pagare le operazioni più costose.
\end{definition}
\begin{note}
    Potenzialmente, ad ogni operazione potrebbe essere assegnato un costo diverso.
\end{note}

\noindent
In generale, utilizzando il \emph{metodo degli accantonamenti}, il \emph{costo
ammortizzato} assegnato alle operazioni meno costose è definito come:
\[\emph{Costo ammortizzato}=\emph{Costo effettivo}+\emph{Credito prodotto}\]
Viceversa, il costo per le operazioni più costose è:
\[\emph{Costo ammortizzato}=\emph{Costo effettivo}-\emph{Credito consumato}\]
Quindi, l'obiettivo, quando si utilizza questo metodo, è dimostrare che la somma
dei \emph{costi ammortizzati} $a_i$ è un limite superiore ai costi effettivi
$c_i$, ovvero che vale:
\[\sum_{i=1}^nc_i\leq\sum_{i=1}^na_i\]
Ovviamente, la dimostrazione deve valere per tutte le sequenze, ma se vale per il
caso pessimo vale anche per tutti gli altri. Il \emph{credito} dopo la $t$-esima
operazione è sempre positivo ed è espresso dalla seguente formula:
\[\sum_{i=1}^ta_i-\sum_{i=1}^tc_i\geq0\]

\begin{definition}[Metodo del potenziale]
    Col metodo del potenziale si associa una funzione di potenziale $\Phi$ ad
    uno stato $S$ della struttura. La funzione $\Phi$ definisce la \q{quantità
    di lavoro} $\Phi(S)$ che è stato contabilizzata nell'analisi ammortizzata,
    ma non ancora spesa.
\end{definition}
\begin{note}
    $\Phi(S)$ rappresenta la quantità di energia potenziale che è stata
    \q{immagazzinata} in quello stato e che può essere spesa per eseguire le
    operazioni più costose.
\end{note}\noindent
In pratica, $\Phi(S)$ può essere vista come la cumulazione dei \emph{crediti
prodotti} dalle operazioni fino al raggiungimento dello stato $S$.

\bigskip\noindent
In generale, il \emph{costo ammortizzato} di un'operazione è pari a:
\[\begin{array}[t]{rcl}
    \emph{Costo ammortizzato} & = & \emph{Costo effettivo}+\emph{Variazione di potenziale}\\
    a_i & = & c_i + (\Phi(S_i)-\Phi(S_{i-1}))
\end{array}\]
dove $S_i$ è lo stato associato all'$i$-esima operazione.
Se proviamo a calcolare il \emph{costo ammortizzato} di una sequenza di $n$
operazioni, vale quanto segue:
\[A\begin{array}[t]{cl}
    = & \sum_{i=1}^na_i\\
    = & \sum_{i=1}^n\left(c_i+\Phi(S_i)-\Phi(S_{i-1})\right)\\
    = & \sum_{i=1}^nc_i+\sum_{i=1}^n\left(\Phi(S_i)-\Phi(S_{i-1})\right)\\
    = & C + \left(\Phi(S_1) - \Phi(S_{0})\right) + \left(\Phi(S_2) - \Phi(S_1)\right)+\dots+\left(\Phi(S_n)-\Phi(S_{n-1})\right)\\
    = & C + \Phi(S_n) - \Phi(S_0)
\end{array}\]
Allora, se $\Phi(S_n)-\Phi(S_0)\geq0$, $A$ è un limite superiore al costo effettivo.

\section{Esempio di analisi ammortizzata}
\begin{eg}[Contatore binario]
Si consideri un contatore binario a $k$ bit implementato come un vettore $A$ di
booleani, nel quale $A[0]$ è il bit meno significativo e $A[k-1]$ il più significativo.
Il valore del contatore corrisponde al risultato della seguente sommatoria:
\[x=\sum_{i=0}^{k-1}A[i]\cdot2^i\]
Alla struttura del contatore è associata soltanto la funzione \texttt{increment}
per l'incremento del valore.

\begin{minicode}{Implementazione funzione increment per contatori binari}
\ind increment(\bc{int}[] A, int k)\\
    \bc{int} i = 0\\
    \indf while (i < k and A[i] == 1) do\hfill\com{Imposta a 0 la prima sequenza di bit a 1}
        A[i] = 0\\
        i = i + 1\\
    \indf if (i < k) then\hfill\com{Se $i < k$ non c'è overflow}
        A[i] = 1
\end{minicode}\noindent
Se $k=8$, si ottiene un contatore binario a 8 bit e la seguente tabella ne mostra
lo stato dopo il sedicesimo incremento.
\begin{table}[h!]
    \centering
    \renewcommand{\arraystretch}{1.2}
    \begin{tabular}{|c|c|c|c|c|c|c|c|c|}
        \hline
        \bm{$x$} & \bm{$A[7]$} & \bm{$A[6]$} & \bm{$A[5]$} & \bm{$A[4]$} &
        \bm{$A[3]$} & \bm{$A[2]$} & \bm{$A[1]$} & \bm{$A[0]$}\\
        \hline
        0 & 0 & 0 & 0 & 0 & 0 & 0 & 0 & 0\\
        \hline
        1 & 0 & 0 & 0 & 0 & 0 & 0 & 0 & 1\\
        \hline
        2 & 0 & 0 & 0 & 0 & 0 & 0 & 1 & 0\\
        \hline
        3 & 0 & 0 & 0 & 0 & 0 & 0 & 1 & 1\\
        \hline
        4 & 0 & 0 & 0 & 0 & 0 & 1 & 0 & 0\\
        \hline
        5 & 0 & 0 & 0 & 0 & 0 & 1 & 0 & 1\\
        \hline
        6 & 0 & 0 & 0 & 0 & 0 & 1 & 1 & 0\\
        \hline
        7 & 0 & 0 & 0 & 0 & 0 & 1 & 1 & 1\\
        \hline
        8 & 0 & 0 & 0 & 0 & 1 & 0 & 0 & 0\\
        \hline
        9 & 0 & 0 & 0 & 0 & 1 & 0 & 0 & 1\\
        \hline
        10 & 0 & 0 & 0 & 0 & 1 & 0 & 1 & 0\\
        \hline
        11 & 0 & 0 & 0 & 0 & 1 & 0 & 1 & 1\\
        \hline
        12 & 0 & 0 & 0 & 0 & 1 & 1 & 0 & 0\\
        \hline
        13 & 0 & 0 & 0 & 0 & 1 & 1 & 0 & 1\\
        \hline
        14 & 0 & 0 & 0 & 0 & 1 & 1 & 1 & 0\\
        \hline
        15 & 0 & 0 & 0 & 0 & 1 & 1 & 1 & 1\\
        \hline
        16 & 0 & 0 & 0 & 1 & 0 & 0 & 0 & 0\\
        \hline
    \end{tabular}
\end{table}

\bigskip\noindent
Proviamo ora a realizzare una stima dall'alto del costo della funzione
\texttt{increment} usando i metodi per l'analisi ammortizzata.

\paragraph{Metodo dell'aggregazione}
Utilizzando questo metodo ci chiediamo quale sia il costo $T(n)$ da pagare per
eseguire $n$ operazioni in sequenza.

Inizialmente potremmo notare che per rappresentare $n$ in binario servono
$k=\lceil\log(n+1)\rceil$ bit. Un'invocazione di \texttt{increment} nel caso
pessimo richiede un tempo $O(k)$, quindi $n$ invocazioni costano $T(n)=O(nk)$.
A questo punto, il costo ammortizzato di ogni operazione è $T(n)/n=O(nk)/n=O(k)=O(\log n)$.

Se però notiamo che il costo per eseguire l'intera sequenza è proporzionale al
numero di bit che vengono modificati, possiamo provare a realizzare una stima
più ristretta. Proviamo quindi a contare per ogni riga e per ogni colonna il
numero di bit che sono stati modificati.

\begin{table}[h!]
    \centering
    \renewcommand{\arraystretch}{1.2}
    \begin{tabular}{|c|c|c|c|c|c|c|c|c|c|}
        \hline
        \bm{$x$} & \bm{$A[7]$} & \bm{$A[6]$} & \bm{$A[5]$} & \bm{$A[4]$} &
        \bm{$A[3]$} & \bm{$A[2]$} & \bm{$A[1]$} & \bm{$A[0]$} & \bm{$\#bit$}\\
        \hline
        0 & 0 & 0 & 0 & 0 & 0 & 0 & 0 & 0 & 0\\
        \hline
        1 & 0 & 0 & 0 & 0 & 0 & 0 & 0 & \color{red}{1} & 1\\
        \hline
        2 & 0 & 0 & 0 & 0 & 0 & 0 & \color{red}{1} & \color{red}{0} & 2\\
        \hline
        3 & 0 & 0 & 0 & 0 & 0 & 0 & 1 & \color{red}{1} & 1\\
        \hline
        4 & 0 & 0 & 0 & 0 & 0 & \color{red}{1} & \color{red}{0} & \color{red}{0} & 3\\
        \hline
        5 & 0 & 0 & 0 & 0 & 0 & 1 & 0 & \color{red}{1} & 1\\
        \hline
        6 & 0 & 0 & 0 & 0 & 0 & 1 & \color{red}{1} & \color{red}{0} & 2\\
        \hline
        7 & 0 & 0 & 0 & 0 & 0 & 1 & 1 & \color{red}{1} & 1\\
        \hline
        8 & 0 & 0 & 0 & 0 & \color{red}{1} & \color{red}{0} & \color{red}{0} & \color{red}{0} & 4\\
        \hline
        9 & 0 & 0 & 0 & 0 & 1 & 0 & 0 & \color{red}{1} & 1\\
        \hline
        10 & 0 & 0 & 0 & 0 & 1 & 0 & \color{red}{1} & \color{red}{0} & 2\\
        \hline
        11 & 0 & 0 & 0 & 0 & 1 & 0 & 1 & \color{red}{1} & 1\\
        \hline
        12 & 0 & 0 & 0 & 0 & 1 & \color{red}{1} & \color{red}{0} & \color{red}{0} & 3\\
        \hline
        13 & 0 & 0 & 0 & 0 & 1 & 1 & 0 & \color{red}{1} & 1\\
        \hline
        14 & 0 & 0 & 0 & 0 & 1 & 1 & \color{red}{1} & \color{red}{0} & 2\\
        \hline
        15 & 0 & 0 & 0 & 0 & 1 & 1 & 1 & \color{red}{1} & 1\\
        \hline
        16 & 0 & 0 & 0 & \color{red}{1} & \color{red}{0} & \color{red}{0} & \color{red}{0} & \color{red}{0} & 5\\
        \hline
        \#bit & 0 & 0 & 0 & 1 & 2 & 4 & 8 & 16 &\\
        \hline
    \end{tabular}
\end{table}\noindent
Guardando l'ultima riga possiamo notare che il bit in posizione $A[0]$ viene
modificato ad ogni incremento, quello in $A[1]$ ogni due, quello in $A[2]$ ogni
4 e così via. Di conseguenza, il costo $T(n)$ è una funzione di questo tipo:
\[T(n)=\sum_{i=0}^{k-1}\left\lfloor\frac{n}{2^i}\right\rfloor\leq n\sum_{i=0}^{k-1}\frac
{1}{2^i}\leq n\sum_{i=0}^{+\infty}\left(\frac{1}{2}\right)^i=2n\]
e quindi il costo ammortizzato vale:
\[\frac{T(n)}{n}\leq\frac{2n}{n}=2=O(1)\]

\paragraph{Metodo degli accantonamenti}
Supponiamo che il costo effettivo della \texttt{increment} sia $d$, dove $d$ è
il numero di bit che vengono modificati ad ogni incremento. Proviamo però ad
assegnare alla \texttt{increment} un costo di $2$ che include 1 per il costo
effettivo che si paga per cambiare un bit da 0 a 1 e 1 per il futuro cambio
dello stesso bit da 1 a 0.

Di conseguenza, in ogni istante, il credito è pari al numero di bit a 1
presenti e quindi il costo totale vale $O(n)$ e il costo ammortizzato $O(1)$.

\paragraph{Metodo del potenziale}
Scegliamo come funzione di potenziale $\Phi$ il numero di bit a 1 presenti nel
contatore. Ad ogni utilizzo della funzione \texttt{increment}, se $t$ è il numero
di bit a 1 incontrati prima del primo 0, il costo effettivo è $1+t$ perché
cambiamo lo stato di tutti i primi $t$ bit a 1 e del primo bit a 0.

La variazione di potenziale, invece, vale $1-t$ in quanto tutti i bit a 1
incontrati diventano 0 e il primo bit a 0 diventa 1. Quindi, se allo stato $S_i$
$\Phi(S_i)=t$, allo stato $S_{i+1}$ $\Phi(S_{i+1})=1$ e, conseguentemente, la
variazione di potenziale tra lo stato $S_{i+1}$ e lo stato $S_i$ vale:
\[\Phi(S_{i+1})-\Phi(S_i)=1-t\]
Il costo ammortizzato di un'invocazione della \texttt{increment}, allora, vale:
\[\emph{Costo ammortizzato}\begin{array}[t]{cl}
    = & \emph{Costo effettivo}+\emph{Variazione di potenziale}\\
    = & (1+t)+(1-t)\\
    = & 2 = O(1)
\end{array}\]
Inoltre, poiché allo stato $S_0$, $\Phi(S_0)=0$ perché non ci sono bit impostati a
1, e allo stato $S_n$, $\Phi(S_n)\geq0$, risulta verificata anche la
condizione $\Phi(S_n)-\Phi(S_0)\geq0$.
\end{eg}

\section{Vettori dinamici}
Possiamo utilizzare le tecniche di \emph{analisi ammortizzata} per stimare la
\emph{complessità} delle operazioni di inserimento e cancellazione nei vettori
con ridimensionamento dinamico della dimensione.

Prima però vediamo brevemente cosa significa \q{ridimensionamento dinamico della
dimensione}. Quando si implementa una \emph{\hyperref[def:27]{sequenza}}
utilizzando un vettore, se ne specifica una dimensione iniziale detta
\emph{capacità}. Ovviamente, non è sempre nota a priori la quantità di elementi
che dovranno essere inseriti nella struttura, e quindi la \emph{capacità} iniziale
può rivelarsi insufficiente. I \emph{vettori dinamici} risolvono questo problema
allocando un nuovo vettore con una capacità maggiore e ricopiando in esso tutti
i valori presenti nel vettore originale, che poi può essere eliminato.

\subsection{Inserimento}
Quando in fase di inserimento si rende necessario aumentare la dimensione del
vettore, esistono due principali metodologie di approccio: \emph{incremento fisso}
e \emph{incremento variabile}.

Il primo approccio prevede che la \emph{capacità} venga incrementata di un fattore
fisso (e.g. $+1$, $+2$, $+10$, \dots), mentre il secondo è dipendente dalla dimensione
attuale della struttura (e.g. $\cdot2$, $\cdot1.5$, \dots).

Ma qual è il metodo migliore? Proviamo a utilizzare l'\emph{analisi ammortizzata}
per realizzare una stima.

\paragraph{Incremento fisso}
Il costo effettivo di un inserimento con ridimensionamento fisso è descritto
dalla seguente espressione:
\[c_i=\begin{cases}
    i & \text{se}\ (i\Mod{d}) = 1\footnotemark\\
    1 & \text{altrimenti}
\end{cases}\]
dove $d$ è sia la dimensione iniziale che il valore di cui viene incrementata.
\footnotetext{L'inserimento ha un costo $i$ quando $(i\Mod{d})=1$ perché se è stata
raggiunta la dimensione massima quell'operazione di modulo vale 1}

\newpage\noindent
Supponiamo $d=4$ e vediamo il costo delle prime 17 operazioni di inserimento:
\begin{table}[ht]
    \renewcommand{\arraystretch}{1.2}
    \centering
    \begin{tabular}{|c|l|}
        \hline
        \bm{$n$} & \textbf{Costo}\\
        \hline
        1 & 1\\
        \hline
        2 & 1\\
        \hline
        3 & 1\\
        \hline
        4 & 1\\
        \hline
        5 & $1+d=5$\\
        \hline
        6 & 1\\
        \hline
        7 & 1\\
        \hline
        8 & 1\\
        \hline
        9 & $1+2d=9$\\
        \hline
        10 & 1\\
        \hline
        11 & 1\\
        \hline
        12 & 1\\
        \hline
        13 & $1+3d=13$\\
        \hline
        14 & 1\\
        \hline
        15 & 1\\
        \hline
        16 & 1\\
        \hline
        17 & $1+4d=17$\\
        \hline
    \end{tabular}
\end{table}

\noindent A questo punto, calcoliamo il costo effettivo di $n$ operazioni:
\[\renewcommand{\arraystretch}{1.3}
    T(n)\begin{array}[t]{cll}
    = & \sum_{i=1}^nc_i\\
    = & n + \sum_{j=1}^{\lfloor n/d\rfloor}d\cdot j\\
    = & n + d\sum_{j=1}^{\lfloor n/d\rfloor}j\\
    = & n + d\frac{\left(\lfloor n/d\rfloor+1\right)\lfloor n/d\rfloor}{2}\\
    \leq & n + \frac{\left(n/d+1\right)n}{2} & \quad\emph{Rimozione dell'intero inferiore}\\
    = & n + \frac{n^2/d+n}{2}=O(n^2)
\end{array}\]
Il \emph{costo ammortizzato} vale:
\[\frac{T(n)}{n}=\frac{O(n^2)}{n}=O(n)\]

\paragraph{Incremento dinamico}
Supponendo che ad ogni incremento la dimensione del vettore venga raddoppiata, il
costo effettivo vale:
\[c_i=\begin{cases}
    i & \text{se}\ \exists k\in\mathbb{Z}^+_0:i=2^k+1\\
    1 & \text{altrimenti}
\end{cases}\]
\newpage\noindent
Se la dimensione iniziale è 1, il costo delle prime 17 operazioni è il seguente:
\begin{table}[h!]
    \renewcommand{\arraystretch}{1.2}
    \centering
    \begin{tabular}{|c|l|}
        \hline
        \bm{$n$} & \textbf{Costo}\\
        \hline
        1 & 1\\
        \hline
        2 & $1+2^0=2$\\
        \hline
        3 & $1+2^1=3$\\
        \hline
        4 & 1\\
        \hline
        5 & $1+2^2=5$\\
        \hline
        6 & 1\\
        \hline
        7 & 1\\
        \hline
        8 & 1\\
        \hline
        9 & $1+2^3=9$\\
        \hline
        10 & 1\\
        \hline
        11 & 1\\
        \hline
        12 & 1\\
        \hline
        13 & 1\\
        \hline
        14 & 1\\
        \hline
        15 & 1\\
        \hline
        16 & 1\\
        \hline
        17 & $1+2^4=17$\\
        \hline
    \end{tabular}
\end{table}

\noindent Il costo effettivo di $n$ operazioni vale:
\[\renewcommand{\arraystretch}{1.3}
    T(n)\begin{array}[t]{cll}
    = & \sum_{i=1}^nc_i\\
    = & n + \sum_{j=0}^{\lfloor\log n\rfloor}2^j\\
    = & n + 2^{\lfloor\log n\rfloor+1}-1\\
    \leq & 2^{\log(n)+1}-1 & \quad\emph{Rimozione dell'intero inferiore}\\
    = & n+2n-1=O(n)
\end{array}\]
e di conseguenza quello \emph{ammortizzato} è:
\[\frac{T(n)}{n}=\frac{O(n)}{n}=O(1)\]
Giunti a questo punto, siamo riusciti a dimostrare che l'incremento dinamico
ha un costo inferiore all'incremento fisso.

\subsection{Cancellazione}
Se il vettore non è ordinato, rimuovere un elemento significa spostare l'ultimo
elemento della \emph{sequenza} nella posizione dell'elemento da rimuovere.

Per ridurre lo spreco di memoria, è opportuno ridurre la dimensione del vettore
quando il \emph{fattore di carico} $\alpha=\frac{\emph{Dim}}
{\emph{Capacità}}$\footnotemark\ scende sotto una certa soglia.
\footnotetext{Il valore \emph{Dim} rappresenta il numero di elementi che
in dato momento sono presenti nella struttura}

L'operazione di riduzione della \emph{capacità} è detta \emph{contrazione} e,
similmente a quanto avviene con l'\emph{espansione}, prevede che venga allocato un nuovo
vettore di dimensione inferiore a quella attuale, che vi vengano copiati i
valori del vettore originale e che quindi quest'ultimo venga eliminato.

\bigskip\noindent
Ma qual è la soglia di \emph{contrazione} ottimale?

Una prima strategia potrebbe essere quella di dimezzare la \emph{capacità} quando
\emph{Dim} raggiunge la metà della \emph{capacità}, cioè quando $\alpha=
\frac{1}{2}$. Il problema di questa scelta è che dopo la \emph{contrazione}
nella struttura non rimangono posizioni libere e quindi un successivo inserimento
richiederebbe di fare subito un'\emph{espansione}.

Una strategia migliore è quella invece di scegliere $\alpha=\frac{1}{4}$. In
questo modo, se quando viene raggiunta la \emph{soglia di contrazione}, si va
a dimezzare la \emph{capacità}, il \emph{fattore di carico} $\alpha$, invece di
aumentare fino a $1$ si fermerà a $\frac{1}{2}$ concedendoci di aggiungere al
vettore tanti elementi quanti sono quelli già presenti prima di dover effettuare
un'\emph{espansione}.

\paragraph{Analisi del costo}
Proviamo a stimare il costo di questa seconda soluzione utilizzando il
\emph{metodo del potenziale}. Scegliamo una funzione di potenziale $\Phi$ che
vale 0 all'inizio e immediatamente dopo un'\emph{espansione} o una
\emph{contrazione} e il cui valore cresca fino a raggiungere il numero di
elementi presenti nella struttura quando $\alpha=1$ e diminuisca fino a un
quarto quando $\alpha$ si riduce alla \emph{soglia di contrazione}.

Quindi, la definizione di $\Phi$ è la seguente:
\[\Phi=\begin{cases}
    2\cdot\emph{Dim}-\emph{Capacità} & \text{se}\ \alpha\geq\frac{1}{2}\\
    \frac{\emph{Capacità}}{2}-\emph{Dim} & \text{se}\ \alpha<\frac{1}{2}
\end{cases}\]
Vediamone il valore in alcuni casi particolari:
\begin{itemize}
    \item $\alpha=\frac{1}{2}$: è stata appena effettuata un'\emph{espansione} o
    una \emph{contrazione} e quindi $\Phi=0$;
    \item $\alpha=1$ la struttura è piena, ovvero $\emph{Dim}=\emph{Capacità}$
    e quindi $\Phi=\emph{Dim}=\emph{Capacità}$;
    \item $\alpha=\frac{1}{4}$: il \emph{fattore di carico} ha raggiunto la
    \emph{soglia di contrazione}, ovvero $\emph{Dim}=\frac{\emph{Capacità}}
    {4}$ e quindi $\Phi=\emph{Dim}$;
\end{itemize}
Una funzione di potenziale di questo tipo ci garantisce che il potenziale presente
nel momento in cui si effettua un'\emph{espansione} o una \emph{contrazione} sia
sufficiente per \q{pagare} quelle stesse operazioni.

Proviamo quindi a calcolare il \emph{costo ammortizzato} in base ai diversi valori
di $\alpha$:
\begin{itemize}
    \item $\alpha\geq\frac{1}{2}$:
    \[a_i\begin{array}[t]{cl}
        = & c_i + \Phi_i - \Phi_{i-1}\\
        = & 1 + (2\emph{Dim}_i - \emph{Capacità}_i) - (2\emph{Dim}_{i-1}-\emph{Capacità}_{i-1})\\
        = & 1 + 2(\emph{Dim}_{i-1}+1) - \emph{Capacità}_i - 2\emph{Dim}_{i-1}+\emph{Capacità}_{i-1}\\
        = & 1 + 2(\emph{Dim}_{i-1}+1) - \emph{Capacità}_{i-1} - 2\emph{Dim}_{i-1}+\emph{Capacità}_{i-1}\\
        = & 1 + 2\emph{Dim}_{i-1}+2 - \emph{Capacità}_{i-1} - 2\emph{Dim}_{i-1}+\emph{Capacità}_{i-1}\\
        = & 3
    \end{array}\]
    \item $\alpha=1$:
    \[a_i\begin{array}[t]{cl}
        = & c_i+\Phi_i-\Phi_{i-1}\\
        = & 1 + \emph{Dim}_{i-i}+(2\emph{Dim}_i+\emph{Capacità}_i)-(2\emph{Dim}_{i-1}-\emph{Capacità}_{i-1})\\
        = & 1 + \emph{Dim}_{i-i}+2(\emph{Dim}_{i-1}+1)-2\emph{Dim}_{i-1}-2\emph{Dim}_{i-1}+\emph{Dim}_{i-1}\\
        = & 3
    \end{array}\]
\end{itemize}
In entrambi i casi il costo è $O(1)$.

\begin{note}
    Poiché non è conveniente che il \emph{fattore di carico} $\alpha$ cresca
    troppo, regole di ridimensionamento simili vengono usate anche nelle
    \emph{tabelle hash}. Solitamente, quando $\alpha$ raggiunge una soglia
    di $0.5$ o $0.75$ la \emph{tabella hash} viene estesa raddoppiandone la
    \emph{capacità}.  Quest'operazione, oltre che dimezzare il \emph{fattore di
    carico}, rimuove anche tutti gli elementi \texttt{deleted}. Il costo nel
    caso peggiore è $O(m)$, ma quello \emph{ammortizzato} è $O(1)$.
\end{note}

\section{Discussione sugli insiemi}
Giunti a questo punto della trattazione, abbiamo introdotto una vasta gamma di
\emph{strutture dati} più o meno complesse e caratterizzata da costi più o meno
convenienti. Abbiamo anche visto che \emph{strutture dati astratte} possono
essere implementate utilizzando diverse \emph{strutture dati concrete} e che
questo influenza il costo delle operazioni. Per chiudere il cerchio, mettiamo ora
a confronto diverse implementazioni di un \emph{\hyperref[def:28]{insieme}}.

\subsection{Insieme come vettore di booleani}
Quando si a che fare con $m$ valori ordinabili e consecutivi è possibile
implementare un \emph{insieme} per quei valori come un vettore di booleani di
dimensione $m$. Ogni posizione del vettore vale \texttt{true} se l'elemento
associato appartiene all'insieme, altrimenti vale \texttt{false}.

\bigskip\noindent
Vediamone l'implementazione:

\begin{code}{Insieme come vettore di booleani}
\begin{minipage}[t]{0.48\textwidth}
\bc{boolean}[] V\hfill\com{Vettore di booleani}
\bc{int} size\hfill\com{Numero di elementi presenti}
\bc{int} capacity\hfill\com{Dimensione massima}
\nl\ind\bc{SET} Set(\bc{int} m)\\
    \bc{SET} t = new \bc{SET}\\
    t.size = 0\\
    t.capacity = m\\
    t.V = new \bc{int}[1\dots m] = {false}\\
    return t\\

\ind\bc{boolean} contains(\bc{int} x)\\
    \indf if (1 $\leq$ x $\leq$ capacity) then\\
        return V[x]\\
    \indf else\\
        return false
\end{minipage}
\hfill
\begin{minipage}[t]{0.48\textwidth}
\ind\bc{int} size()\\
    return size\\

\ind insert(\bc{int} x)\\
    \indf if (1 $\leq$ x $\leq$ capacity) then\\
        \indff if (not V[x]) then\\
            size = size + 1\\
            V[x] = true\\

\ind remove(\bc{int} x)\\
\indf if (1 $\leq$ x $\leq$ capacity) then\\
    \indff if (V[x]) then\\
        size = size - 1\\
        V[x] = false
\end{minipage}
\newline\nl\ind\bc{SET} union(\bc{SET} A, \bc{SET} B)\\
    \bc{int} newSize = max(A.capacity, B.capacity)\\
    \bc{SET} C = Set(newSize)\\
    \ind for (i = 1 to A.capacity) do\\
        \indf if (A.contains(i)) then\\
            \indff C.insert(i)\\
    \ind for (i = 1 to B.capacity) do\\
        \indf if (B.contains(i)) then\\
            \indff C.insert(i)\\
    \ind return C
\end{code}
\begin{codecont}
\ind\bc{SET} difference(\bc{SET} A, \bc{SET} B)\\
    \bc{SET} C = Set(A.capacity)\\
    for (i = 1 to A.capacity) do\\
        \indf if (A.contains(i) and not B.contains(i)) then\\
            \indff C.insert(i)\\
    \ind return C\\
\nl\ind\bc{SET} intersection(\bc{SET} A, \bc{SET} B)\\
    \bc{int} newSize = min(A.capacity, B.capacity)\\
    \bc{SET} C = Set(newSize)\\
    for (i = 1 to newSize) do\\
        \indf if (A.contains(i) and B.contains(i)) do\\
            \indff C.insert(i)\\
    \indent return C
\end{codecont}

\bigskip\noindent
I vantaggi di questa implementazione sono certamente la semplicità e
l'efficienza delle operazioni di inserimento, rimozione e verifica di appartenenza.
Tuttavia, la memoria occupata è indipendente dal numero di elementi effettivamente
contenuti, così come lo sono le operazioni che visitano tutti gli elementi.
Ad esempio, le operazioni di \texttt{union}, \texttt{difference} e
\texttt{intersection} hanno costo $O(m)$.

\subsection{Insieme come lista non ordinata}
In un \emph{insieme} implementato come \emph{lista} non ordinata le operazioni
di inserimento, ricerca e rimozione hanno un costo $O(n)$ dove $n$ è il numero
di elementi presenti. Per le operazioni di unione, intersezione e differenza
non va meglio, infatti dati due \emph{insiemi} $A$ e $B$ di dimensione $n_A$ e
$n_B$, la \emph{complessità} è $O(n_An_B)$.

\begin{note}
    Se supponiamo di sapere che un elemento non appartiene all'\emph{insieme},
    l'inserimento può avere costo $O(1)$.
\end{note}

\begin{minicode}{Insieme come lista non ordinata}
\ind\bc{SET} union(\bc{SET} A, \bc{SET} B)\\
    \bc{SET} C = Set()\\
    \indf foreach (s $\in$ A) do\hfill\com{Costa $O(n_A)$}
        C.insert(s)\hfill\com{Costa $O(n_C)$\footnotemark}
    \indf foreach (s $\in$ B) do\hfill\com{Costa $O(n_B)$}
        C.insert(s)\hfill\com{Costa $O(n_C)$}
    \indf return C\hfill\com{Abbiamo pagato $O(n_An_C+n_Bn_C)=O(n_Cm)$}

\ind\bc{SET} difference(\bc{SET} A, \bc{SET} B)\\
    \bc{SET} C = Set()\\
    \indf foreach (s $\in$ A) do\hfill\com{Costa $O(n_A)$}
        \indff if (not B.contains(s))\hfill\com{Costa $O(n_B)$}
            C.insert(s)\hfill\com{Possiamo supporre ci costi $O(1)$}
    \indf return C\hfill\com{Abbiamo pagato $O(n_An_B1)=O(n_An_B)$}
\end{minicode}
\footnotetext{Avremmo anche potuto supporre di pagare $O(1)$ in quanto
sicuramente nessuno degli elementi di $A$ apparteneva a $C$}
\begin{codecont}
\ind\bc{SET} intersection(\bc{SET} A, \bc{SET} B)\\
    \bc{SET} C = Set()\\
    \ind foreach (s $\in$ A) do\hfill\com{Costa $O(n_A)$}
        \indf if (B.contains(s))\hfill\com{Costa $O(n_B)$}
            \indff C.insert(s)\hfill\com{Possiamo supporre ci costi $O(1)$}
    \indent return C\hfill\com{Abbiamo pagato $O(n_An_B1)=O(n_An_B)$}
\end{codecont}

\begin{note}
    Il costo dell'implementazione come vettore non ordinato è in tutto e per
    tutto equivalente all'implementazione come \emph{lista} non ordinata.
\end{note}

\subsection{Insieme come lista ordinata}
Con le \emph{liste} ordinate i costi rimangono identici ad eccezione di
unione, intersezione e differenza il cui costo si riduce a $O(n)$.

\begin{minicode}{Insieme come lista ordinata}
\ind\bc{SET} union(\bc{SET} A, \bc{SET} B)\\
    \bc{SET} C = Set()\\
    \bc{POS} pos$_A$ = A.head()\\
    \bc{POS} pos$_B$ = B.head()\\
    \indf while (not A.finished(pos$_A$) and not B.finished(pos$_B$)) do\hfill\com{$O(max(n_A,n_B))$}
        C.insert(C.tail(), A.read(pos$_A$))\hfill\com{L'inserimento in coda costa $O(1)$}
        \indff if (A.read(pos$_A$) == B.read(pos$_B$)) then\\    
            pos$_A$ = A.next(pos$_A$)\\
            pos$_B$ = B.next(pos$_B$)\\
        \indff else if (A.read(pos$_A$) < B.read(pos$_B$)) then\\
            pos$_A$ = A.next(pos$_A$)\\
        \indff else\\
            pos$_B$ = B.next(pos$_B$)\\
    \indf return C\\

\ind\bc{SET} difference(\bc{SET} A, \bc{SET} B)\\
    \bc{SET} C = Set()\\
    \bc{POS} pos$_A$ = A.head();\\
    \bc{POS} pos$_B$ = B.head();\\
    \indf while (not A.finished(pos$_A$) and not B.finished(pos$_B$)) do\hfill\com{$O(max(n_A,n_B))$}
        \indff if  (A.read(pos$_A$) $\neq$ B.read(pos$_B$)) then\\
            C.insert(C.tail(), A.read(pos$_A$))\hfill\com{L'inserimento in coda costa $O(1)$}
            \indfff if (A.read(pos$_A$) > B.read(pos$_B$)) then\\
                pos$_B$ = B.next(pos$_B$)\\
            \indfff pos$_A$ = A.next(pos$_A$)\\
    \indf return C\\
\end{minicode}
\newpage
\begin{codecont}
\ind\bc{SET} intersection(\bc{SET} A, \bc{SET} B)\\
    \bc{SET} C = Set()\\
    \bc{POS} pos$_A$ = A.head();\\
    \bc{POS} pos$_B$ = B.head();\\
    \ind while (not A.finished(pos$_A$) and not B.finished(pos$_B$)) do\hfill\com{$O(max(n_A,n_B))$}
        \indf if  (A.read(pos$_A$) == B.read(pos$_B$)) then\\
            \indff C.insert(C.tail(), A.read(pos$_A$))\hfill\com{L'inserimento in coda costa $O(1)$}
            \indff pos$_A$ = A.next(pos$_A$)\\
            \indff pos$_B$ = B.next(pos$_B$)\\
        \indf else if (A.read(pos$_A$) < B.read(pos$_B$)) then\\
            \indff pos$_A$ = A.next(pos$_A$)\\
        \indf else\\
            \indff pos$_B$ = B.next(pos$_B$)\\
    \indent return C\\
\end{codecont}
\begin{note}
    Il costo dell'implementazione come vettore ordinato è equivalente
    all'implementazione come \emph{lista} ordinata, tranne per la funzione
    \texttt{contains} che nel vettore ha costa $O(\log n)$ in quanto è
    possibile usare un algoritmo di ricerca dicotomica.
\end{note}

\subsection{Confronto generale}
In generale, se $n$ è il numero di elementi presenti in un \emph{insieme} ed \hspace{-3pt}
$m$ è la capacità di quell'\emph{insieme}, la \emph{complessità} delle operazioni
a seconda dell'implementazione usata è riassunta dalla tabella sottostante.

\begin{table}[h!]
\resizebox{\linewidth}{!}{
\centering
\renewcommand{\arraystretch}{1.2}
\begin{tabular}{|l|c|c|c|c|c|c|}
    \hline
    {
        \diagbox[height=40pt, width=115pt, outerleftsep=-10pt]
        {\textbf{Struttura}}{\textbf{Operazione}}
    } &
    $\begin{array}[c]{c}
        \textbf{\texttt{contains}}\\
        \textbf{\texttt{lookup}}
    \end{array}$ & \textbf{\texttt{insert}} &   
    \textbf{\texttt{remove}} & \textbf{\texttt{min}} & \textbf{\texttt{foreach}}
    & \textbf{Ordinabile}\\
    \hline
    Vettore booleano & $O(1)$ & $O(1)$ & $O(1)$ & $O(m)$ & $O(m)$ & Sì\\
    \hline
    Lista non ordinata & $O(n)$ & $O(n)$ & $O(n)$ & $O(n)$ & $O(n)$ & No\\
    \hline
    Lista ordinata & $O(n)$ & $O(n)$ & $O(n)$ & $O(1)$ & $O(n)$ & Sì\\
    \hline
    Vettore ordinato & $O(\log n)$ & $O(n)$ & $O(n)$ & $O(1)$ & $O(n)$ & Sì\\
    \hline
    Albero bilanciato & $O(\log n)$ & $O(\log n)$ & $O(\log n)$ & $O(\log n)$ & $O(n)$ & Sì\\
    \hline
    $\begin{array}[c]{c}
        \text{Tabella Hash}\\
        \text{Mem. interna}
    \end{array}$ & $O(1)$ & $O(1)$ & $O(1)$ & $O(m)$ & $O(m)$ & No\\
    \hline
    $\begin{array}[c]{c}
        \text{Tabella Hash}\\
        \text{Mem. esterna}
    \end{array}$ & $O(1)$ & $O(1)$ & $O(1)$ & $O(m+n)$ & $O(m+n)$ & No\\
    \hline
\end{tabular}}
\end{table}