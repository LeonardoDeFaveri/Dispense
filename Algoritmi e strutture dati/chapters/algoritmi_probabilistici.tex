\chapter{Algoritmi probabilistici}
L'ultimo argomento che ci rimane da affrontare riguarda i
cosiddetti \emph{algoritmi probabilistici}, ovvero quel genere di algoritmi
che applicano il calcolo delle probabilità ai dati di output. Possiamo
distinguere due categoria: gli algoritmi la cui correttezza è probabilistica
e gli algoritmi che sono corretti, ma hanno un tempo di funzionamento
probabilistico. I primi vengono chiamati \emph{Algoritmi Montecarlo},
i secondi \emph{Algoritmi Las Vegas}.

\section{Algoritmi Montecarlo}
\subsection{Test di primalità}
\begin{problem}[Test di primalità]
    Dato un valore intero $n$, stabilire se è primo oppure no.
\end{problem}

\noindent
L'algritmo naïf per questo problema prova tutti i numeri in $[2,\lfloor
\sqrt{n}\rfloor]$ e se nessuno di essi è un divisore di $n$, allora $n$
è certamente primo.

\begin{minicode}{Algoritmo naïf}
\ind\bc{boolean} isPrime(\bc{int} n)\\
    \indf for (i = 2 to $\lfloor\sqrt{n}\rfloor$) do\\
        \indff if (n / i == $\lfloor$n / i$\rfloor$) then\\
            return false\\
    \indf return true
\end{minicode}

\noindent
Questa prima soluzione è certamente corretta, ma terribilmente inefficiente.

\paragraph{Soluzione basata sul Piccolo Teorema di Fermat}
\begin{definition}[Piccolo Teorema di Fermat]
    Se $n\in\mathbb{N}$ è primo, allora vale:
    \[b^{n-1}\Mod{n}\equiv 1\quad\forall b\in\mathbb{N}\ t.c.\ 2\leq b<n\]
\end{definition}

\begin{minicode}{Prima implementazione della soluzione}
\ind\bc{boolean} isPrime(\bc{int} n)\\
    \bc{int} b = random(2, n - 1)\hfill\com{Restituisce un numero in $[2,n-1]$}
    \indf if (b$^{n - 1}$ mod n $\neq$ 1) then\\
        return false\\
    \indf return true
\end{minicode}

\noindent
Il \emph{Piccolo Teorema di Fermat} è tale per cui se esiste un valore di $b$
per il quale la condizione non è verificata, $n$ certamente non è primo.
Tuttavia, esistono numeri composti per i quali esiste un $b\in[2,n-1]$ tale
per cui $b^{n-1}\Mod{n}\equiv 1$. Questi numeri si chiamano pseudo-primi e
la loro esistenza fa sì che la soluzione che abbiamo realizzato possa
restituire dei falsi positivi.

Per mitigare il rischio di falsi positivi è possibile verificare più volte
la condizione del \emph{Piccolo Teorema di Fermat} per diversi valori di $b$.

\begin{minicode}{Seconda implementazione della soluzione}
\ind\bc{boolean} isPrime(\bc{int} n)\\
    \indf for (i = 1 to k) do\\
        \bc{int} b = random(2, n - 1)\hfill\com{Restituisce un numero in $[2,n-1]$}
        \indff if (b$^{n - 1}$ mod n $\neq$ 1) then\\
            return false\\
    \indf return true
\end{minicode}

\noindent
Questa versione è migliore della precedente, ma per via dei \emph{numeri di
Carmichael}, numeri composti che soddisfano la condizone del
\emph{Piccolo Teorema di Fermat} per ogni $b$, la probabilità di ottenere falsi
positivi rimane alta.

\paragraph{Algoritmo di Miller-Rabin}
Questo algoritmo si basa sul seguente teorema:
\begin{definition}[Teorema di Miller-Rabin]
    Se $n\in\mathbb{N}$ è primo, allora per ogni intero $b$ con $2\leq b<n$
    valgono entrambe le seguenti condizioni:
    \begin{enumerate}
        \item $mcd(n,b)=1$
        \item $b^m\Mod{n}\equiv 1\ \vee\ \exists\; i\in\mathbb{N}\::\:0\leq
        i<v$ per cui $b^{m2^i}\Mod{n}\equiv n-1$
    \end{enumerate}
    per $m$ dispari e $n-1=m2^v$.
\end{definition}

\begin{note}
    La scrittura $n-1=m2^v$ ci dice che la rappresentazione di $n-1$ in
    binario è pari a quella di $m$ seguita da $v$ zeri.
\end{note}

\noindent
Quello che l'\emph{Algoritmo di Miller-Rabin} fa è scegliere casualmente un
valore di $b$ e provare a vedere se almeno una delle proprietà risulta
falsa. Se è così, il numero $n$ sicuramente non è primo. Il problema di
questa strategia è che per alcuni numeri composti esistono valori di $b$
che verificano le due proprietà. I valori di $b$ per i quali $n$ risulta
composto sono chiamati \emph{testimoni} e Rabin ha dimostrato che
per ogni numero composto $n$, ci sono almeno $\frac{3}{4}(n-1)$
\emph{testimoni} nell'intervallo $[2,n-1]$. Di conseguenza, il \emph{test di
primalità} ha una probabilità inferiore a $\frac{1}{4}$ di restituire un falso
positivo. Come per l'altro algoritmo, possiamo ridurre la propabilità di errore
ripetendo più volte il test. Il risultato è un algoritmo come il seguente:

\begin{minicode}{Implementazione della soluzione}
\ind\bc{boolean} isPrime(\bc{int} n)\\
    \indf for (i = 1 to k) do\\
        \bc{int} b = random(2, n - 1)\\
        \indff if (isComposite(n, b)) then\hfill\com{Testa le condizioni di compostezza}
            return false\\
    \indf return true
\end{minicode}

\paragraph{Complessità}
L'\emph{Algoritmo di Miller-Rabin} ha una probabilità di errore di
$\left(\frac{1}{4}\right)^k$ e una \emph{complessità} pari a
$O(k\log^2 n\log\log n\log\log\log n)$.

\subsection{Identificazione di espressioni polinomiali nulle}
\begin{problem}[Identificazione di espressioni polinomiali nulle]
    Data un'espressione algebrica polinomiale $p(x_1,\dots,x_n)$ in $n$
    variabili, stabilire se $p$ è identificamente nulla oppure no.
\end{problem}

\noindent
Questo problema può essere risolto semplificando il polinomio, ma soluzioni
di questo tipo sono molto complesse. Un'alternativa probabilistica è quella
di generare una $n$-upla di valori $v_1,\dots,v_n$. Quindi, si risolve
l'equazione $x=p(v_1,\dots,v_n)$ e, se $x\neq 0$, il polinomio certamente
non è identicamente nullo. Altrimeni, potrebbe esserlo.

È dimostrato che se ogni $v_i$ è un valore intero compreso tra $1$ e $2d$,
dove $d$ è il grado del polinomio, la probabilità di falsi positivi non
supera $\frac{1}{2}$. Quindi ripetendo l'algoritmo $k$ volte, la
probabilità di errore si riduce a $\left(\frac{1}{2}\right)^k$.

\section{Algoritmi Las Vegas}
\subsection{Problema della selezione}
\begin{problem}[Problema della selezione]
    Dato un vettore $A$ contenente $n$ valori e un valore $1\leq k\leq n$,
    trovare l'elemento che occuperebbe la posizione $k$ se il vettore fosse
    ordinato.
\end{problem}
\begin{note}
    Il problema del calcolo della mediana è un sottobroblema della
    \emph{selezione} con $k=\left\lfloor\frac{n}{2}\right\rfloor$.
\end{note}

\noindent
Una soluzione banale è quella di ordinare il vettore pagando $O(n\log n)$ e
cercare il valore in posizione $k$. Possiamo fare meglio di così?

\paragraph{Approccio basato su Heap}
\begin{minicode}{Implementazione soluzione basata su Heap}
\ind\bc{ITEM} heapSelect(\bc{ITEM}[] A, \bc{int} n, \bc{int} k)\\
    buildHeap(A)\\
    \indf for (i = 1 to k - 1) do\\
        deleteMin(A, n)\\
    \indf return deleteMin(A, n)
\end{minicode}

\paragraph{Complessità}
Le funzioni \texttt{heapBuild} e \texttt{deleteMin} costano rispettivamente
$O(n)$ e $O(\log n)$ e, poiché la \texttt{deleteMin} viene invocata $k$ volte,
il costo totale è $O(n+k\log n)$. Se $k=O\left(\frac{n}{\log n}\right)$, la
\emph{complessità} si riduce a $O(n)$, ma se $k=\frac{n}{2}$, si attesta a
$O(n+n\log n)=O(n\log n)$. Da questo deduciamo che la soluzione proposta è
conveniente solo se i valori di $k$ sono molto piccoli. Possono andare bene
anche valori molto vicini a $n$, ma l'algoritmo va modificato in modo che
utilizzi un \emph{albero max-heap} e rimuova ogni volta il valore massimo.

\paragraph{Approccio basato su Quicksort}
Questo algoritmo si comporta come il \emph{\hyperlink{sec:quick_sort}{Quicksort}}
perché sceglie un valore detto \emph{pivot} e sposta alla sua sinistra o alla
sua destra tutti gli elementi del sottovettore con valore inferiore o superiore.
Fatto questo, esprime $k$ rispetto all'indice di inizio del sottovettore che sta
considerando e, se il valore di $k$ coincide con quello appena calcolato, allora
il valore cercato si trova nella posizione del \emph{pivot}. Se $k$ è più piccolo,
la ricerca continua in maniera ricorsiva nel sottovettore sinistro rispetto al
\emph{pivot}, oppure continua in quello destro se $k$ è maggiore. In quest'ultimo,
caso il valore di $k$ viene ridotto in modo da mantenerlo coerente con i limiti
del sottovettore.

\begin{figure}[h!]
\centering
\begin{graph}
    \definecolor{lightblue}{RGB}{129, 208, 226}
    \tikzset{
        cell/.style={rectangle, draw, minimum size=10mm},
        empty/.style={minimum size=10mm}
    }

    \foreach \i in {0,...,15} {
        \ifthenelse{\i = 7}{
            \node[cell, fill=Dandelion] (\i) at (\i, 0) {};
        }{
            \ifthenelse{\i > 2 \and \i < 12}{
                \node[cell, fill=leaf] (\i) at (\i, 0) {};
            }{
                \node[cell] (\i) at (\i, 0) {};
            }
            
        }
    }

    \node[empty] (j) at (7,0) [label=above:{$j$}] {};
    \node[empty] (f) at (3,0) [label=above:{\emph{first}}] {};
    \node[empty] (f) at (11,0) [label=above:{\emph{last}}] {};

    \node[inner sep=0] (i3) at (3,-1) {};
    \node[inner sep=0] (ij) at (7,-1) {};

    \draw[<->] (i3) edge[line width=1.2pt] node[midway, below] {$q$} (ij);
\end{graph}
\end{figure}

\begin{minicode}{Implementazione della soluzione basata su Quicksort}
\ind\bc{ITEM} selection(\bc{ITEM}[] A, \bc{int} first, \bc{int} last, \bc{int} k)\\
    \indf if (first == last) then\\
        return A[first]\\

    \indf\bc{int} j = pivot(A, first, last)\\
    \indf\bc{int} q = j - first + 1\hfill\com{Indice del valore ricercato rispetto al sottovettore}
    \indf if (k == q) then\\
        return A[j]\\
    \indf else if (k < q) then\\
        return selection(A, first, j - 1, k)\\
    \indf else\\
        return selection(A, j + 1, last, k - q)
\end{minicode}

\paragraph{Complessità}
La similitudine con il \emph{Quicksort} si ripresenta anche nella
\emph{complessità} perché, se il vettore è ordinato e di volta in volta
viene scelto come \emph{pivot} il primo elemento del sottovettore,
l'\emph{equazione di ricorrenza} assume la seguente forma:
\[T(n)=\begin{cases}
    1 & n\leq 1\\
    T(n-1)+n & n>1
\end{cases}\]
Per il \emph{\nameref{def:21}}, la \emph{forma chiusa} è $T(n)=O(n^2)$.

Allo stesso modo, se ad ogni scelta del \emph{pivot} il sottovettore viene
diviso in due metà, rientriamo nel caso ottimo e l'\emph{equazione di ricorrenza}
vale:
\[T(n)=\begin{cases}
    1 & n\leq 1\\
    T(n/2)+n & n>1
\end{cases}\]
Per il \emph{\nameref{def:19}}, la \emph{forma chiusa} è $T(n)=O(n)$.

\bigskip\noindent
Come si comporta l'algoritmo nel caso medio?

Per rispondere a questa domanda proviamo ad assumere che la funzione
\texttt{pivot} restituisca con la stessa probabilità una qualsiasi posizione
del vettore $A$. In questo modo, la \emph{funzione di ricorrenza} diventa
$n$ più la media fra il costo di tutte le possibili scelte del \emph{pivot},
ovvero:
\[\begin{array}{rcll}
    T(n) & = & n+\frac{1}{n}\sum_{q=1}^n T(\max\{q-1,n-q\}) & \quad\text{Media su $n$ casi}\\
    & \leq & n+\frac{1}{n}\sum_{q=\lfloor n/2\rfloor}^{n-1}2T(q) & \quad n>1\\
    & \leq & n+\frac{1}{n}\sum_{q=\lfloor n/2\rfloor}^{n-1}2cq & \quad\text{Sostituzione}\\
    & \leq & n+\frac{2c}{n}\sum_{q=\lfloor n/2\rfloor}^{n-1}q & \quad\text{Raccoglimento $2c$}\\
    & = & n+\frac{2c}{n}\left(\sum_{q=1}^{n-1}q-\sum_{q=1}^{\lfloor n/2\rfloor-1}q\right) & \quad\text{Sottrazione prima parte}\\
    & = & n+\frac{2c}{n}\left(\frac{n(n-1)}{2}-\frac{\lfloor n/2\rfloor
    \left(\lfloor n/2\rfloor -1\right)}{2}\right)\\
    & \leq & n+\frac{2c}{n}\left(\frac{n(n-1)}{2}-\frac{\left(n/2-1\right)\left(n/2-2\right)}{2}
    \right) & \quad\text{Rimozione limite inferiore}\\
    & = & n+\frac{2c}{n}\frac{\left(n^2-n-\left(\frac{1}{4}n^2-\frac{3}{2}n+2\right)\right)}{2}\\
    & = & n+\frac{c}{n}\left(n^2-n-\left(\frac{1}{4}n^2-\frac{3}{2}n+2\right)\right) & \quad\text{Semplificazione del $2$}\\
    & = & n+\frac{c}{n}\left(\frac{3}{4}n^2+\frac{1}{2}n-2\right)\\
    & \leq & n+\frac{c}{n}\left(\frac{3}{4}n^2+\frac{1}{2}n\right) & \quad\text{Rimozione del $-2$}\\
    & = &n+\frac{3}{4}cn+\frac{1}{2}c\\
    & \leq & cn
\end{array}\]
L'ultima disequazione è vera per $c>4$ e $n\geq\frac{2c}{c-4}$, quindi la \emph{complessità}
del caso medio è $T(n)=O(n)$. Questa soluzione è probabilistica perché il tempo
d'esecuzione dipende dalla selezione dei \emph{pivot}.

\begin{note}
    Per forzare la funzione \texttt{pivot} a sceglire equiprobabilisticamente
    tutti i valori compresi in $[1,n]$, possiamo forzare la cosa scambiando il
    valore in una posizione casuale con il primo: \texttt{A[random(first, last)]
    $\leftrightarrow$ A[first]}\footnotemark.
\end{note}

\footnotetext{L'implementazione di \texttt{pivot} sceglie sempre l'elemento
in prima posizione}

\paragraph{Approccio deterministico}
Se supponessimo di avere a disposizione una funzione in grado di restituire in
tempo $O(n)$ un valore che dista al più $\frac{3}{10}n$ dal mediano di un vettore,
potremmo usare quella funzione per risolvere il \emph{problema della selezione}.

L'idea sarebbe poi quella di dividere gli $n$ valori in gruppi di $5$. Quindi,
per ogni gruppo $S_i$, con $i\in[1,\lceil n/5\rceil]$, usare la suddetta funzione
per ricavare il mediano $M_i$. A quel punto, usando la stessa funzione si può
ricavare il mediano $m$ delle mediane $[M_1,\dots,M_{\lceil n/5\rceil}]$, e usare
quell'$m$ come \emph{pivot} per poi richiamare ricorsivamente l'algoritmo
sul sottovettore opportuno, seguendo lo stesso approccio della soluzione
basata su \emph{Quicksort}. Una volta raggiunto un sottovettore sufficientemente
piccolo si potrebbe usare un algoritmo di ordinamento per determinare il valore
ricercato.

\begin{figure}[h!]
\centering
\begin{graph}
    \tikzset{
        cell/.style={rectangle, draw, minimum size=10mm},
        separator/.style={rectangle, draw, minimum height=10mm, minimum width=1mm, inner sep=0}
    }

    \def\numbers{1 4 7 10 13 2 5 8 11 14 3 6 9 12 15}
    \readarray\numbers\num[1,15]

    \foreach \i in {1,...,5} {
        \ifthenelse{\i = 3}{
            \node[cell] (\i) at (\i) {\bm{$\num[1,\i]$}};
        }{
            \node[cell] (\i) at (\i) {$\num[1,\i]$};
        }
    }
    \node[separator] (s1) at (5) [xshift=5.52mm] {};
    \foreach \i in {6,...,10} {
        \ifthenelse{\i = 8}{
            \node[cell] (\i) at (\i) [xshift=1mm] {\bm{$\num[1,\i]$}};
        }{
            \node[cell] (\i) at (\i) [xshift=1mm] {$\num[1,\i]$};
        }
    }
    \node[separator] (s1) at (9) [xshift=15.52mm] {};
    \foreach \i in {11,...,15} {
        \ifthenelse{\i = 13}{
            \node[cell] (\i) at (\i) [xshift=2mm] {\bm{$\num[1,\i]$}};
        }{
            \node[cell] (\i) at (\i) [xshift=2mm] {$\num[1,\i]$};
        }
    }
\end{graph}
\caption{La mediana delle mediane è $8$ che è anche la mediana dell'intero vettore}
\end{figure}

\newpage
\begin{minicode}{Implementazione soluzione deterministica}
\ind\bc{ITEM} select(\bc{ITEM}[] A, \bc{int} first, \bc{int} last, \bc{int} k)\\
    \com{Se la dimensione è inferiore a 10, ordina il vettore e resituisce}
    \com{il $k$-esimo elemento di $A[first\dots last]$}
    \indf if (last - first + 1 $\leq$ 10) then\\
        insertionSort(A, first, last)\hfill\com{Ordina unicamente il sottovettore}
        return A[first + k - 1]\\
    \indf\com{Divide $A$ in $\lceil n/5\rceil$ sottovettori di dimensione $5$ e
    ne calcola la mediana}
    \indf\bc{ITEM}[] M = new \bc{ITEM}[1\dots$\lceil$n / 5$\rceil$]\\
    \indf for (i = 1 to $\lceil$n / 5$\rceil$) do\\
        M[i] = median5(A, first + (i - 1) $\cdot$ 5, last)\\
    \indf\com{Individua la mediana delle mediane e la usa come \emph{pivot}}
    \indf\bc{ITEM} m = select(M, 1, $\lceil$n / 5$\rceil$, $\lceil\lceil$n / 5$\rceil$ / 2$\rceil$)\\
    \indf\bc{int} j = pivot(A, first, last, m)\hfill\com{$m$ è il \emph{pivot} e $j$ ne è l'indice}
    \indf\bc{int} q = j - first + 1\hfill\com{Indice del valore ricercato rispetto al sottovettore}
    \indf if (k == q) then\\
        return m\\
    \indf else if (k < q) then\\
        return select(A, first, j - 1, k)\\
    \indf else\\
        return select(A, j + 1, last, k - q)
\end{minicode}

\paragraph{Complessità}
Il calcolo dei mediani $M_i$ richiede al più $6\lceil n/5\rceil$ confronti, la
prima invocazione di \texttt{select} opera su $\lceil n/5\rceil$ elementi, mentre
la seconda su un massimo di $\frac{7}{10}n$ o, più esattamente, su $n-3\cdot\lceil
\lceil n/5\rceil/2\rceil$ elementi. Quindi, la \texttt{select} esegue $O(n)$
confronti nel caso pessimo. La \emph{funzione di ricorrenza} è la seguente:
\[T(n)=T(n/5)+T(7n/10)+\frac{11}{5}n=O(n)\]

\section{Bloom filter}
Concludiamo il capitolo esaminando i \emph{bloom filters}, una \emph{struttura
dati probabilistica} ottenuta unendo i \emph{bit set} alle \emph{tabelle hash}. Il
risultato è quindi una \emph{struttura dinamica} che utilizza pochi bit (e.g. 10
bit) per ogni oggetto, non memorizza i dati e quindi non consente cancellazioni e
ha, come preannunciato, una risposta probabilistica nelle letture.

\paragraph{Specifica}
\begin{code}{Bloom filter}
    \com{Inserisce l'elemento $k$ nella \emph{struttura}}
    insert(\bc{ITEM} k)
    \nl\com{Controlla se $k$ è presente o meno nella struttura}
    \com{Se restituisce \textbf{false}, $k$ è certamente assente, altrimenti}
    \com{potrebbe essere presente oppure no}
    \bc{boolean} container(\bc{ITEM} k)
\end{code}

\noindent
La probabilità di ottenere falsi positivi diminuisce all'aumentare dei bit usati
per ogni oggetto. In ogni implementazione è quindi importante raggiungere un
trade-off tra occupazione di memoria e probabilità di errore. In particolare,
se $\epsilon$ è la probabilità di errore che si vuole garantire, il numero di
bit che devono essere usati per ogni oggetto è $1.44\log(\frac{1}{\epsilon})$.

\paragraph{Implementazione}
L'idea è che inizialmente la struttura sia costituita da un vettore di bit
inizializzati $0$. Quando un valore deve essere inserito, vengono calcolate le
posizioni dei bit che rappresentano l'oggetto utilizzando $H$ funzioni hash.
Ciascuna funzione restituisce una posizione e il bit corrispondente viene
impostato a $1$. Similmente, per la verifica di esistenza, si controllano i
bit nelle posizioni in cui dovrebbe trovarsi l'elemento e, se almeno uno è $0$,
l'elemento è certamente assente. Caso contrario, potrebbe essere presente o
potrebbe essere un falso positivo. Quest'ultima possibilità è dovuto al fatto
che elementi diversi potrebbero condividere parte dei bit.

\begin{minicode}{Bloom filter}
\ind insert(\bc{ITEM} k)\\
    \indf for (i = 1 to H) do\\
        A[h$_i$(k)] = true
\nl\ind\bc{boolean} contains(\bc{ITEM} k)\\
    \indf for (i = 1 to H) do\\
        \indff if (A[h$_i$(k)] == false) then\\
            return false\\
    \indf return true
\end{minicode}

\begin{figure}[h!]
\centering
\subfloat[Situazione iniziale]{
\resizebox{0.48\textwidth}{!}{\begin{graph}
    \tikzset{
        cell/.style={rectangle, draw, minimum size=10mm}
    }

    \foreach \i in {0,...,11} {
        \node[cell] (\i) [xshift=10mm*\i] {$0$};
    }
\end{graph}}}
\hfill
\subfloat[Situazione dopo 3 inserimenti]{
\resizebox{0.48\textwidth}{!}{\begin{graph}
    \tikzset{
        cell/.style={rectangle, draw, minimum size=10mm}
    }

    \def\numbers{0 1 0 1 0 1 0 0 1 1 1 1 1 0}
    \readarray\numbers\num[1,12]

    \foreach \i in {1,...,12} {
        \node[cell] (\i) [xshift=10mm*(\i-1)] {$\num[1,\i]$};
    }
    
    \node[cell] (k1) [above of=3, yshift=10mm] {$k_1$};
    \node[cell] (k2) [above of=6, yshift=10mm] {$k_2$};
    \node[cell] (k3) [above of=9, yshift=10mm] {$k_3$};

    \path[->]   (k1) edge[out=-90, in=90, looseness=1] (2)
                (k1) edge[out=-90, in=90, looseness=1] (6)
                (k1) edge[out=-90, in=90, looseness=1] (10)
                (k2) edge[out=-90, in=90, looseness=1] (4)
                (k2) edge[out=-90, in=90, looseness=1] (6)
                (k2) edge[out=-90, in=90, looseness=1] (9)
                (k3) edge[out=-90, in=90, looseness=1] (9)
                (k3) edge[out=-90, in=90, looseness=1] (11)
                (k3) edge[out=-90, in=90, looseness=1] (12);
\end{graph}}}\\
\subfloat[Test di esistenza per un elemento presente]{
\resizebox{0.48\textwidth}{!}{\begin{graph}
    \tikzset{
        cell/.style={rectangle, draw, minimum size=10mm}
    }

    \def\numbers{0 1 0 1 0 1 0 0 1 1 1 1 1 0}
    \readarray\numbers\num[1,12]

    \foreach \i in {1,...,12} {
        \node[cell] (\i) [xshift=10mm*(\i-1)] {$\num[1,\i]$};
    }
    
    \node[cell] (k2) [above of=6, yshift=10mm] {$k_2$};

    \path[->]   (k2) edge[out=-90, in=90, looseness=1] (4)
                (k2) edge[out=-90, in=90, looseness=1] (6)
                (k2) edge[out=-90, in=90, looseness=1] (9);
\end{graph}}}
\hfill
\subfloat[Test di esistenza per un elemento non presente]{
\resizebox{0.48\textwidth}{!}{\begin{graph}
    \tikzset{
        cell/.style={rectangle, draw, minimum size=10mm}
    }

    \def\numbers{0 1 0 1 0 1 0 0 1 1 1 1 1 0}
    \readarray\numbers\num[1,12]

    \foreach \i in {1,...,12} {
        \node[cell] (\i) [xshift=10mm*(\i-1)] {$\num[1,\i]$};
    }
    
    \node[cell] (k4) [above of=6, yshift=10mm] {$k_4$};

    \path[->]   (k4) edge[out=-90, in=90, looseness=1] (1)
                (k4) edge[out=-90, in=90, looseness=1] (7)
                (k4) edge[out=-90, in=90, looseness=1] (9);
\end{graph}}}\\
\subfloat[Test di esistenza per un falso positivo]{
\resizebox{0.48\textwidth}{!}{\begin{graph}
    \tikzset{
        cell/.style={rectangle, draw, minimum size=10mm}
    }

    \def\numbers{0 1 0 1 0 1 0 0 1 1 1 1 1 0}
    \readarray\numbers\num[1,12]

    \foreach \i in {1,...,12} {
        \node[cell] (\i) [xshift=10mm*(\i-1)] {$\num[1,\i]$};
    }
    
    \node[cell] (k5) [above of=6, yshift=10mm] {$k_5$};

    \path[->]   (k4) edge[out=-90, in=90, looseness=1] (2)
                (k4) edge[out=-90, in=90, looseness=1] (6)
                (k4) edge[out=-90, in=90, looseness=1] (9);
\end{graph}}}
\caption{Esempio di evoluzione di un \emph{bloom filter}}
\end{figure}

\paragraph{Alcune formule}
\begin{definition}[Probabilità di errore]
    In un bloom filter costruito su $m$ bit totali e $H$ funzioni hash e
    che contiene $n$ elementi, la probabilità $\epsilon$ di ottenere un falso
    positivo è:
    \[\epsilon=\left(1-e^{\frac{-Hn}{m}}\right)^H\]    
\end{definition}
\begin{definition}[Numero ottimale di funzioni hash]
    In un bloom filter costruito su $m$ bit totali e contenente $n$
    elementi, il numero ottimale di funzioni hash è:
    \[H=\frac{m}{n}\ln 2\]
\end{definition}
\begin{definition}[Numero di bit totali richiesti]
    In un bloom filter contenente $n$ elementi e caratterizzato da una probabilità
    d'errore $\epsilon$, il numero di bit totali $m$ richiesti è:
    \[m=-\frac{n\ln\epsilon}{(\ln 2)^2}\]
\end{definition}

\paragraph{Casi d'uso}
Questo tipo di \emph{struttura dati} è molto utilizzato dai browser per
memorizzare gli URL pericolosi. In particolare, ad ogni richiesta il browser
verifica l'esistenza dell'URL cercato nella struttura e, se è presente,
probabilmente si tratta di una destinazione pericolosa e quindi è necessario
agire di conseguenza. Se invece non è presente, la destinazione è sicura.