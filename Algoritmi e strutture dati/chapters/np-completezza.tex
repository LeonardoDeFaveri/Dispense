\chapter{Problemi intrattabili e teoria dell'NP-completezza}
\section{Introduzione}
Finora, con l'unica eccezione della sezione sul \emph{backtracking}, abbiamo
trattato unicamente problemi con soluzioni in tempo \emph{polinomiale}, ovvero
problemi le cui soluzioni sono eseguibili in tempo $O(n^k)$ con $k\in\mathbb{R}^+$.
Esistono anche problemi che richiedono un tempo \emph{esponenziale} o che
addirittura non sono risolvibili (e.g. halting problem).

Ciò che andremo ad introdurre in questo capitolo invece, sono una serie di
problemi per i quali non è chiaro se esista una soluzione \emph{polinomiale} o
meno. Vedremo anche come tutti questi problemi siano in realtà legati tra loro,
in modo che se esiste una soluzione \emph{polinomiale} per uno di essi, allora ne
esiste una anche per gli altri. Viceversa, se si riesce a dimostrare che uno
di essi non è risolvibile in tempo \emph{polinomiale}, allora non lo è nessuno.

\bigskip\noindent
Procediamo dando alcune definizioni.

\begin{definition}[Problema astratto]
    Un problema astratto è una relazione binaria $R\subseteq I\times S$ tra un
    insieme $I$ di istanze del problema e un insieme $S$ di soluzioni.
\end{definition}
\begin{note}
    Ad esempio, nel problema della \emph{ricerca del cammino minimo tra due
    nodi}, un'istanza del problema è la tupla $(V,R,u,v)$, mentre una soluzione
    è una sequenza di \emph{nodi} $(v_1,\dots,v_n)$.
\end{note}

\noindent
Come visto nei capitoli scorsi, esistono varie tipologie di problema. Le
principali sono tre: \emph{ottimizzazione}, \emph{ricerca} e \emph{decisione}.

\begin{definition}[Problema di ottimizzazione]
    Data un'istanza, trovare la soluzione ottima secondo un insieme di criteri
    prestabiliti.
\end{definition}
\begin{definition}[Problema di ricerca]
    Data un'istanza, trovare una possibile soluzione tra quelle esistenti.
\end{definition}
\begin{definition}[Problema di decisione]
    Data un'istanza, verificare se soddisfa o meno una data proprietà.
\end{definition}
\begin{note}
    Nei \emph{problemi di decisione}, $R$ è una funzione del tipo $R:I\to\{0,1\}$.
\end{note}

\noindent
Informalmente, possiamo dimostrare che i \emph{problemi di ottimizzazione} e di
\emph{decisione} sono equivalenti.

\begin{proof}[Dimostrazione]
    Se è possibile risolvere efficientemente un \emph{problema di ottimizzazione},
    allora è possibile usare la soluzione a quel problema per verificare
    efficientemente la proprietà interessata dal \emph{problema di decisione}
    associato.

    Da questa affermazione possiamo derivare la seguente: se non è possibile
    risolvere efficientemente un \emph{problema di decisione}, allora non è
    nemmeno possibile risolvere efficientemente il \emph{problema di
    ottimizzazione} associato.
\end{proof}
\begin{note}
    Ad esempio, nel problema della \emph{ricerca del cammino tra due nodi}, se
    si conosce il \emph{cammino minimo} tra essi, è possibile possibile 
    risolvere efficientemente un \emph{problema di decisione} nel quale ci si
    chiede se esista un \emph{cammino} di lunghezza non superiore a un qualche
    valore $k$.
\end{note}

\noindent
Questa dimostrazione, sebbene informale, ci permette di proseguire la trattazione
concentrandoci unicamente su \emph{problemi decisionali}, che sono più facili sia
da definire che da elaborare.

\section{Riduzioni}
\begin{definition}[Riduzione polinomiale]
    Dati due problemi decisionali $R_1\subseteq I_1\times\{0,1\}$ e
    $R_2\subseteq I_2\times\{0,1\}$, diciamo che $R_1$ è riducibile
    polinomialmente a $R_2$, e scriviamo in simboli $R_1\leq_p R_2$, se esiste
    una funzione $f:I_1\to I_2$ che sia calcolabile in tempo polinomiale e tale
    per cui, per ogni istanza $x$ del problema $R_1$ e ogni soluzione $s\in\{0,1\}$,
    sia vero che $(x,s)\in R_1\Leftrightarrow(f(x), s)\in R_2$.
\end{definition}

\begin{figure}[h!]
\centering
\scalebox{1}{\begin{graph}
    \definecolor{pink}{RGB}{255, 204, 204}
    \definecolor{yellow}{RGB}{255, 251, 214}
    \tikzset{
        cell/.style={fill=pink, draw, rectangle, minimum size=12mm, inner sep=0},
        bg/.style={fill=yellow, draw, rectangle, inner sep=0, minimum size=20mm}
    }
    
    \node[bg] (1) [minimum width=70mm, label={[yshift=7mm]-90:{$A_1$}}] {};
    \node[cell] (f) at (1) [xshift=-25mm] {$f$};
    \node[cell] (a2) at (1) [xshift=25mm] {$A_2$};

    \node[] (0) at (f) [xshift=-35mm] {};
    \node[] (1) at (a2) [xshift=35mm] {};

    \draw[->]   (f) edge node[above] {$f(x)\in I_2$} (a2);
    \draw[->]   (0) edge node[above] {$x\in I_1$} (f);
    \draw[->]   (a2) edge node[above] {$s\in\{0,1\}$} (1);
\end{graph}}
\caption{\emph{Riduzione polinomiale}}
\end{figure}

\noindent
Proseguiamo la trattazione con una carrellata di problemi che, come vedremo,
possono essere legati tra loro mediante \emph{riduzioni polinomiali}.

\subsection{Colorazione di grafi}
\begin{definition}[Colorazione di grafi]
    Dati un grafo non orientato $G=(V,E)$ e un insieme di colori $C$, una
    colorazione dei vertici è una funzione $f:V\to C$ che assegna ad ogni nodo
    uno dei colori in $C$ in maniera tale per cui nessuna coppia di nodi adiacenti
    ha lo stesso colore
\end{definition}
\begin{problem}[Colorabilità di un grafo (GRAPH-COLORING)]
    Dato un grafo non orientato $G=(V,E)$ e un valore $k$, determinare se
    esiste una colorazione di $G$ con $k$ colori.
\end{problem}

\subsection{Sudoku}
\begin{problem}[Risolvibilità di un sudoku (SUDOKU)]
    Data una matrice $n^2\times n^2$ con alcuni numeri già inseriti, determinare
    se esiste un modo per assegnare i numeri restanti in modo coerente con le
    regole dei Sodoku.
\end{problem}

\noindent
Il problema sulla \emph{risolvibilità di un sudoku} può essere \emph{ridotto
polinomialmente} al problema sulla \emph{colorabilità di un grafo}. In
particolare, è possibile tradurre la matrice del sudoku in un \emph{grafo} in
cui ogni valore della matrice diventa un \emph{nodo} del \emph{grafo} e in cui
tra due \emph{nodi} esiste un \emph{arco} soltanto se, all'interno della matrice,
quei \emph{nodi} sono sulla stessa riga, sulla stessa colonna o sulla stessa
diagonale. Formalmente gli insiemi dei \emph{nodi} e degli \emph{archi} sono
definiti come segue:
\[V=\left\{(x,y):1\leq x\leq n^2, 1\leq y\leq n^2\right\}\]
\[E=\left\{[(x,y),(x',y')]: x=x' \vee y=y' \vee \left(
    \left\lceil\frac{x}{n}\right\rceil =
    \left\lceil\frac{x'}{n}\right\rceil \wedge
    \left\lceil\frac{y}{n}\right\rceil = \left\lceil\frac{y'}{n}\right\rceil
\right)\right\}\]
L'insieme dei colori è $C=\{1,\dots,n\}$.

\begin{figure}[h!]
\centering

\resizebox*{0.98\textwidth}{!}{
    \begin{graph}
        \definecolor{r}{rgb}{0.9, 0.17, 0.31}
        \definecolor{y}{rgb}{0.93, 0.53, 0.18}
        \definecolor{g}{rgb}{0, 0.42, 0.24}
        \definecolor{b}{rgb}{0, 0, 0 0}
        \definecolor{lavendergray}{rgb}{0.77, 0.76, 0.82}
        \def\numbers{%
            1 2 3 4
            3 4 1 2
            2 3 4 1
            4 1 2 3
        }
        \readarray\numbers\num[4,4]
        \def\colors{%
            r y g b
            g b r y
            y b b r
            b r y g
        }
        \readarray\colors\col[4,4]

        \tikzset{
          noder/.style={circle, draw, minimum size=5mm, fill=r},
          nodey/.style={circle, draw, minimum size=5mm, fill=y},
          nodeg/.style={circle, draw, minimum size=5mm, fill=g},
          nodeb/.style={circle, draw, minimum size=5mm, fill=b},
          empty/.style={inner sep=0em, minimum size=10mm},
          cell/.style={rectangle, draw=lavendergray, minimum size=10mm, font=\large},
          square/.style={rectangle, draw, minimum size=20mm},
          node distance=15mm
        }

        \foreach \x in {1,...,4}
            \foreach \y in {1,...,4}
            {
                \node[cell] (m\x\y) at (\x-1,-\y-1) [text={\col[\y,\x]}] {$\num[\y,\x]$};
            }

        \node[square] (q1) at (m11) [xshift=5mm, yshift=-5mm, label=above left:{$A$}] {};
        \node[square] (q2) at (m13) [xshift=5mm, yshift=-5mm, label=below left:{$C$}] {};
        \node[square] (q3) at (m31) [xshift=5mm, yshift=-5mm, label=above right:{$B$}] {};
        \node[square] (q4) at (m33) [xshift=5mm, yshift=-5mm, label=below right:{$D$}] {};

        \node[empty] (arrow) [right of=m43, yshift=5mm] {$\Longrightarrow$};
        \node[empty] (0) [right of=arrow, xshift=55mm] {};
        \node[empty] (w) [left of=0, xshift=-25mm] {};
        \node[empty] (e) [right of=0, xshift=25mm] {};
        \node[empty] (s) [below of=0, yshift=-25mm] {};
        \node[empty] (n) [above of=0, yshift=25mm] {};

        \node[nodeg] (w1) [above of=w] {};
        \node[nodey] (w2) [above left of=w, yshift=-5mm] {};
        \node[noder] (w3) [below left of=w, yshift=5mm] {};
        \node[nodeb] (w4) [below of=w] {};

        \node[noder] (n1) [left of=n] {};
        \node[nodey] (n2) [above left of=n, xshift=5mm] {};
        \node[nodeg] (n3) [above right of=n, xshift=-5mm] {};
        \node[nodeb] (n4) [right of=n] {};

        \node[nodeg] (e1) [above of=e] {};
        \node[nodeb] (e2) [above right of=e, yshift=-5mm] {};
        \node[noder] (e3) [below right of=e, yshift=5mm] {};
        \node[nodey] (e4) [below of=e] {};

        \node[nodey] (s1) [right of=s] {};
        \node[nodeg] (s2) [below right of=s, xshift=-5mm] {};
        \node[nodeb] (s3) [below left of=s, xshift=5mm] {};
        \node[noder] (s4) [left of=s] {};

        % Archi tra nodi della stessa matrice nxn: solo archi dritti
        \path[-]    (w1) edge (w2)
                    (w2) edge (w3)
                    (w3) edge (w4)
                    (n1) edge (n2)
                    (n2) edge (n3)
                    (n3) edge (n4)
                    (e1) edge (e2)
                    (e2) edge (e3)
                    (e3) edge (e4)
                    (s1) edge (s2)
                    (s2) edge (s3)
                    (s3) edge (s4);

        \path[-, bend right=10]
                    (w3) edge (w1)
                    (w4) edge (w1)
                    (w4) edge (w2)
                    (n1) edge (n3)
                    (n1) edge (n4)
                    (n2) edge (n4)
                    (e1) edge (e3)
                    (e1) edge (e4)
                    (e2) edge (e4)
                    (s1) edge (s3)
                    (s1) edge (s4)
                    (s2) edge (s4);

        \path[-, bend right=10]
                    (w1) edge (n4)
                    (n4) edge (e4)
                    (e4) edge (s4)
                    (s4) edge (w1)
                    (w2) edge (n3)
                    (n3) edge (e3)
                    (e3) edge (s3)
                    (s3) edge (w2)
                    (w3) edge (n2)
                    (n2) edge (e2)
                    (e2) edge (s2)
                    (s2) edge (w3)
                    (w4) edge (n1)
                    (n1) edge (e1)
                    (e1) edge (s1)
                    (s1) edge (w4);

        % Archi orizzontanti e verticali
        \path[-]    (w1) edge (e4)
                    (w2) edge (e3)
                    (w3) edge (e2)
                    (w4) edge (e1)
                    (n1) edge (s1)
                    (n2) edge (s2)
                    (n3) edge (s3)
                    (n4) edge (s4);

        \draw[-] (w1)+(0,5mm) arc(90:270:1.5cm and 2cm) node[midway, left] {$D$};
        \draw[-] (n1)+(-5mm,0) arc(180:0:2cm and 1.5cm) node[midway, above] {$A$};
        \draw[-] (e1)+(0,5mm) arc(90:270:-1.5cm and 2cm) node[midway, right] {$B$};
        \draw[-] (s1)+(5mm,0) arc(0:180:2cm and -1.5cm) node[midway, below] {$C$};
    \end{graph}
}
\caption{Trasformazione \emph{matrice-grafo}}
\end{figure}

\noindent
Questo significa che $\text{SUDOKU}\leq_p\text{GRAPH-COLORING}$, quindi una
soluzione al problema della colorazione può essere usata per risolvere il
problema del sudoku.

\subsection{Insieme indipendente}
\begin{definition}[Insieme indipendente]
    Dato un grafo orientato $G=(V,E)$, un insieme $S\subseteq V$ è un insieme
    indipendente se e solo se nessun arco in $E$ unisce due nodi in $S$. Ovvero,
    $S$ è un insieme indipendente se vale la seguente relazione:
    \[x\notin S\vee y\notin S\quad\forall(x,y)\in E\]
\end{definition}