\chapter{Backtracking}
\section{Introduzione}
La tecnica del \emph{backtrack} è usata per risolvere problemi in cui è
necessario esplorare l'intero spazio delle soluzioni o quando la soluzione
va ricercata in un insieme molto ampio. Andando maggiormente nello specifico,
il \emph{backtracking} può essere utilizzato nelle seguenti categorie di
problemi:
\begin{itemize}
    \item \emph{Enumerazioni}: problemi che richiedono di elencare tutte le
    soluzioni ammissibili;
    \item \emph{Conteggio}: problemi che richiedono di contare tutte le
    soluzioni ammissibili;
    \item \emph{Ricerca}: problemi che richiedono di trovare una soluzione
    ammissibile in uno spazio delle soluzioni molto grande;
    \item \emph{Ottimizzazione}: problemi che richiedono di trovare una delle
    soluzioni ammissibili ottime rispetto ad un certo criterio di valutazione;
\end{itemize}
\begin{note}
    Il \emph{backtracking} è paragonabili alla tecnica del \emph{brute force}.
\end{note}

\noindent
Il problema che sorge quando si sceglie di analizzare l'intero spazio delle
soluzioni è che questo potrebbe avere una dimensione \emph{superpolinomiale}
e richiedere un tempo inaccettabile per essere esaminato. Per questo motivo,
è consigliato usare il \emph{backtracking} solo quando è l'unica via possibile.

Ciò che ci interessa però, è che la tecnica del \emph{backtrack} ci fornisce un
approccio sistematico all'esplorazione di uno spazio di ricerca, utilizzando la
ricorsione per memorizzare le scelte fatte e consentendoci di \q{ritornare sui
nostri passi} qualora quelle scelte si rivelassero errate o non ottimali.

\begin{note}
    Per alcuni problemi è possibile realizzare anche soluzioni iterative
    unendo approcci \emph{greedy} alla possibilità di annullare le scelte fatte.
\end{note}

\section{Problema dell'elencazione dei sottoinsiemi}
\begin{problem}[Problema dell'elencazione dei sottoinsiemi]
    Dato un insieme $\{1,\dots,n\}$ elencarne tutti i sottoinsiemi.
\end{problem}

\noindent
Rappresentiamo una soluzione come un vettore di scelte $S[1\dots n]$ nel quale
il contenuto di $S[i]$ è preso tra i valori dell'insieme originale
non ancora scelti. In particolare, l'indice $i$ rappresenta l'indice della
prossima decisione da prendere e la soluzione parziale $S[1\dots i-1]$
contiene tutte le decisioni prese finora.

Procedendo in modo ricorsivo possiamo dire che nel caso base $S[1\dots i-1]$ è
una soluzione ammissibile e quindi può essere processata facendo terminare la
ricorsione. Altrimenti, calcoliamo l'insieme $C$ delle scelte possibili, quindi,
per ogni $c\in C$ scriviamo $c$ nella scelta $S[i]$ e continuiamo la ricerca per
$i+1$.