\chapter{Backtracking}
\section{Introduzione}
La tecnica del \emph{backtrack} è usata per risolvere problemi in cui è
necessario esplorare l'intero spazio delle soluzioni o quando la soluzione
va ricercata in un insieme molto ampio. Andando maggiormente nello specifico,
il \emph{backtracking} può essere utilizzato nelle seguenti categorie di
problemi:
\begin{itemize}
    \item \emph{Enumerazioni}: problemi che richiedono di elencare tutte le
    soluzioni ammissibili;
    \item \emph{Conteggio}: problemi che richiedono di contare tutte le
    soluzioni ammissibili;
    \item \emph{Ricerca}: problemi che richiedono di trovare una soluzione
    ammissibile in uno spazio delle soluzioni molto grande;
    \item \emph{Ottimizzazione}: problemi che richiedono di trovare una delle
    soluzioni ammissibili ottime rispetto ad un certo criterio di valutazione;
\end{itemize}
\begin{note}
    Il \emph{backtracking} è paragonabili alla tecnica del \emph{brute force}.
\end{note}

\noindent
Il problema che sorge quando si sceglie di analizzare l'intero spazio delle
soluzioni è che questo potrebbe avere una dimensione \emph{superpolinomiale}
e richiedere un tempo inaccettabile per essere esaminato. Per questo motivo,
è consigliato usare il \emph{backtracking} solo quando è l'unica via possibile.

Ciò che ci interessa però, è che la tecnica del \emph{backtrack} ci fornisce un
approccio sistematico all'esplorazione di uno spazio di ricerca, utilizzando la
ricorsione per memorizzare le scelte fatte e consentendoci di \q{ritornare sui
nostri passi} qualora quelle scelte si rivelassero errate o non ottimali.

\begin{note}
    Per alcuni problemi è possibile realizzare anche soluzioni iterative
    unendo approcci \emph{greedy} alla possibilità di annullare le scelte fatte.
\end{note}

\section{Problema dell'enumerazione dei sottoinsiemi}
\begin{problem}[Problema dell'enumerazione dei sottoinsiemi]
    Dato un insieme $\{1,\dots,n\}$ elencarne tutti i sottoinsiemi.
\end{problem}

\noindent
Rappresentiamo una soluzione come un vettore di scelte $S[1\dots n]$ nel quale
il contenuto di $S[i]$ è preso tra i valori dell'insieme originale
non ancora scelti. In particolare, l'indice $i$ rappresenta l'indice della
prossima decisione da prendere e la soluzione parziale $S[1\dots i-1]$
contiene tutte le decisioni prese finora.

Procedendo in modo ricorsivo possiamo dire che nel caso base $S[1\dots i-1]$ è
una soluzione ammissibile e quindi può essere processata facendo terminare la
ricorsione. Altrimenti, calcoliamo l'insieme $C$ delle scelte possibili, quindi,
per ogni $c\in C$ scriviamo $c$ nella scelta $S[i]$ e continuiamo la ricerca per
$i+1$.

\subsection{Approccio generale}
Come precedentemente accennato, il \emph{backtracking} ci fornisce un modo
sistematico di analizzare uno spazio delle soluzioni. Questo si traduce nella
possibilità di definire, per ogni classe di problema, uno schema generale per
l'algoritmo risolutivo. Il problema dell'\emph{enumerazione dei sottoinsiemi}
appartiene alla classe dei \emph{problemi di enumerazione}, e l'algoritmo
risolutivo segue grosso modo il seguente schema:

\begin{minicode}{Schema generale per algoritmi di enumerazione}
\ind enumeration($\langle$\emph{dati problema}$\rangle$, \bc{ITEM}[] S,
    \bc{int} i, $\langle$\emph{dati parziali}$\rangle$)\\
    \com{Verifica se $S[1\dots i-1]$ contiene una soluzione ammissibile}
    \indf if (accept($\langle$\emph{dati problema}$\rangle$, S, i,
        $\langle$\emph{dati parziali}$\rangle$))\\
        \com{\q{Processa} la soluzione (e.g. stampa, conta, \dots)}
        processSolution($\langle$\emph{dati problema}$\rangle$, S, i,
        $\langle$\emph{dati parziali}$\rangle$)\\
    \indf else\\
        \com{Calcola l'insieme delle scelte in funzione di $S[1\dots i-1]$}
        \bc{SET} C = choices($\langle$\emph{dati problema}$\rangle$, S, i,
        $\langle$\emph{dati parziali}$\rangle$)\\
        \com{Itera sull'insieme delle scelte}
        \indff foreach (c $\in$ C) do\\
            S[i] = c\\
            \com{Chiamata ricorsiva}
            enumeration($\langle$\emph{dati problema}$\rangle$, S, i + 1,
            $\langle$\emph{dati parziali}$\rangle$)
\end{minicode}

\paragraph{Albero delle decisioni}
L'uso della ricorsione per esaminare lo spazio delle soluzioni, ci consente di
rappresentare quest'ultimo come un \emph{albero}, in cui la \emph{radice}
rappresenta la soluzione parziale vuota, ciascun \emph{nodo} interno è associato
a una soluzione parziale e, infine, le \emph{foglie} sono le soluzioni ammissibili.

\begin{figure}[h!]
\centering
\scalebox{0.9}{\begin{graph}
    \node[main] (0) [label=above:{$S[0]$}] {};
    \node[main] (1) [below left of=0, xshift=-40, label=above:{$S[1]$}] {};
    \node[main] (2) [below right of=0, xshift=40] {};
    \node[main] (3) [below left of=1, xshift=0, label=above:{$S[2]$}] {};
    \node[main] (4) [below right of=1, xshift=0] {};
    \node[main] (5) [below left of=2, xshift=0] {};
    \node[main] (6) [below right of=2, xshift=0] {};
    \node[main] (9) [below left of=3, xshift=20,
        label={[above, xshift=-1.5mm]:{$S[3]$}}] {};
    \node[main] (10) [below right of=3, xshift=-20,
        label={[above, xshift=1.5mm]:{$S[3']$}}] {};
    \node[main] (11) [below left of=4, xshift=20] {};
    \node[main] (12) [below right of=4, xshift=-20] {};
    \node[main] (13) [below left of=5, xshift=20] {};
    \node[main] (14) [below right of=5, xshift=-20] {};
    \node[main] (15) [below left of=6, xshift=20] {};
    \node[main] (16) [below right of=6, xshift=-20] {};
    
    \path[-]    (0) edge (1)
                (0) edge (2)
                (1) edge (3)
                (1) edge (4)
                (2) edge (5)
                (2) edge (6)
                (3) edge (9)
                (3) edge (10)
                (4) edge (11)
                (4) edge (12)
                (5) edge (13)
                (5) edge (14)
                (6) edge (15)
                (6) edge (16);
\end{graph}}
\caption{Esempio di \emph{albero delle decisioni}}
\end{figure}

\paragraph{Implementazione ricorsiva}
\begin{minicode}{Implementazione ricorsiva della soluzione}
\ind subset(\bc{int} n)\\
    \bc{int}[] S = new \bc{int}[1\dots n]\hfill\com{Vettore delle scelte}
    subsetRec(n, S, 1)\\

\ind subsetRec(\bc{int} n, \bc{int}[] S, \bc{int} i)\\
    \com{Una soluzione è ammissibile dopo $n$ scelte}
    \indf if (i > n) then\\
        print S\\
    \indf else\\
        \bc{SET} C = \{0, 1\}\hfill\com{Ogni valore può essere preso o non preso}
        \indff foreach (c $\in$ C) do\\
            S[i] = c\\
            subsetRec(n, S, i + 1)
\end{minicode}

\paragraph{Complessità}
Come richiesto dal problema, l'algoritmo esplora tutto lo spazio delle soluzioni
e quindi la \emph{complessità} è $T(n)=\Theta(n2^n)$.

\paragraph{Implementazione iterativa}
Per questo particolare problema, a parità di \emph{complessità}, possiamo
scrivere anche una versione iterativa della soluzione.
\begin{minicode}{Implementazioe iterativa della soluzione}
\ind subset(\bc{int} n)\\
    \indf for (i = 0 to 2$^n$ - 1) do\\
        print "\{"\hfill\com{Usiamo le parentesi graffe per distinguere i sottoinsiemi}
        \indff for (j = 0 to n - 1) do\\
            \indfff if (i \&\& 2$^j$ $\neq$ 0) do\hfill\com{\&\&: operatore di bitwise and}
                print j, " "\\
        \indff println $\}$
\end{minicode}

\section{Problema dell'enumerazione delle permutazioni}
\begin{problem}[Problema dell'enumerazione delle permutazioni]
    Dato un insieme $A$, stamparne tutte le permutazioni.
\end{problem}

\noindent
Questo è un altro \emph{problema di enumerazione} quindi segue lo stesso schema
appena visto.

\begin{minicode}{Prima implementazione della soluzione}
\ind permutations(\bc{SET} A)\\
    \bc{int} n = size(A)\\
    \bc{int}[] S = new \bc{int}[1\dots n]\\
    permRec(A, S, 1)
\end{minicode}
\newpage
\begin{codecont}
permRec(\bc{SET} A, \bc{ITEM}[] S, \bc{int} i)\\
    \ind if (A.isEmpty()) then\hfill\com{Se $A$ è vuoto, $S$ è ammissibile}
        \indf print S\\
    \ind else\\
        \indf\bc{SET} C = copy(A)\hfill\com{Nel ciclo usiamo una copia di $A$}
        \indf foreach (c $\in$ C) do\\
            \indff S[i] = c\\
            \indff A.remove(c)\\
            \indff permRec(n, S, i + 1)\\
            \indff A.insert(C)
\end{codecont}

\begin{eg}[Esempio d'esecuzione]
Se $A=\{1,2,3\}$, l'albero delle soluzioni associato è il seguente:

\begin{figure}[h!]
\centering
\begin{graph}
    \node[main] (0) {};
    \node[main] (2) [below of=0, label=above left:{$B$}] {};
    \node[main] (1) [left of=2, xshift=-30mm, label=above:{$A$}] {};
    \node[main] (3) [right of=2, xshift=30mm, label=above:{$C$}] {};
    \node[main] (4) [below left of=1, label=above:{$B$}] {};
    \node[main] (5) [below right of=1, label=above:{$C$}] {};
    \node[main] (6) [below left of=2, label=above:{$A$}] {};
    \node[main] (7) [below right of=2, label=above:{$C$}] {};
    \node[main] (8) [below left of=3, label=above:{$A$}] {};
    \node[main] (9) [below right of=3, label=above:{$B$}] {};
    \node[main] (10) [below of=4, label=above left:{$C$},
        label=below:{$\{A,B,C\}$}] {};
    \node[main] (11) [below of=5, label=above left:{$B$},
        label=below:{$\{A,C,B\}$}] {};
    \node[main] (12) [below of=6, label=above left:{$C$},
        label=below:{$\{B,A,C\}$}] {};
    \node[main] (13) [below of=7, label=above left:{$A$},
        label=below:{$\{B,C,A\}$}] {};
    \node[main] (14) [below of=8, label=above left:{$B$},
        label=below:{$\{C,A,B\}$}] {};
    \node[main] (15) [below of=9, label=above left:{$A$},
        label=below:{$\{C,B,A\}$}] {};
    
    \path[-]    (0) edge (1)
                (0) edge (2)
                (0) edge (3)
                (1) edge (4)
                (1) edge (5)
                (2) edge (6)
                (2) edge (7)
                (3) edge (8)
                (3) edge (9)
                (4) edge (10)
                (5) edge (11)
                (6) edge (12)
                (7) edge (13)
                (8) edge (14)
                (9) edge (15);
\end{graph}
\end{figure}
\end{eg}

\paragraph{Complessità}
Il costo delle copie dell'\emph{insieme} lungo i \emph{cammini radice-foglia}
è $\sum_{i=1}^nO(i)=O(n^2)$ e poiché il numero di \emph{foglie} è $n!$, la
\emph{complessità} totale è $T(n)=O(n^2n!)$.

\paragraph{Versione migliorata dell'algoritmo}
\begin{minicode}{Seconda implementazione della soluzione}
\ind permutations(\bc{ITEM}[] S, \bc{int} n)\\
    permRec(S, n)\\

\ind permRec(\bc{ITEM}[] S, \bc{int} i)\\
    \indf if (i == 1) then\hfill\com{C'è solo una permutazione}
        println S\\
    \indf else\\
        \indff for (j = 1 to i) do\\
            swap(S, i, j)\\
            permRec(S, i - 1)\\
            swap(S, i, j)\\
\end{minicode}

\noindent
In questa soluzione, ad ogni iterazione del ciclo \texttt{for} vengono scambiati
di posizione gli elementi $S[i]$ e $S[j]$, quindi viene continuata ricorsivamente
l'esecuzione sul sottoinsieme $S[1\dots i - 1]$. Infine, vengono di nuovo
scambiati gli elementi $S[i]$ e $S[j]$.

\paragraph{Complessità}
Per ogni permutazione vengono eseguite $2n$ \texttt{swap} e poiché le permutazioni
sono $n!$, la \emph{complessità} di questa versione è $T(n)=O(n!n)$.

\section[Problema dell'enumerazione dei k-sottoinsiemi]{Problema dell'enumerazione dei \bm{$k$}-sottoinsiemi}
\begin{problem}[Problema dell'enumerazione dei \bm{$k$}-sottoinsiemi]
    Dato  un insieme $\{1,\dots,n\}$, elencarne tutti i sottoinsiemi di
    dimensione $k$.
\end{problem}

\begin{minicode}{Prima implementazione della soluzione}
\ind kSubset(\bc{int} n, \bc{int} k)\\
    \bc{int}[] S = new \bc{int}[1\dots n]\\
    kssRec(n, k, S, 1)\\

\noindent\com{Conta il numero di elementi del sottoinsieme}
\rmindent\ind\bc{int} count(\bc{int}[] S, \bc{int} n)\\
    \bc{int} count = 0\\
    \indf for (i = 1 to n) do\\
        count = count + S[i]\\
    \indf return count\\

\ind kssRec(\bc{int} n, \bc{int} k, \bc{int}[] S, \bc{int} i)\\
    \bc{int} size = count(S, n)\\
    \indf if (i > n and size == k) then\hfill\com{La soluzione trovata è ammissibile}
        processSolution(S, n)\\
    \indf else if (i > n and size $\neq$ k) then\hfill\com{La soluzione trovata non è ammissibile}
        return\\
    \indf else\\
        \indff foreach (c $\in$ \{0, 1\}) do\\
            S[i] = c\\
            kssRec(n, k S, i + 1)
\end{minicode}

\noindent
La \emph{complessità} è $O(2^nn)$ perché ogni elemento può essere incluso oppure
no in un sottoinsieme e la \texttt{count} costa sempre $O(n)$. Ciò che è
diverso dai problemi precedenti è che potremmo arrivare ad una soluzione non
ammissibile e, di conseguenza, possiamo modificare lo \nameref{code:122}
introducendo un controllo sull'inammissibilità di una soluzione.

\begin{code}{Schema generale completo per algoritmi di enumerazione}
\noindent\noindent enumeration($\langle$\emph{dati problema}$\rangle$, \bc{ITEM}[] S,
    \bc{int} i, $\langle$\emph{dati parziali}$\rangle$)\\
    \indent\com{Verifica se $S[1\dots i-1]$ contiene una soluzione ammissibile}
    \indent if (accept($\langle$\emph{dati problema}$\rangle$, S, i,
        $\langle$\emph{dati parziali}$\rangle$))\\
        \indf\com{\q{Processa} la soluzione (e.g. stampa, conta, \dots)}
        \indf processSolution($\langle$\emph{dati problema}$\rangle$, S, i,
        $\langle$\emph{dati parziali}$\rangle$)\\
    \ind\com{Verifica se $S[1\dots i-1]$ contiene una soluzione non ammissibile}
    \ind else if (reject($\langle$\emph{dati problema}$\rangle$, S, i,
    $\langle$\emph{dati parziali}$\rangle$)) then\\
        \indf return\\
    \ind else\\
        \indf\com{Calcola l'insieme delle scelte in funzione di $S[1\dots i-1]$}
        \indf\bc{SET} C = choices($\langle$\emph{dati problema}$\rangle$, S, i,
        $\langle$\emph{dati parziali}$\rangle$)\\
        \indf\com{Itera sull'insieme delle scelte}
        \indf foreach (c $\in$ C) do\\
            \indff S[i] = c\\
            \indff\com{Chiamata ricorsiva}
            \indff enumeration($\langle$\emph{dati problema}$\rangle$, S, i + 1,
            $\langle$\emph{dati parziali}$\rangle$)
\end{code}

\noindent
Ritornando all'implementazione della soluzione, possiamo eliminare l'invocazione
alla \texttt{count} e ridurre a $O(2^n)$ la \emph{complessità}.

\begin{minicode}{Seconda implementazione della soluzione}
\ind kssRec(\bc{int} n, \bc{int} missing, \bc{int}[] S, \bc{int} i)\\
    \indf if (i > n and missing == 0) then\hfill\com{Se $missing=0$ la soluzione è ammissibile}
        processSolution(S, n)\\
    \indf else if (i > n or missing < 0) then\\
        return\\
    \indf else\\
        \indff foreach (c $\in$ \{0, 1\}) do\\
            S[i] = c\\
            kssRec(n, missing - c, S, i + 1)
\end{minicode}
\begin{note}
    L'implementazione della funzione wrapper \texttt{kSubset} rimane invariata.
\end{note}

\noindent
Il parametro \texttt{missing} tiene traccia del numero di elementi che devono
ancora essere inseriti nel sottoinsieme per renderlo una soluzione.

\paragraph{Albero delle decisioni}
Eseguendo l'algoritmo con input $n=3$ e $k=1$, l'\emph{albero delle decisioni}
risultante è il seguente:

\begin{figure}[h!]
\centering
\begin{graph}
    \tikzset{emph/.style={main, color=red, line width=1.3pt}}

    \node[main] (0) {};
    \node[main] (1) [below left of=0, xshift=-40] {};
    \node[emph] (2) [below right of=0, xshift=40] {};
    \node[main] (3) [below left of=1, xshift=0] {};
    \node[emph] (4) [below right of=1, xshift=0] {};
    \node[main] (5) [below left of=2, xshift=0] {};
    \node[main] (6) [below right of=2, xshift=0] {};
    \node[main] (9) [below left of=3, xshift=20, yshift=-5mm] {};
    \node[emph] (10) [below right of=3, xshift=-20, yshift=-5mm] {};
    \node[main] (11) [below left of=4, xshift=20, yshift=-5mm] {};
    \node[main] (12) [below right of=4, xshift=-20, yshift=-5mm] {};
    \node[main] (13) [below left of=5, xshift=20, yshift=-5mm] {};
    \node[main] (14) [below right of=5, xshift=-20, yshift=-5mm] {};
    \node[main] (15) [below left of=6, xshift=20, yshift=-5mm] {};
    \node[main] (16) [below right of=6, xshift=-20, yshift=-5mm] {};

    \path[-]    (0) edge node[above left] {$0$} (1)
                (0) edge node[above right] {$1$} (2)
                (1) edge node[above left] {$0$} (3)
                (1) edge node[above right] {$1$} (4)
                (2) edge node[above left] {$0$} (5)
                (2) edge node[above right] {$1$} (6)
                (3) edge node[left] {$0$} (9)
                (3) edge node[right] {$1$} (10)
                (4) edge node[left] {$0$} (11)
                (4) edge node[right] {$1$} (12)
                (5) edge node[left] {$0$} (13)
                (5) edge node[right] {$1$} (14)
                (6) edge node[left] {$0$} (15)
                (6) edge node[right] {$1$} (16);
\end{graph}
\caption{\emph{Albero delle decisioni}}
\end{figure}

\noindent
I \emph{nodi} rossi rappresentano il punto in cui il valore di \texttt{missing}
arriva a $0$ per la prima volta nel \emph{cammino radice-foglia}. Continuare la
ricerca dopo quei \emph{nodi} non ha senso, perché ogni nuovo elemento aggiunto
al sottoinsieme porterebbe ad una soluzione inammissibile. Questo ci suggerisce
che l'implementazione potrebbe essere migliorata ulteriormente \q{cancellando}
quei \emph{sottoalberi} dall'insieme delle soluzioni. Questa operazione prende
il nome di \emph{pruning}.

\begin{minicode}{Terza implementazione della soluzione con pruning totale}
\ind kssRec(\bc{int} n, \bc{int} missing, \bc{int}[] S, \bc{int} i)\\
    \indf if (missing == 0) then\hfill\com{Se $missing=0$ la soluzione è ammissibile}
        processSolution(S, n)\\
    \indf else if (i > n or missing < 0 or n - (n - 1) < missing) then\\
        return\\
    \indf else\\
        \indff foreach (c $\in$ \{0, 1\}) do\\
            S[i] = c\\
            kssRec(n, missing - c, S, i + 1)
\end{minicode}

\noindent
Oltre a rimuovere i \emph{sottoalberi} in eccedenza, possiamo togliere anche
quelli che di sicuro non porteranno ad alcuna soluzione ammissibile. Per esempio,
se $k=2$, nell'\emph{albero} precedente avremmo potuto eseguire il \emph{pruning}
del \emph{sottoalbero} radicato nel primo \emph{nodo} da sinistra del secondo
\emph{livello}. Questo perché, anche scegliendo di inserire nel sottoinsieme
tutti i valori successivi, non si sarebbe potuto arrivare ad annullare il valore
di \texttt{missing}.

\begin{note}
    Quando valutiamo se eseguire o meno il \emph{pruning} di un certo
    \emph{sottoalbero}, prendiamo la decisione in base alla soluzione parziale
    associata al \emph{nodo radice} del \emph{sottoalbero}.
\end{note}

\begin{minicode}{Implementazione ripulita della soluzione}
\ind kssRec(\bc{int} n, \bc{int} missing, \bc{int}[] S, \bc{int} i)\\
    \indf if (missing == 0) then\hfill\com{Se $missing=0$ la soluzione è ammissibile}
        processSolution(S, n)\\
    \indf else if (i $\leq$ n and 0 < missing $\leq$ n - (n - 1)) then\\
        \indff foreach (c $\in$ \{0, 1\}) do\\
            S[i] = c\\
            kssRec(n, missing - c, S, i + 1)
\end{minicode}

\paragraph{Complessità}
Sebbene l'introduzione del \emph{pruning} non riduca la \emph{complessità},
permette comunque di migliorare l'efficienza per valori di $k$ vicini agli
estremi $1$, $n$. Per quei valori di $k$ infatti, la porzione di \emph{albero}
che viene eliminata è sostanziale. Viceversa il guadagno è soltanto marginale
per valori di $k$ vicini a $\frac{n}{2}$.

\section{Problema del subset sum}
\begin{problem}[Problema del subset sum]
    Dati un vettore $A$ contenente $n$ interi positivi ed un intero positivo
    $k$, dire se esiste un sottoinsieme di indici $S\subseteq\{1\dots n\}$ tale
    per cui $\sum_{i\in S}A[i]=k$.
\end{problem}

\noindent
Ad esempio, se $A=[1,4,3,12,7,2,21,55]$ e $k=23$, due possibili soluzioni sono
$S_1=\{2,4,5\}$ perché $4+12+7=23$, e $S_2=\{6,7\}$ perché $2+21=23$.

\begin{note}
    Questo problema può essere risolto in tempo $O(kn)$ sfruttando la
    \emph{programmazione dinamica}, tuttavia qui ne vediamo una soluzione
    tramite \emph{backtracking} di tempo $O(2^n)$ in cui però interrompiamo
    l'esecuzione alla prima soluzione trovata.
\end{note}

\subsection{Approccio generale}
Lo schema generale è simile a quello usato per l'\emph{enumerazione}, ma alla
prima soluzione ammissibile trovata, l'algoritmo termina restituendo
\texttt{true}.

\begin{minicode}{Schema generale completo con interruzione}
\ind\bc{boolean} enumeration($\langle$\emph{dati problema}$\rangle$, \bc{ITEM}[] S,
    \bc{int} i, $\langle$\emph{dati parziali}$\rangle$)\\
    \com{Verifica se $S[1\dots i-1]$ contiene una soluzione ammissibile}
    \indf if (accept($\langle$\emph{dati problema}$\rangle$, S, i,
        $\langle$\emph{dati parziali}$\rangle$))\\
        \com{\q{Processa} la soluzione (e.g. stampa, conta, \dots)}
        processSolution($\langle$\emph{dati problema}$\rangle$, S, i,
        $\langle$\emph{dati parziali}$\rangle$)\\
        return true\\
    \indf\com{Verifica se $S[1\dots i-1]$ contiene una soluzione non ammissibile}
    \indf else if (reject($\langle$\emph{dati problema}$\rangle$, S, i,
    $\langle$\emph{dati parziali}$\rangle$))\\
        return false\\
    \indf else\\
        \com{Calcola l'insieme delle scelte in funzione di $S[1\dots i-1]$}
        \bc{SET} C = choices($\langle$\emph{dati problema}$\rangle$, S, i,
        $\langle$\emph{dati parziali}$\rangle$)\\
        \com{Itera sull'insieme delle scelte}
        \indff foreach (c $\in$ C) do\\
            S[i] = c\\
            \com{Chiamata ricorsiva}
            \indfff if (enumeration($\langle$\emph{dati problema}$\rangle$, S, i + 1,
            $\langle$\emph{dati parziali}$\rangle$)) then\\
                return true\\
        \indff return false
\end{minicode}

\paragraph{Implementazione}
\begin{minicode}{Implementazione della soluzione}
\ind\bc{boolean} subsetSum(\bc{int}[] A, \bc{int} n, \bc{int} k)\\
    \bc{int}[] S = new \bc{int}[1\dots n]\\
    return ssRec(A, n, k, S, 1)\\

\ind\bc{boolean} ssRec(\bc{int}[] A, \bc{int} n, \bc{int} missing, \bc{int}[] S, \bc{int} i)\\
    \indf if (missing == 0) then\\
        return true\\
    \indf else if (i > n or missing < 0) then\\
        return false\hfill\com{I valori sono terminati o la somma è eccessiva}
    \indf else\\
        \indff foreach (c $\in$ \{0, 1\}) do\\
            S[i] = c\\
            \indfff if (ssRec(A, n, missing - A[i] $\cdot$ c, S, i + 1)) then\\
                return true\\
        \indff return false
\end{minicode}

\section{Problema dell'inviluppo convesso}
\begin{definition}[Poligono convesso]
    Un poligono nel piano è detto essere convesso se ogni segmento di retta
    che congiunge due punti del poligono rimane sempre dentro il poligono
    stesso.
\end{definition}
\begin{problem}[Problema dell'inviluppo convesso]
    Dati $n$ punti $p_1,\dots,p_n$ nel piano, con $n\geq 3$, determinare
    l'inviluppo convesso, ovvero il poligono convesso di superficie minima
    contenente tutti gli $n$ punti.
\end{problem}

\noindent
Intuitivamente, poiché stiamo cercando il \emph{poligono convesso} di superficie
minima, i vertici di quel poligono dovranno coincidere con alcuni o tutti i
punti del piano. In particolare, presi due punti, il segmento che li congiunge è
un lato del poligono se tutti i rimanenti $n-2$ punti stanno dalla \q{stessa
parte}. Per verificare la condizione usiamo la seguente funzione:

\begin{minicode}{Implementazione sameSide}
\ind\bc{boolean} sameSide(\bc{POINT} p$_1$, \bc{POINT} p$_2$, \bc{POINT} p, \bc{POINT} q)\\
    \bc{float} dx = p$_2$.x - p$_1$.x\\
    \bc{float} dy = p$_2$.y - p$_1$.y\\
    \bc{float} dx$_1$ = p.x - p$_1$.x\\
    \bc{float} dy$_1$ = p.y - p$_1$.y\\
    \bc{float} dx$_2$ = q.x - p$_2$.x\\
    \bc{float} dy$_2$ = q.y - p$_2$.y\\
    \com{Restituisce \bc{true} se $p$ e $q$ stanno dalla stessa parte della retta
    tra $p_1$\,e\,$p_2$}
    return ((dx $\cdot$ dy$_1$ - dy $\cdot$ dx$_1$) $\cdot$
    (dx $\cdot$ dy$_2$ - dy $\cdot$ dx$_2$) $\geq$ 0)
\end{minicode}

\noindent
Questa soluzione è intuitiva e facile da implementare, ma la \emph{complessità}
è $O(n^3)$, perché le possibili coppie di punti sono $O(n^2)$ e per ognuna bisogna
verificare che i rimanenti $n-2$ punti stiano dalla \q{stessa parte}.

\subsection{Algoritmo di Jarvis}
Anche detto \emph{Gift Packing}, l'\emph{Algoritmo di Jarvis} parte dal punto
più a sinistra dell'insieme, ovvero dal punto $p_0$ con ordinata minore. Quindi,
viene calcolato l'angolo d'inclinazione delle rette passanti per $p_0$ e
ogni altro punto rispetto alla verticale. 

\smallskip\noindent
\begin{minipage}{0.48\textwidth}
Il punto $p_1$ associato alla retta con angolo minore viene selezionato come
vertice del poligono. Successivamente, si considera la retta $r$ passante per
$p_{i-1}$ e $p_{i-2}$ e si confrontano gli angoli che le rette passanti per
$p_{i-1}$ e tutti i punti rimanenti formano con $r$. Ad essere selezionato come
vertice è il punto $p_i$ associato all'angolo minore. Il tutto viene ripetuto
fino a quando ad essere selezionato non è di nuovo $p_0$.
\end{minipage}\hfill
\begin{minipage}{0.48\textwidth}
\centering
\begin{graph}
    \tikzset{
        point/.style={circle, fill, minimum width=4pt, inner sep=0},
        empty/.style={inner sep=0},
        pos/.style args={#1:#2 from #3}{
          at=(#3.#1), anchor=#1+180, shift=(#1:#2)
        }
    }
    \node[point] (0) [label=right:$p_0$] {};
    \node[point] (1) [pos=65:18mm from 0, label=right:$p_1$] {};
    \node[point] (2) [pos=45:15mm from 1, label=below:$p_2$] {};
    \node[point] (3) [pos=-20:30mm from 2, label=right:$p_3$] {};
    \node[point] (4) [pos=20:10mm from 1] {};
    \node[point] (5) [pos=-18:10mm from 1] {};
    \node[point] (6) [pos=20:38mm from 0] {};
    \node[point] (7) [pos=12:44mm from 0] {};
    \node[point] (8) [pos=6:55mm from 0] {};
    \node[point] (9) [pos=-3:25mm from 0] {};
    \node[point] (10) [pos=-5:47mm from 0] {};
    \node[point] (11) [pos=-25:25mm from 0] {};

    \node[empty] (e1) [pos=90:35mm from 0] {};
    \node[empty] (e2) [pos=-90:12mm from 0] {};

    \draw[->] (0)+(0, -0.5cm) arc (270:65:0.5cm);
    \node[empty] (e3) [pos=65:13.5mm from 0] {};
    \draw[->] (e3) arc (245:45:0.5cm);
    \node[empty] (e4) [pos=45:10.5mm from 1] {};
    \draw[->] (e4) arc (225:-20:0.5cm);

    \path[-]    (0) edge (1)
                (1) edge (2)
                (2) edge (3);
            
    \draw[-, dashed] (e1) -- (e2);
\end{graph}
\end{minipage}

\paragraph{Complessità}
La selezione del punto $p_0$ costa $O(n)$ e la ricerca dello spigolo successivo
costa di nuovo $O(n)$ perché viene considerato ogni punto rimanente. Alla fine,
se i vertici del poligono sono $h$, l'algoritmo ha una \emph{complessità} totale
di $O(nh)$.

\subsection{Algoritmo di Graham}
L'\emph{Algoritmo di Graham} inserisce nell'\emph{inviluppo convesso} il punto di
ordinata minima e ordina i rimanenti in base all'angolo formato dalla retta
passante per ciascun punto e il punto di ordinata minima e la retta orizzontale.
Il primo punto dell'ordinamento viene aggiunto all'\emph{inviluppo}.

A questo punto, siano $p_j$ e $p_{j-1}$ l'ultimo e il penultimo punto
dell'\emph{inviluppo}. Per tutti i punti $p_i$, con $i\in[3,n]$, se $p_1$ e
$p_i$ non si trovano dalla stessa parte rispetto alla retta passante per $p_j$
e $p_{j-1}$, il punto $p_j$ viene rimosso dall'\emph{inviluppo}. Se invece
la scansione dei punti termina senza che $p_j$ debba essere rimosso, $p_i$
viene aggiunto all'\emph{inviluppo}.

\begin{minicode}{Implementazione Algoritmo di Graham}
\ind\bc{STACK} graham(\bc{POINT}[] p, \bc{int} n)\\
    \bc{int} min = 1\\
    \indf for (i = 2 to n) do\\
        \indff if (p[i].y < p[min].y) then\\
            min = i\\
    \indf p[1] $\leftrightarrow$ p[min]\hfill\com{Sposta all'inizio il punto
    di ordinata minima}
    \indf\{ Ordina $p[2,\dots, n]$ in base all'angolo formato con all'asse
    orizzontale \}\\
    \indf\{ Elimina i punti allineati mantenendo solo quello più distante
    da $p_1$ \}\\
    \indf\{ Aggiorna $n$ riducendolo del numero di punti eliminati \}\\
    \indf\bc{STACK} S = Stack()\\
    \indf S.push(p[1])\\
    \indf S.push(p[2])\\
    \indf for (i = 3 to n) do\\
        \indff while (not sameSide(S.top(), S.top2(), p[1], p[i])) do\\
            S.pop()\\
        \indff S.push(p[i])\\
    \indf return S
\end{minicode}

\begin{eg}[Esempio d'esecuzione]
    Consideriamo il seguente insieme di punti:
    \begin{figure}[h!]
    \centering
    \subfloat[Insieme iniziale grezzo]{
    \begin{graph}
        \tikzset{
            point/.style={circle, fill, minimum width=4pt, inner sep=0},
            empty/.style={inner sep=0},
            pos/.style args={#1:#2 from #3}{
              at=(#3.#1), anchor=#1+180, shift=(#1:#2)
            }
        }
        \node[point] (0) {};
        \node[point] (1) [pos=65:18mm from 0] {};
        \node[point] (2) [pos=45:15mm from 1] {};
        \node[point] (3) [pos=-20:30mm from 2] {};
        \node[point] (4) [pos=20:10mm from 1] {};
        \node[point] (5) [pos=-18:10mm from 1] {};
        \node[point] (6) [pos=20:38mm from 0] {};
        \node[point] (7) [pos=12:44mm from 0] {};
        \node[point] (8) [pos=6:55mm from 0] {};
        \node[point] (9) [pos=-3:25mm from 0] {};
        \node[point] (10) [pos=-5:47mm from 0] {};
        \node[point] (11) [pos=-25:25mm from 0] {};

        \node[empty] (e) at (11) [yshift=-6.25mm] {};
    \end{graph}}
    \hspace{1.5cm}
    \subfloat[Insieme ordinato]{
    \begin{graph}
        \tikzset{
            point/.style={circle, fill, minimum width=4pt, inner sep=0},
            empty/.style={inner sep=0},
            pos/.style args={#1:#2 from #3}{
              at=(#3.#1), anchor=#1+180, shift=(#1:#2)
            }
        }
        \node[point] (0) [label=left:{$12$}] {};
        \node[point] (1) [pos=65:18mm from 0, label=left:{$11$}] {};
        \node[point] (2) [pos=45:15mm from 1, label=above:{$8$}] {};
        \node[point] (3) [pos=-20:30mm from 2, label=above left:{$5$}] {};
        \node[point] (4) [pos=20:10mm from 1, label=above left:{$9$}] {};
        \node[point] (5) [pos=-18:10mm from 1, label=above left:{$10$}] {};
        \node[point] (6) [pos=20:38mm from 0, label=above:{$6$}] {};
        \node[point] (7) [pos=12:44mm from 0, label=right:{$4$}] {};
        \node[point] (8) [pos=6:55mm from 0, label=above:{$3$}] {};
        \node[point] (9) [pos=-3:25mm from 0, label=above:{$7$}] {};
        \node[point] (10) [pos=-5:47mm from 0, label=right:{$2$}] {};
        \node[point] (11) [pos=-25:25mm from 0, label=below:{$1$}] {};

        \path[-, dashed]
                    (11) edge (0)
                    (11) edge (1)
                    (11) edge (2)
                    (11) edge (3)
                    (11) edge (4)
                    (11) edge (5)
                    (11) edge (6)
                    (11) edge (7)
                    (11) edge (8)
                    (11) edge (9)
                    (11) edge (10);
            
        \node[empty] (e1) at (11) [xshift=-35mm] {};
        \node[empty] (e2) at (11) [xshift=35mm] {};
        
        \draw[-, dashed] (e1) edge (e2);
        \draw[-] (11)+(14.5mm,0) arc[
            start angle=0,
            end angle=16,
            radius=15mm
        ];
        \node[empty] (alpha) [pos=8:15mm from 11] {$\alpha$};
    \end{graph}}
    \end{figure}

    \noindent
    L'Algoritmo di Graham inserisce nell'inviluppo convesso i punti $1$ e
    $2$. Poi, a partire dal terzo, controlla che ogni punto stia dalla stessa
    parte del primo rispetto all'ultimo lato dell'inviluppo.

    \newpage
    \begin{figure}[ht]
    \ContinuedFloat
    \centering
    \subfloat[$i=3$]{
    \begin{graph}
        \tikzset{
            point/.style={circle, fill, minimum width=4pt, inner sep=0},
            empty/.style={inner sep=0},
            pos/.style args={#1:#2 from #3}{
                at=(#3.#1), anchor=#1+180, shift=(#1:#2)
            }
        }
        \node[point] (0) [label=left:{$12$}] {};
        \node[point] (1) [pos=65:18mm from 0, label=left:{$11$}] {};
        \node[point] (2) [pos=45:15mm from 1, label=above:{$8$}] {};
        \node[point] (3) [pos=-20:30mm from 2, label=above left:{$5$}] {};
        \node[point] (4) [pos=20:10mm from 1, label=above left:{$9$}] {};
        \node[point] (5) [pos=-18:10mm from 1, label=above left:{$10$}] {};
        \node[point] (6) [pos=20:38mm from 0, label=above:{$6$}] {};
        \node[point] (7) [pos=12:44mm from 0, label=right:{$4$}] {};
        \node[point] (8) [pos=6:55mm from 0, label=above:{$3$}] {};
        \node[point] (9) [pos=-3:25mm from 0, label=above:{$7$}] {};
        \node[point] (10) [pos=-5:47mm from 0, label=right:{$2$}] {};
        \node[point] (11) [pos=-25:25mm from 0, label=below:{$1$}] {};

        \path[-]    (11) edge (10)
                    (10) edge (8);
    \end{graph}}
    \hspace{1.5cm}
    \subfloat[$i=4$]{
    \begin{graph}
        \tikzset{
            point/.style={circle, fill, minimum width=4pt, inner sep=0},
            empty/.style={inner sep=0},
            pos/.style args={#1:#2 from #3}{
                at=(#3.#1), anchor=#1+180, shift=(#1:#2)
            }
        }
        \node[point] (0) [label=left:{$12$}] {};
        \node[point] (1) [pos=65:18mm from 0, label=left:{$11$}] {};
        \node[point] (2) [pos=45:15mm from 1, label=above:{$8$}] {};
        \node[point] (3) [pos=-20:30mm from 2, label=above left:{$5$}] {};
        \node[point] (4) [pos=20:10mm from 1, label=above left:{$9$}] {};
        \node[point] (5) [pos=-18:10mm from 1, label=above left:{$10$}] {};
        \node[point] (6) [pos=20:38mm from 0, label=above:{$6$}] {};
        \node[point] (7) [pos=12:44mm from 0, label=right:{$4$}] {};
        \node[point] (8) [pos=6:55mm from 0, label=above:{$3$}] {};
        \node[point] (9) [pos=-3:25mm from 0, label=above:{$7$}] {};
        \node[point] (10) [pos=-5:47mm from 0, label=right:{$2$}] {};
        \node[point] (11) [pos=-25:25mm from 0, label=below:{$1$}] {};

        \path[-]    (11) edge (10)
                    (10) edge (8)
                    (8) edge (7);
    \end{graph}}\\
    \subfloat[$i=5$]{
    \begin{graph}
        \tikzset{
            point/.style={circle, fill, minimum width=4pt, inner sep=0},
            empty/.style={inner sep=0},
            pos/.style args={#1:#2 from #3}{
                at=(#3.#1), anchor=#1+180, shift=(#1:#2)
            }
        }
        \node[point] (0) [label=left:{$12$}] {};
        \node[point] (1) [pos=65:18mm from 0, label=left:{$11$}] {};
        \node[point] (2) [pos=45:15mm from 1, label=above:{$8$}] {};
        \node[point] (3) [pos=-20:30mm from 2, label=above left:{$5$}] {};
        \node[point] (4) [pos=20:10mm from 1, label=above left:{$9$}] {};
        \node[point] (5) [pos=-18:10mm from 1, label=above left:{$10$}] {};
        \node[point] (6) [pos=20:38mm from 0, label=above:{$6$}] {};
        \node[point] (7) [pos=12:44mm from 0, label=right:{$4$}] {};
        \node[point] (8) [pos=6:55mm from 0, label=above:{$3$}] {};
        \node[point] (9) [pos=-3:25mm from 0, label=above:{$7$}] {};
        \node[point] (10) [pos=-5:47mm from 0, label=right:{$2$}] {};
        \node[point] (11) [pos=-25:25mm from 0, label=below:{$1$}] {};

        \node[empty] (e) at (7) {\scalebox{1.5}{\bm{$\times$}}};

        \path[-]    (11) edge (10)
                    (10) edge (8)
                    (8) edge[dashed] (7)
                    (8) edge (3);
    \end{graph}}
    \hspace{1.5cm}
    \subfloat[$i=6$]{
    \begin{graph}
        \tikzset{
            point/.style={circle, fill, minimum width=4pt, inner sep=0},
            empty/.style={inner sep=0},
            pos/.style args={#1:#2 from #3}{
                at=(#3.#1), anchor=#1+180, shift=(#1:#2)
            }
        }
        \node[point] (0) [label=left:{$12$}] {};
        \node[point] (1) [pos=65:18mm from 0, label=left:{$11$}] {};
        \node[point] (2) [pos=45:15mm from 1, label=above:{$8$}] {};
        \node[point] (3) [pos=-20:30mm from 2, label=above left:{$5$}] {};
        \node[point] (4) [pos=20:10mm from 1, label=above left:{$9$}] {};
        \node[point] (5) [pos=-18:10mm from 1, label=above left:{$10$}] {};
        \node[point] (6) [pos=20:38mm from 0, label=above:{$6$}] {};
        \node[point] (7) [pos=12:44mm from 0, label=right:{$4$}] {};
        \node[point] (8) [pos=6:55mm from 0, label=above:{$3$}] {};
        \node[point] (9) [pos=-3:25mm from 0, label=above:{$7$}] {};
        \node[point] (10) [pos=-5:47mm from 0, label=right:{$2$}] {};
        \node[point] (11) [pos=-25:25mm from 0, label=below:{$1$}] {};

        \path[-]    (11) edge (10)
                    (10) edge (8)
                    (8) edge (3)
                    (3) edge (6);
    \end{graph}}\\
    \subfloat[$i=7$]{
    \begin{graph}
        \tikzset{
            point/.style={circle, fill, minimum width=4pt, inner sep=0},
            empty/.style={inner sep=0},
            pos/.style args={#1:#2 from #3}{
                at=(#3.#1), anchor=#1+180, shift=(#1:#2)
            }
        }
        \node[point] (0) [label=left:{$12$}] {};
        \node[point] (1) [pos=65:18mm from 0, label=left:{$11$}] {};
        \node[point] (2) [pos=45:15mm from 1, label=above:{$8$}] {};
        \node[point] (3) [pos=-20:30mm from 2, label=above left:{$5$}] {};
        \node[point] (4) [pos=20:10mm from 1, label=above left:{$9$}] {};
        \node[point] (5) [pos=-18:10mm from 1, label=above left:{$10$}] {};
        \node[point] (6) [pos=20:38mm from 0, label=above:{$6$}] {};
        \node[point] (7) [pos=12:44mm from 0, label=right:{$4$}] {};
        \node[point] (8) [pos=6:55mm from 0, label=above:{$3$}] {};
        \node[point] (9) [pos=-3:25mm from 0, label=above:{$7$}] {};
        \node[point] (10) [pos=-5:47mm from 0, label=right:{$2$}] {};
        \node[point] (11) [pos=-25:25mm from 0, label=below:{$1$}] {};

        \path[-]    (11) edge (10)
                    (10) edge (8)
                    (8) edge (3)
                    (3) edge (6)
                    (6) edge (9);
    \end{graph}}
    \hspace{1.5cm}
    \subfloat[$i=8$]{
    \begin{graph}
        \tikzset{
            point/.style={circle, fill, minimum width=4pt, inner sep=0},
            empty/.style={inner sep=0},
            pos/.style args={#1:#2 from #3}{
                at=(#3.#1), anchor=#1+180, shift=(#1:#2)
            }
        }
        \node[point] (0) [label=left:{$12$}] {};
        \node[point] (1) [pos=65:18mm from 0, label=left:{$11$}] {};
        \node[point] (2) [pos=45:15mm from 1, label=above:{$8$}] {};
        \node[point] (3) [pos=-20:30mm from 2, label=above left:{$5$}] {};
        \node[point] (4) [pos=20:10mm from 1, label=above left:{$9$}] {};
        \node[point] (5) [pos=-18:10mm from 1, label=above left:{$10$}] {};
        \node[point] (6) [pos=20:38mm from 0, label=above:{$6$}] {};
        \node[point] (7) [pos=12:44mm from 0, label=right:{$4$}] {};
        \node[point] (8) [pos=6:55mm from 0, label=above:{$3$}] {};
        \node[point] (9) [pos=-3:25mm from 0, label=above:{$7$}] {};
        \node[point] (10) [pos=-5:47mm from 0, label=right:{$2$}] {};
        \node[point] (11) [pos=-25:25mm from 0, label=below:{$1$}] {};

        \node[empty] (e1) at (6) {\scalebox{1.5}{\bm{$\times$}}};
        \node[empty] (e2) at (9) {\scalebox{1.5}{\bm{$\times$}}};

        \path[-]    (11) edge (10)
                    (10) edge (8)
                    (8) edge (3)
                    (3) edge[dashed] (6)
                    (6) edge[dashed] (9)
                    (3) edge (2);
    \end{graph}}
    \end{figure}

    \noindent
    L'algoritmo continua in questo modo provando tutti i punti.
\end{eg}

\paragraph{Complessità}
Il calcolo degli angoli tra le rette e la rimozione dei punti allineati
costa $O(n)$ e l'ordinamento costa $O(n\log n)$. Una volta ordinati, i punti
vengono aggiunti e rimossi dall'\emph{inviluppo} al massimo una volta
ciascuno, quindi la \emph{complessità} finale è $T(n)=O(n\log n)$.