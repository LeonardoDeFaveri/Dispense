\chapter{Ricerca locale}
\section{Introduzione}
La \emph{ricerca locale} è una tecnica risolutiva che, partendo da una soluzione
ammissibile qualunque, ricerca una soluzione migliore nelle \q{vicinanze} di
quella nota. Questo processo continua fino ad arrivare a una soluzione localmente
ottima. La particolarità di questa tecnica è che la soluzione trovata potrebbe
non essere la migliore possibile.

\begin{figure}[h!]
\centering
\begin{tikzpicture}
\begin{axis}[xmin = 0, xmax = 12, ymin = -1.5, ymax = 2.0,
        axis lines = left,
        xlabel = $x$,
        ylabel = {\emph{Costo}},
        ylabel style={rotate=-90},
        every axis x label/.style={
            at={(ticklabel* cs:1)},
            anchor=west,
        },
        every axis y label/.style={
            at={(ticklabel* cs:1)},
            anchor=south,
        },
        ticks=none
    ]
    \addplot[
        domain=0:12,
        samples = 200,
        smooth
    ] {exp(-x/10)*( cos(deg(x)) + sin(deg(x))/10 )};

    \node[label={below:{$A$}}, circle, fill, inner sep=1.51pt]
        at (axis cs:pi,{-exp(-pi/10)}) {};
    \node[label={below:{$B$}}, circle, fill, inner sep=1.5pt]
        at (axis cs:{3*pi},{-exp(-3*pi/10)}) {};
\end{axis}
\end{tikzpicture}
\caption{Possibili soluzioni locali}
\end{figure}

\noindent
Nel grafico di cui sopra, se la ricerca partisse da una soluzione \q{vicina} a
$B$, un algoritmo basato su \emph{ricerca locale} potrebbe convergere sul punto
$B$, sebbene esso non rappresenti una soluzione ottima globale.

\begin{minicode}{Implementazione generica di un algoritmo basato su ricerca locale}
\ind\bc{SOL} ricercaLocale()\\
    \bc{SOL} sol = \{ Una qualsiasi soluzione ammissibile \}\\
    \indf while ($\exists$S $\in$ I(Sol) migliore di sol) do\\
        sol = S\\
    \indf return sol
\end{minicode}

\section{Problema della ricerca del flusso massimo}
\begin{definition}[Rete di flusso]
    Una rete di flusso $G=(V,E,s,t,c)$ è data da un grafo orientato $G=(V,E)$,
    da due nodi $s,t\in V$, detti rispettivamente sorgente e pozzo, e da una
    funzione di capacità $c:V\times V\to\mathbb{R}^+$ tale per cui $c(x,y)=0$
    $\forall (x,y)\notin E$.
\end{definition}
\begin{note}
    Assumiamo che per ogni \emph{nodo} $x\in V$ esista un \emph{cammino} da $s$
    in $t$ passante per esso, e ignoriamo qualsiasi \emph{nodo} che non goda di
    questa proprietà.
\end{note}
\begin{definition}[Flusso]
    Un flusso in $G$ è una funzione $f:V\times V\to\mathbb{R}$ che soddisfa le
    seguenti proprietà:
    \begin{enumerate}
        \item Vincolo sulla capacità: $f(x,y)\leq c(x,y)$ $\forall x,y\in V$;
        \item Antisimmetria: $f(x,y)=-f(x,y)$ $\forall x,y\in V$;
        \item Conservazione del flusso: $sum_{y\in V}f(x,y)=0$ $\forall x\in V-\{s,t\}$;
    \end{enumerate}
\end{definition}

\noindent
La prima proprietà stabilisce che un \emph{flusso} non può mai eccedere la
capacità di un \emph{arco}. In particolare, se la capacità di un \emph{arco}
tra due \emph{nodi} $A$ e $B$ è 7, e $f(A,B)=5$, nei \emph{grafi} useremo la
seguente notazione:

\begin{figure}[h!]
    \centering
    \begin{graph}
        \node[main] (a) {$A$};
        \node[main] (b) [right of=a, xshift=15mm] {$B$};

        \path[->]   (a) edge node[above] {$5/7$} (b);
    \end{graph}
    \caption{\emph{Vincolo sulla capacità}}
\end{figure}

\noindent
La proprietà di \emph{antisimmetria} ci dice invece che il \emph{flusso}
che attraversa un \emph{arco} in direzione $(x,y)$ è l'opposto del \emph{flusso}
che attraverso lo stesso \emph{arco} in direzione $(y,x)$.

\begin{figure}[h!]
    \centering
    \begin{graph}
        \node[main] (a) {$A$};
        \node[main] (b) [right of=a, xshift=15mm] {$B$};

        \path[->]   (a) edge[bend left=25] node[above] {$5$} (b)
                    (b) edge[bend left=25, dashed] node[above] {$-5$} (a);
    \end{graph}
    \caption{\emph{Antisimmetria}}
\end{figure}

\noindent
Infine, la proprietà di \emph{conservazione del flusso} ci dice che per ogni
\emph{nodo} diverso dalla \emph{sorgente} e dal \emph{pozzo}, la somma dei
\emph{flussi} entranti ed uscenti è sempre pari a $0$.

\begin{figure}[h!]
    \centering
    \begin{graph}
        \node[main] (a) {$A$};
        \node[main] (c) [below right of=a, xshift=10mm] {$C$};
        \node[main] (b) [below left of=c, xshift=-10mm] {$B$};
        \node[main] (d) [above right of=c, xshift=10mm] {$D$};
        \node[main] (e) [below right of=c, xshift=10mm] {$E$};

        \path[->]   (a) edge[bend left=25] node[above] {$2$} (c)
                    (c) edge[bend left=25, dashed] node[above, xshift=1mm] {$-2$} (a)
                    (c) edge[bend left=25] node[above] {$1$} (d)
                    (d) edge[bend left=25, dashed] node[above, xshift=-1mm] {$-1 $} (c)
                    (c) edge[bend left=25] node[above] {$4$} (e)
                    (e) edge[bend left=25, dashed] node[above, xshift=1mm] {$-4$} (c)
                    (b) edge[bend left=25] node[above] {$3$} (c)
                    (c) edge[bend left=25, dashed] node[above, xshift=-1mm] {$-3$} (b);
    \end{graph}
    \caption{\emph{Conservazione del flusso}}
\end{figure}

\begin{definition}[Valore del flusso]
    Il valore di un flusso $f$ è definito come:
    \[|f|=\sum_{(s,x)\in E}f(s,x)\]
    ovvero come la quantità di flusso uscente da $s$.
\end{definition}

\noindent
Date tutte queste definizioni possiamo finalmente enunciare il problema in
questione.

\begin{problem}[Problema della ricerca del flusso massimo]
    Data una rete $G=(V,E,s,t,c)$, trovare un flusso che abbia valore massimo
    fra tutti i flussi associabili alla rete. In particolare, ricercare il flusso
    $f^*$ tale per cui $|f^*|=\max\{|f|\}$.
\end{problem}

\subsection{Metodo delle reti residue}
L'algoritmo che intendiamo realizzare è detto \emph{metodo delle reti residue}
perché è basato su un approccio additivo. In particolare, partendo da un
\emph{flusso} $f$ inizialmente nullo, si ripetono le seguenti operazioni:
\begin{enumerate}
    \item \q{Sottrazione} del \emph{flusso} $f$ alla rete attuale, ottenendo
    una \emph{rete residua};
    \item Ricerca di un secondo \emph{flusso} $g$ all'interno della \emph{rete
    residua};
    \item Somma di $g$ ad $f$;
\end{enumerate}
fino a quando $g$ non diventa un \emph{flusso nullo}.

\begin{note}
    Questo particolare algoritmo permette sempre di ottenere il \emph{flusso
    massimo}, quindi non soffre del problema al quale avevamo accennato
    inizialmente.
\end{note}

\noindent
Diamo ora delle altre definizioni:
\begin{definition}[Flusso nullo]
    È definito flusso nullo la funzione $f_0:V\times V\to\mathbb{R^+}$ tale che:
    \[f(x,y)=0\qquad\forall x,y\in V\]
\end{definition}

\begin{figure}[h!]
    \centering
    \begin{graph}
        \node[main] (a) [label=above left:{$s$}] {$A$};
        \node[main] (b) [above right of=a, xshift=10mm] {$B$};
        \node[main] (c) [below right of=a, xshift=10mm] {$C$};
        \node[main] (d) [below right of=b, xshift=10mm,
            label=above right:{$t$}] {$D$};

        \node[] (e) [left of=a, xshift=-10mm] {$|f|=0$};

        \path[->]   (a) edge node[above left] {$0/2$} (b)
                    (b) edge node[above right] {$0/3$} (d)
                    (a) edge node[below left] {$0/2$} (c)
                    (c) edge node[right] {$0/1$} (b)
                    (c) edge node[below right] {$0/1$} (d);
    \end{graph}
    \caption{\emph{Flusso nullo}}
\end{figure}

\newpage
\begin{definition}[Somma di flussi]
    Per ogni coppia di flussi $f_1$ e $f_2$ in $G$, definiamo il flusso somma
    $g=f_1+f_2$ come un flusso tale che:
    \[g(x,y)=f_1(x,y)+f_2(x,y)\qquad\forall x,y\in V\]
\end{definition}

\begin{figure}[h!]
    \centering
    \begin{graph}
        \node[main] (a1) [label=above left:{$s$}] {$A$};
        \node[main] (b1) [above right of=a1, xshift=10mm] {$B$};
        \node[main] (c1) [below right of=a1, xshift=10mm] {$C$};
        \node[main] (d1) [below right of=b1, xshift=10mm,
            label=above right:{$t$}] {$D$};

        \path[->]   (a1) edge node[above left] {$1/2$} (b1)
                    (b1) edge node[above right] {$1/3$} (d1)
                    (a1) edge node[below left] {$1/2$} (c1)
                    (c1) edge node[right] {$0/1$} (b1)
                    (c1) edge node[below right] {$1/1$} (d1);

        \node[] (plus) [below of=c1, yshift=10mm] {$+$};

        \node[main] (b2) [below of=plus, yshift=10mm] {$B$};
        \node[main] (a2) [below left of=b2, xshift=-10mm,
            label=above left:{$s$}] {$A$};
        \node[main] (c2) [below right of=a2, xshift=10mm] {$C$};
        \node[main] (d2) [below right of=b2, xshift=10mm,
            label=above right:{$t$}] {$D$};

        \path[->]   (a2) edge node[above left] {$1/2$} (b2)
                    (b2) edge node[above right] {$2/3$} (d2)
                    (a2) edge node[below left] {$1/2$} (c2)
                    (c2) edge node[right] {$1/1$} (b2)
                    (c2) edge node[below right] {$0/1$} (d2);

        \node[] (equals) [right of=plus, xshift=20mm] {$=$};

        \node[main] (a3) [right of=equals, label=above left:{$s$}] {$A$};
        \node[main] (b3) [above right of=a3, xshift=10mm] {$B$};
        \node[main] (c3) [below right of=a3, xshift=10mm] {$C$};
        \node[main] (d3) [below right of=b3, xshift=10mm,
            label=above right:{$t$}] {$D$};

        \path[->]   (a3) edge node[above left] {$2/2$} (b3)
                    (b3) edge node[above right] {$3/3$} (d3)
                    (a3) edge node[below left] {$1/2$} (c3)
                    (c3) edge node[right] {$1/1$} (b3)
                    (c3) edge node[below right] {$1/1$} (d3);
    \end{graph}
    \caption{\emph{Somma di flussi}}
\end{figure}

\begin{definition}[Capacità residua]
    È definita capacità residua di un flusso $f$ in una rete $G=(V,E,s,t,c)$ una
    funziona $c_f:V\times V\to\mathbb{R^+}$ tale che:
    \[c_f(x,y)=c(x,y)-f(x,y)\qquad\forall x,y\in V\]
\end{definition}

\begin{figure}[h!]
    \centering
    \begin{graph}
        \node[main] (a) {$A$};
        \node[main] (b) [right of=a, xshift=15mm] {$B$};

        \path[->]   (a) edge[bend left=25] node[above] {$5+\underline{2}/7$} (b)
                    (b) edge[bend left=25, dashed] node[above] {$-5+\underline{5}/0$} (a);
    \end{graph}
    \caption{\emph{Capacità residua}}
\end{figure}

\noindent
Possiamo notare che gli \emph{archi all'indietro} che avevamo visto enunciando
la proprietà di \emph{antisimmetria} dei \emph{flussi}, si originano dalla
definizione di \emph{capacità residua}. Infatti, poiché l'\emph{arco} $(B,A)$
non esiste, $c(B,A)=0$ e quindi, vale la seguente catena di uguaglianze:
\[c_f(B,A)\begin{array}[t]{cll}
    = & c(B,A)-f(B,A) & \quad\text{Dalla definizione di \emph{capacità residua}}\\
    = & c(B,A)-(-f(A,B)) & \quad\text{Per la proprietà di \hyperref[def:144]
        {\emph{antisimmetria}}}\\
    = & 0 - (-5) = 5
\end{array}\]

\newpage
\begin{definition}[Rete residua]
    Data una rete di flusso $G=(V,E,s,t,c)$ e un flusso $f$ su $G$, una rete
    $G_f=(V,E_f,s,t,c_f)$ è una rete residua di $G$ rispetto ad $f$, se è tale
    per cui:
    \[c_f(x,y)>0\qquad\forall x,y\in V\::\:(x,y)\in E_f\]
\end{definition}

\begin{figure}[h!]
\centering
\begin{graph}
    \node[main] (a) [label=above left:{$s$}] {$A$};
    \node[main] (b) [above right of=a, xshift=40mm, yshift=10mm] {$B$};
    \node[main] (c) [below right of=a, xshift=40mm, yshift=-10mm] {$C$};
    \node[main] (d) [below right of=b, xshift=40mm, yshift=-10mm,
        label=above right:{$t$}] {$D$};

    \path[->]   (a) edge[bend left=25, dotted] node[above left]
                    {$2+\underline{0}/2$} (b)
                (b) edge[bend left=25, dashed] node[above left]
                    {$-2+\underline{2}/0$} (a)
                (b) edge[bend left=25] node[above right]
                    {$2+\underline{1}/3$} (d)
                (d) edge[bend left=25, dashed] node[above right]
                    {$-2+\underline{2}/0$} (b)
                (a) edge[bend right=25] node[below left]
                    {$1+\underline{1}/2$} (c)
                (c) edge[bend right=25, dashed] node[below left]
                    {$-1+\underline{1}/0$} (a)
                (c) edge[bend left=25] node[left]
                    {$0+\underline{1}/1$} (b)
                (b) edge[bend left=25, dashed] node[right]
                    {$-0+\underline{0}/0$} (c)
                (c) edge[bend right=25, dotted] node[below right]
                    {$1+\underline{0}/1$} (d)
                (d) edge[bend right=25, dashed] node[below right]
                    {$-1+\underline{1}/0$} (c);
\end{graph}
\caption{Rete residua}
\end{figure}

\noindent
Gli \emph{archi} $(A,B)$ e $(C,D)$ sono stati saturati, ovvero $f(A,B)=c(A,B)$
e $f(C,D)=c(C,D)$. Di conseguenza, la loro \emph{capacità residua} è $0$ e
quindi non fanno parte della \emph{rete residua}.

\begin{minicode}{Schema generale dell'algoritmo}
\ind\bc{int}[][] maxFlow(\bc{GRAPH} G, \bc{NODE} s, \bc{NODE} t, \bc{int}[][] c)\\
    \bc{FLOW} f = f$_0$\hfill\com{Inizializza $f$ a un \emph{flusso nullo}}
    \bc{int}[][] c$_f$ = c\hfill\com{Inizialmente la \emph{capacità residua} è
    pari a quella iniziale}
    \bc{FLOW} g\\
    \indf do\\
        g = \{ trova un \emph{flusso} in c$_f$ tale che $|g|>0$, altrimenti f$_0$ \}\\
        f = f + g\\
        c$_f$ = \{ \emph{capacità} della \emph{rete residua} di $G$ rispetto ad $f$ \}\\
    \indf while (g $\neq$ f$_0$)\\
    \indf return f
\end{minicode}

\paragraph{Dimostrazione}
Per dimostrare la correttezza dell'algoritmo dobbiamo dimostrare il seguente
lemma:

\begin{definition}[Lemma della somma di flussi]
    Se $f$ è un flusso in $G$ e $g$ è un flusso in $G_f$, allora anche $f+g$ è
    un flusso in $G$.
\end{definition}
\begin{proof}[Dimostrazione]
    Dobbiamo dimostrare che $f+g$ rispetta tutte e tre le proprietà dei
    \emph{flussi}.

    \paragraph{Vincolo sulla capacità}
    Per ogni coppia di \emph{nodi} $x,y\in V$ vale la seguente catena di
    disuguaglianze:
    \[\begin{array}{rcll}
        g(x,y) & \leq & c_f(x,y) & \quad\text{È vero perché $g$ è un
            \emph{flusso} in $G_f$}\\
        f(x,y)+g(x,y) & \leq & c_f(x,y)+f(x,y)\\
        (f+g)(x,y) &  \leq & c(x,y)-f(x,y)+f(x,y) & \quad\text{Definizione di
            \hyperref[def:148]{\emph{capacità residua}}}\\
        (f+g)(x,y) & \leq & c(x,y) & \quad\text{Semplificazione algebrica}
    \end{array}\]

    \paragraph{Antisimmetria}
    Per ogni coppia di \emph{nodi} $x,y\in V$ vale la seguente catena di
    uguaglianze:
    \[\begin{array}{rcll}
        f(x,y)+g(x,y) & = & -f(y,x)-g(x,y)\\
        f(x,y)+g(x,y) & = & -(f(x,y)+g(x,y))\\
        (f+g)(x,y) & = & -(f+g)(x,y)
    \end{array}\]

    \paragraph{Conservazione del flusso}
    Per ogni coppia di \emph{nodi} $x,y\in V$ vale la seguente catena di
    uguaglianze:
    \[\sum_{y\in V}(f+g)(x,y)\begin{array}[t]{cl}
        = & \sum_{y\in V}(f(x,y)+g(x,y))\\
        = & \sum_{y\in V}f(x,y)+\sum_{y\in V}g(x,y)\\
        = & 0 + 0\\
        = & 0
    \end{array}\]
\end{proof}

\subsection{Metodo dei cammini aumentanti}
Quindi l'algoritmo che abbiamo descritto è corretto, tuttavia rimane un problema:
come facciamo a trovare il \emph{flusso} $g$?

Una soluzione è quella proposta da Ford e Fulkerson nel 1956, e raffinata poi
da Edmonds e Karp nel 1972, che fa uso dei cosiddetti \emph{cammini aumentanti}.
In particolare, viene identificato nella \emph{rete residua} $G_f$ un
\emph{cammino} $C=(v_0,\dots,v_n)$ tale per cui $v_0=s$ e $v_n=t$. Quindi, si
definisce la \emph{capacità} del \emph{cammino} come la minore tra le
\emph{capacità} degli \emph{archi} attraversati, ovvero:
\[c_f(C)=\min_{i\in\{2,\dots,n\}}c_f(v_{i-1},v_i)\]

\begin{note}
    La \emph{capacità} di un \emph{cammino} è definibile anche come la
    \emph{capacità} del suo \q{collo di bottiglia}.
\end{note}

\noindent
Infine, il \emph{flusso} addizionale $g$ è definito in modo che $g(v_{i-1},v_i)=
c_f(C)$, $g(v_i,v_{i-1})=-c_f(C)$ per la proprietà \emph{antisimmetrica} e
$g(x,y)=0$ per tutte le coppie $x,y\notin C$.

\begin{figure}[h!]
\centering
\subfloat[\emph{Grafo} iniziale]{\resizebox*{0.48\textwidth}{!}{\begin{graph}
    \node[main] (a) [label=above left:{$s$}] {$a$};
    \node[main] (b) [above right of=a, xshift=10mm] {$b$};
    \node[main] (c) [below right of=a, xshift=10mm] {$c$};
    \node[main] (d) [right of=b, xshift=10mm] {$d$};
    \node[main] (e) [right of=c, xshift=10mm] {$e$};
    \node[main] (f) [below right of=d, xshift=10mm, label={above right:{$t$}}] {$f$};

    \path[->]   (a) edge node[above left] {$2$} (b)
                (a) edge node[below left] {$8$} (c)
                (b) edge node[right] {$2$} (c)
                (b) edge node[above] {$7$} (d)
                (b) edge node[above right] {$3$} (e)
                (c) edge node[below] {$6$} (e)
                (d) edge node[right] {$2$} (e)
                (d) edge node[above right] {$3$} (f)
                (e) edge node[below right] {$4$} (f);
\end{graph}}}\hfill
\subfloat[Scelta di un \emph{cammino}]{\resizebox*{0.48\textwidth}{!}{\begin{graph}
    \node[main] (a) [label=above left:{$s$}] {$a$};
    \node[main] (b) [above right of=a, xshift=10mm] {$b$};
    \node[main] (c) [below right of=a, xshift=10mm] {$c$};
    \node[main] (d) [right of=b, xshift=10mm] {$d$};
    \node[main] (e) [right of=c, xshift=10mm] {$e$};
    \node[main] (f) [below right of=d, xshift=10mm, label={above right:{$t$}}] {$f$};

    \path[->]   (a) edge[line width=1.3pt] node[above left] {$2$} (b)
                (a) edge node[below left] {$8$} (c)
                (b) edge node[right] {$2$} (c)
                (b) edge node[above] {$7$} (d)
                (b) edge[line width=1.3pt] node[above right] {$3$} (e)
                (c) edge node[below] {$6$} (e)
                (d) edge node[right] {$2$} (e)
                (d) edge node[above right] {$3$} (f)
                (e) edge[line width=1.3pt] node[below right] {$4$} (f);
\end{graph}}}\\
\subfloat[Definizione del \emph{flusso} $g$]{\resizebox*{0.48\textwidth}{!}{\begin{graph}
    \node[main] (a) [label=above left:{$s$}] {$a$};
    \node[main] (b) [above right of=a, xshift=10mm] {$b$};
    \node[main] (c) [below right of=a, xshift=10mm] {$c$};
    \node[main] (d) [right of=b, xshift=10mm] {$d$};
    \node[main] (e) [right of=c, xshift=10mm] {$e$};
    \node[main] (f) [below right of=d, xshift=10mm, label={above right:{$t$}}] {$f$};

    \path[->]   (a) edge[bend left=25] node[above left] {$2/2$} (b)
                (b) edge[bend left=25, dashed] node[above, xshift=-2mm] {$-2/0$} (a)
                (a) edge node[below left] {$8$} (c)
                (b) edge node[right] {$2$} (c)
                (b) edge node[above] {$7$} (d)
                (b) edge[bend left=25] node[above right, yshift=-2mm] {$2/3$} (e)
                (e) edge[bend left=25, dashed] node[above, xshift=3mm] {$-2/3$} (b)
                (c) edge node[below] {$6$} (e)
                (d) edge node[right] {$2$} (e)
                (d) edge node[above right] {$3$} (f)
                (e) edge[bend right=25] node[below right] {$2/4$} (f)
                (f) edge[bend right=25, dashed] node[below, xshift=2mm] {$-2/4$} (e);
\end{graph}}}
\caption{\emph{Cammini aumentanti} in generale}
\end{figure}

\begin{minicode}{Schema generale dell'algoritmo}
\ind\bc{int}[][] maxFlow(\bc{GRAPH} G, \bc{NODE} s, \bc{NODE} t, \bc{int}[][] c)\\
    \bc{int}[][] f = new \bc{int}[1\dots G.size()][1\dots G.size()]\hfill\com{\emph{Flusso} parziale}
    \bc{int}[][] g = new \bc{int}[1\dots G.size()][1\dots G.size()]\hfill
        \com{\emph{Flusso}\,da\;\emph{cammino\,aumentante}}
    \bc{int}[][] c$_f$ = new \bc{int}[1\dots G.size()][1\dots G.size()]\hfill\com{\emph{Capacità residua}}
    \indf foreach (x,y $\in$ G.V()) do\\
        f[x][y] = 0\hfill\com{Inizializza un \emph{flusso nullo}}
        c$_f$[x][y] = c[x][y]\\
    \indf do\\
        g = \{ \emph{flusso} associato ad un \emph{cammino aumentante} in $c_f$, oppure $f_0$ \}\\
        \indff foreach (x,y $\in$ G.V()) do\\
            f[x][y] = f[x][y] + g[x][y]\hfill\com{\emph{Somma dei flussi}:\:$f=f+g$}
            c$_f$ = c[x][y] - f[x][y]\hfill\com{Aggiorna la \emph{capacità residua}}
    \indf while (g $\neq$ f$_0$)\\
    \indf return f
\end{minicode}

\begin{eg}[Esempio d'esecuzione]
Supponiamo di eseguire l'algoritmo sullo stesso \emph{grafo} di prima:

\begin{figure}[h!]
\centering
\subfloat[\emph{Grafo} iniziale e $f=f_0$]{\resizebox*{0.48\textwidth}{!}{\begin{graph}
    \node[main] (a) [label=above left:{$s$}] {$a$};
    \node[main] (b) [above right of=a, xshift=10mm] {$b$};
    \node[main] (c) [below right of=a, xshift=10mm] {$c$};
    \node[main] (d) [right of=b, xshift=10mm] {$d$};
    \node[main] (e) [right of=c, xshift=10mm] {$e$};
    \node[main] (f) [below right of=d, xshift=10mm, label={above right:{$t$}}] {$f$};

    \path[->]   (a) edge node[above left] {$2$} (b)
                (a) edge node[below left] {$8$} (c)
                (b) edge node[right] {$2$} (c)
                (b) edge node[above] {$7$} (d)
                (b) edge node[above right] {$3$} (e)
                (c) edge node[below] {$6$} (e)
                (d) edge node[right] {$2$} (e)
                (d) edge node[above right] {$3$} (f)
                (e) edge node[below right] {$4$} (f);
\end{graph}}}\hfill
\subfloat[Scelta di un \emph{cammino} in $G_f$]{\resizebox*{0.48\textwidth}{!}{\begin{graph}
    \node[main] (a) [label=above left:{$s$}] {$a$};
    \node[main] (b) [above right of=a, xshift=10mm] {$b$};
    \node[main] (c) [below right of=a, xshift=10mm] {$c$};
    \node[main] (d) [right of=b, xshift=10mm] {$d$};
    \node[main] (e) [right of=c, xshift=10mm] {$e$};
    \node[main] (f) [below right of=d, xshift=10mm, label={above right:{$t$}}] {$f$};

    \path[->]   (a) edge[line width=1.3pt] node[above left] {$2$} (b)
                (a) edge node[below left] {$8$} (c)
                (b) edge node[right] {$2$} (c)
                (b) edge node[above] {$7$} (d)
                (b) edge[line width=1.3pt] node[above right] {$3$} (e)
                (c) edge node[below] {$6$} (e)
                (d) edge node[right] {$2$} (e)
                (d) edge node[above right] {$3$} (f)
                (e) edge[line width=1.3pt] node[below right] {$4$} (f);
\end{graph}}}\\
\subfloat[Aggiornamento di $f$]{\resizebox*{0.48\textwidth}{!}{\begin{graph}
    \node[main] (a) [label=above left:{$s$}] {$a$};
    \node[main] (b) [above right of=a, xshift=10mm] {$b$};
    \node[main] (c) [below right of=a, xshift=10mm] {$c$};
    \node[main] (d) [right of=b, xshift=10mm] {$d$};
    \node[main] (e) [right of=c, xshift=10mm] {$e$};
    \node[main] (f) [below right of=d, xshift=10mm, label={above right:{$t$}}] {$f$};

    \path[->]   (b) edge[dashed] node[above, xshift=-2mm] {$2$} (a)
                (a) edge node[below left] {$8$} (c)
                (b) edge node[right] {$2$} (c)
                (b) edge node[above] {$7$} (d)
                (b) edge[bend left=15] node[above right, yshift=-2mm] {$1$} (e)
                (e) edge[bend left=15, dashed] node[above, xshift=3mm, yshift=-1mm] {$2$} (b)
                (c) edge node[below] {$6$} (e)
                (d) edge node[right] {$2$} (e)
                (d) edge node[above right] {$3$} (f)
                (e) edge[bend right=15] node[below right, xshift=-1mm] {$2$} (f)
                (f) edge[bend right=15, dashed] node[below, xshift=2mm, yshift=1mm] {$2$} (e);

    \node[] (l2) [right of=f] {$f(b,e)=2$};
    \node[] (l1) [above of=l2, yshift=-13mm] {$f(a,b)=2$};
    \node[] (l3) [below of=l2, yshift=13mm] {$f(e,f)=2$};

    \node[inner sep=0] (0) [below of=c, yshift=9.5mm] {};
\end{graph}}}
\hfill
\subfloat[Scelta di un \emph{cammino} in $G_f$]{\resizebox*{0.48\textwidth}{!}{\begin{graph}
    \node[main] (a) [label=above left:{$s$}] {$a$};
    \node[main] (b) [above right of=a, xshift=10mm] {$b$};
    \node[main] (c) [below right of=a, xshift=10mm] {$c$};
    \node[main] (d) [right of=b, xshift=10mm] {$d$};
    \node[main] (e) [right of=c, xshift=10mm] {$e$};
    \node[main] (f) [below right of=d, xshift=10mm, label={above right:{$t$}}] {$f$};

    \path[->]   (b) edge[dashed] node[above, xshift=-2mm] {$2$} (a)
                (a) edge[line width=1.3pt] node[below left] {$8$} (c)
                (b) edge node[right] {$2$} (c)
                (b) edge node[above] {$7$} (d)
                (b) edge[bend left=15] node[above right, yshift=-2mm] {$1$} (e)
                (e) edge[bend left=15, dashed] node[above, xshift=3mm, yshift=-1mm] {$2$} (b)
                (c) edge[line width=1.3pt] node[below] {$6$} (e)
                (d) edge node[right] {$2$} (e)
                (d) edge node[above right] {$3$} (f)
                (e) edge[bend right=15, line width=1.3pt] node[below right,
                    xshift=-1mm] {$2$} (f)
                (f) edge[bend right=15, dashed] node[below, xshift=2mm, yshift=1mm] {$2$} (e);
\end{graph}}}\\
\subfloat[Aggiornamento di $f$]{\resizebox*{0.48\textwidth}{!}{\begin{graph}
    \node[main] (a) [label=above left:{$s$}] {$a$};
    \node[main] (b) [above right of=a, xshift=10mm] {$b$};
    \node[main] (c) [below right of=a, xshift=10mm] {$c$};
    \node[main] (d) [right of=b, xshift=10mm] {$d$};
    \node[main] (e) [right of=c, xshift=10mm] {$e$};
    \node[main] (f) [below right of=d, xshift=10mm, label={above right:{$t$}}] {$f$};

    \path[->]   (b) edge[dashed] node[above, xshift=-2mm] {$2$} (a)
                (a) edge[bend right=15] node[below left,  yshift=1mm] {$6$} (c)
                (c) edge[bend right=15, dashed] node[below left, yshift=1mm,
                    xshift=1mm] {$2$} (a)
                (b) edge node[right] {$2$} (c)
                (b) edge node[above] {$7$} (d)
                (b) edge[bend left=15] node[above right, yshift=-2mm] {$1$} (e)
                (e) edge[bend left=15, dashed] node[above, xshift=3mm, yshift=-1mm] {$2$} (b)
                (c) edge[bend right=15] node[below] {$4$} (e)
                (e) edge[bend right=15, dashed] node[below] {$2$} (c)
                (d) edge node[right] {$2$} (e)
                (d) edge node[above right] {$3$} (f)
                (f) edge[dashed] node[below, xshift=2mm, yshift=1mm] {$4$} (e);
    
    \node[] (l2) [right of=f] {$f(a,c)=2$};
    \node[] (l1) [above of=l2, yshift=-13mm] {$f(c,e)=2$};
    \node[] (l3) [below of=l2, yshift=13mm] {$f(e,f)=\cancel{2}\,4$};

    \node[inner sep=0] (0) [below of=c, yshift=6.5mm] {};
\end{graph}}}
\hfill
\subfloat[Scelta di un \emph{cammino} in $G_f$]{\resizebox*{0.48\textwidth}{!}{\begin{graph}
    \node[main] (a) [label=above left:{$s$}] {$a$};
    \node[main] (b) [above right of=a, xshift=10mm] {$b$};
    \node[main] (c) [below right of=a, xshift=10mm] {$c$};
    \node[main] (d) [right of=b, xshift=10mm] {$d$};
    \node[main] (e) [right of=c, xshift=10mm] {$e$};
    \node[main] (f) [below right of=d, xshift=10mm, label={above right:{$t$}}] {$f$};

    \path[->]   (b) edge[dashed] node[above, xshift=-2mm] {$2$} (a)
                (a) edge[bend right=15, line width=1.3pt] node[below left,
                    yshift=1mm] {$6$} (c)
                (c) edge[bend right=15, dashed] node[below left, yshift=1mm,
                    xshift=1mm] {$2$} (a)
                (b) edge node[right] {$2$} (c)
                (b) edge[line width=1.3pt] node[above] {$7$} (d)
                (b) edge[bend left=15] node[above right, yshift=-2mm] {$1$} (e)
                (e) edge[bend left=15, dashed, line width=1.3pt] node[above, xshift=3mm, yshift=-1mm] {$2$} (b)
                (c) edge[bend right=15, line width=1.3pt] node[below] {$4$} (e)
                (e) edge[bend right=15, dashed] node[below] {$2$} (c)
                (d) edge node[right] {$2$} (e)
                (d) edge[line width=1.3pt] node[above right] {$3$} (f)
                (f) edge[dashed] node[below, xshift=2mm, yshift=1mm] {$4$} (e);
\end{graph}}}\\
\end{figure}
\newpage
\begin{figure}[ht!]
\ContinuedFloat
\centering
\subfloat[Aggiornamento di $f$]{\begin{graph}
    \node[main] (a) [label=above left:{$s$}] {$a$};
    \node[main] (b) [above right of=a, xshift=10mm] {$b$};
    \node[main] (c) [below right of=a, xshift=10mm] {$c$};
    \node[main] (d) [right of=b, xshift=10mm] {$d$};
    \node[main] (e) [right of=c, xshift=10mm] {$e$};
    \node[main] (f) [below right of=d, xshift=10mm, label={above right:{$t$}}] {$f$};

    \path[->]   (b) edge[dashed] node[above, xshift=-2mm] {$2$} (a)
                (a) edge[bend right=15] node[below left,  yshift=1mm] {$4$} (c)
                (c) edge[bend right=15, dashed] node[below left, yshift=1mm,
                    xshift=1mm] {$4$} (a)
                (b) edge node[right] {$2$} (c)
                (b) edge[bend left=15] node[above] {$5$} (d)
                (d) edge[bend left=15, dashed] node[above] {$2$} (b)
                (b) edge node[above right, yshift=-2mm] {$3$} (e)
                (c) edge[bend right=15] node[below] {$2$} (e)
                (e) edge[bend right=15, dashed] node[below] {$4$} (c)
                (d) edge node[right] {$2$} (e)
                (d) edge[bend left=15] node[above right] {$1$} (f)
                (f) edge[bend left=15, dashed] node[above right, yshift=-1mm] {$2$} (d)
                (f) edge[dashed] node[below, xshift=2mm, yshift=1mm] {$4$} (e);
    
    \node[] (l3) [right of=f] {$f(b,e)=\cancel{2}\,0$};
    \node[] (l2) [above of=l3, yshift=-13mm] {$f(c,e)=4$};
    \node[] (l1) [above of=l2, yshift=-13mm] {$f(a,c)=4$};
    \node[] (l4) [below of=l3, yshift=13mm] {$f(b,d)=2$};
    \node[] (l5) [below of=l4, yshift=13mm] {$f(d,f)=2$};
\end{graph}}
\end{figure}

\noindent
Nel passaggio \emph{(f)}, il cammino scelto passa per l'\emph{arco} $(e,b)$
che però è un arco all'indietro, quindi $f(e,b)=-2$. Il motivo per il quale
possiamo usarlo è che $c_f(e,b)=c(e,b)-f(e,b)=2$, tuttavia, usarlo significa
\q{annullare} il precedente flusso attraverso $(b,e)$ e infatti al passo
\emph{(g)} $f(b,e)=0$. Questo \q{tornare indietro} è caratteristico degli
algoritmi basati su ricerca locale.
\end{eg}

\noindent
Rimane comunque il problema di ricercare un \emph{cammino}, ma per quello
possiamo usare una \hyperref[code:48]{\emph{visita in ampiezza}} o in
\hyperref[code:49]{\emph{profondità}}.

\begin{minicode}{Implementazione maxFlow con metodo dei cammini aumentanti}
\ind\bc{int}[][] maxFlow(\bc{int}[][] c, \bc{int} n, \bc{NODE} s, \bc{NODE} t)\\
    \bc{int}[][] f = new \bc{int}[1\dots n][1\dots n]\hfill\com{Matrice del \emph{flusso}}
    \bc{boolean}[] visited = new \bc{int}[1\dots n] = \{false\}\\
    \com{Rimane nel ciclo fino a quando esistono \emph{cammini} da $s$ a $t$}
    \indf while (dfs(c, f, n, s, t, visited, $+\infty$) > 0) do\\
        \indff for (i = 1 to n) do\\
            visited[i] = false\hfill\com{Resetta il vettore per la \emph{visita} successiva}
    \indf return f\\

\nl\com{Versione modificata della dfs}
\rmindent\ind\bc{int} dfs(\bc{int}[][] c, \bc{int}[][] f,  \bc{int} n, \bc{int} s,
    \bc{int} t, \bc{boolean}[] visited, \bc{int}\:minCap)\\
    \indf if (s == t) then\\
        return min\\
    \indf visited[i] = true\\
    \indf for (i = 1 to n) do\\
        \indff if (c[s][i] > 0 and not visited[i]) then\\
            \bc{int} val = dfs(c, f, n, i, t, visited, min(minCap, c[s][i]))\\
            \indfff if (val > 0) then\\
                c[s][i] = c[s][i] - val\hfill\com{Aggiorna la \emph{capacità}
                dell'\emph{arco} $(s,i)$}
                c[i][s] = c[i][s] + val\hfill\com{Aggiorna la \emph{capacità}
                dell'\emph{arco} $(i,s)$}
                f[s][i] = f[s][i] + val\hfill\com{Aggiorna la \emph{capacità}
                del \emph{flusso} in $(s,i)$}
                f[i][s] = f[i][s] - val\hfill\com{Aggiorna la \emph{capacità}
                del \emph{flusso} in $(i,s)$}
                return val\\
    \indf return 0\hfill\com{Se l'esecuzione arriva qui non è stato trovato un
        \emph{cammino}}
\end{minicode}

\paragraph{Note sull'implementazione}
Il \emph{grafo} è definito esclusivamente mediante la \emph{matrice delle
capacità} e in particolare, esiste un \emph{arco} $(x,y)$ solo se $c[x][y]>0$.
Il parametro \texttt{n} rappresenta il numero di \emph{nodi}. Nella
\texttt{dfs}, il parametro \texttt{minCap} indica il più basso valore di
\emph{capacità} incontrato nella ricerca del \emph{cammino}.

\paragraph{Dimostrazione}
La dimostrazione di correttezza per l'algoritmo ci richiede di dare delle altre
definizioni.

\begin{definition}[Taglio]
    Un taglio $(S,T)$ della rete di flusso $G=(V,E,s,t,c)$ è una partizione di
    $V$ in due sottoinsiemi disgiunti tali per cui $S=V-T$,$s\in S$ e $t\in T$.
\end{definition}

\begin{figure}[h!]
\centering
\begin{graph}
    \node[main] (a) [label=above left:{$s$}] {$a$};
    \node[main] (b) [above right of=a, xshift=10mm] {$b$};
    \node[main] (c) [below right of=a, xshift=10mm] {$c$};
    \node[main] (d) [right of=b, xshift=10mm] {$d$};
    \node[main] (e) [right of=c, xshift=10mm] {$e$};
    \node[main] (f) [below right of=d, xshift=10mm, label={above right:{$t$}}] {$f$};

    \path[->]   (a) edge node[above left] {$2$} (b)
                (a) edge node[below left] {$8$} (c)
                (b) edge node[right] {$2$} (c)
                (b) edge node[above] {$7$} (d)
                (b) edge node[above right] {$3$} (e)
                (c) edge node[below] {$6$} (e)
                (d) edge node[right] {$2$} (e)
                (d) edge node[above right] {$3$} (f)
                (e) edge node[below right] {$4$} (f);

    \node[] (0) at (e) [shift={(-25mm, -7.5mm)}] {};
    \draw[-, dashed] (d)+(12mm, 2mm) -- (0);

    \node[] (sets) [above right of=f, xshift=10mm] {$\begin{array}{l}
        S=\{a,b,c,d\}\\
        T=\{e,f\}
    \end{array}$};
\end{graph}
\caption{\emph{Taglio} di una \emph{rete di flusso}}
\end{figure}

\begin{definition}[Capacità di un taglio]
    La capacità $C(S,T)$ attraverso il taglio $(S,T)$ è definita come:
    \[C(S,T)=\sum_{x\in S,y\in T}c(x,y)\]
\end{definition}

\begin{figure}[h!]
\centering
\begin{graph}
    \node[main] (a) [label=above left:{$s$}] {$a$};
    \node[main] (b) [above right of=a, xshift=10mm] {$b$};
    \node[main] (c) [below right of=a, xshift=10mm] {$c$};
    \node[main] (d) [right of=b, xshift=10mm] {$d$};
    \node[main] (e) [right of=c, xshift=10mm] {$e$};
    \node[main] (f) [below right of=d, xshift=10mm, label={above right:{$t$}}] {$f$};

    \path[->]   (a) edge node[above left] {$2$} (b)
                (a) edge node[below left] {$8$} (c)
                (b) edge node[right] {$2$} (c)
                (b) edge node[above] {$7$} (d)
                (b) edge[line width=1.3pt] node[above right] {$3$} (e)
                (c) edge[line width=1.3pt] node[below] {$6$} (e)
                (d) edge[line width=1.3pt] node[right] {$2$} (e)
                (d) edge[line width=1.3pt] node[above right] {$3$} (f)
                (e) edge node[below right] {$4$} (f);

    \node[] (0) at (e) [shift={(-25mm, -7.5mm)}] {};
    \draw[-, dashed] (d)+(12mm, 2mm) -- (0);

    \node[] (sets) [right of=f, shift={(5mm, 9mm)}] {$\begin{array}[t]{l}
        S=\{a,b,c,d\}\\
        T=\{e,f\}\\
        \\
        C(S,T)=14
    \end{array}$};
\end{graph}
\caption{\emph{Capacità di un taglio}}
\end{figure}

\begin{definition}[Flusso netto di un taglio]
    Se $f$ è un flusso in $G$, il flusso netto $F_f(S,T)$ attraverso $(S,T)$ è:
    \[F_f(S,T)=\sum_{x\in S,y\in T}f(x,y)\]
\end{definition}

\newpage
\begin{figure}[ht!]
\centering
\begin{graph}
    \node[main] (a) [label=above left:{$s$}] {$a$};
    \node[main] (b) [above right of=a, xshift=10mm] {$b$};
    \node[main] (c) [below right of=a, xshift=10mm] {$c$};
    \node[main] (d) [right of=b, xshift=10mm] {$d$};
    \node[main] (e) [right of=c, xshift=10mm] {$e$};
    \node[main] (f) [below right of=d, xshift=10mm, label={above right:{$t$}}] {$f$};

    \path[->]   (a) edge node[above left] {$2/2$} (b)
                (a) edge node[below left] {$4/8$} (c)
                (b) edge node[right] {$0/2$} (c)
                (b) edge node[above] {$2/7$} (d)
                (b) edge[line width=1.3pt] node[above right] {$0/3$} (e)
                (c) edge[line width=1.3pt] node[below] {$4/6$} (e)
                (d) edge[line width=1.3pt] node[right] {$0/2$} (e)
                (d) edge[line width=1.3pt] node[above right] {$2/3$} (f)
                (e) edge node[below right] {$4/4$} (f);

    \node[] (0) at (e) [shift={(-25mm, -7.5mm)}] {};
    \draw[-, dashed] (d)+(12mm, 2mm) -- (0);

    \node[] (sets) [right of=f, shift={(5mm, 5mm)}] {$\begin{array}[t]{l}
        S=\{a,b,c,d\}\\
        T=\{e,f\}\\
        \\
        C(S,T)=14\\
        F_f(S,T)=6
    \end{array}$};
\end{graph}
\caption{\emph{Flusso netto di un taglio}}
\end{figure}

\begin{definition}[Lemma del valore del flusso di un taglio]
    Dati un flusso $f$ e un taglio $(S,T)$, il flusso netto $F_f(S,T)$ che
    attraverso il taglio è uguale a $|f|$.
\end{definition}

\begin{figure}[h!]
\centering
\begin{graph}
    \node[main] (a) [label=above left:{$s$}] {$a$};
    \node[main] (b) [above right of=a, xshift=10mm] {$b$};
    \node[main] (c) [below right of=a, xshift=10mm] {$c$};
    \node[main] (d) [right of=b, xshift=10mm] {$d$};
    \node[main] (e) [right of=c, xshift=10mm] {$e$};
    \node[main] (f) [below right of=d, xshift=10mm, label={above right:{$t$}}] {$f$};

    \path[->]   (a) edge node[above left] {$2/2$} (b)
                (a) edge[line width=1.3pt] node[below left] {$4/8$} (c)
                (b) edge node[right] {$0/2$} (c)
                (b) edge[line width=1.3pt] node[above] {$2/7$} (d)
                (b) edge[line width=1.3pt] node[above right] {$0/3$} (e)
                (c) edge[line width=1.3pt] node[below] {$4/6$} (e)
                (d) edge[line width=1.3pt] node[right] {$0/2$} (e)
                (d) edge[line width=1.3pt] node[above right] {$2/3$} (f)
                (e) edge node[below right] {$4/4$} (f);

    \node[] (0) at (e) [shift={(-25mm, -7.5mm)}] {};
    \draw[-, dashed] (d)+(12mm, 2mm) -- (0);

    \node[] (1) at (c) [shift={(-12mm, -5mm)}] {};
    \draw[-, dashed] (b)+(10.5mm, 6mm) -- (1);

    \node[] (sets) [right of=f, shift={(15mm, 5mm)}] {$\begin{array}[t]{l}
        S=\{a,b,c,d\}, T=\{e,f\}\\
        S'=\{a,b\}, T'=\{c,d,e,f\}\\
        \\
        F_f(S,T)=6\\
        F_{f}(S',T')=6
    \end{array}$};
\end{graph}
\caption{\emph{Valore del flusso di un taglio}}
\end{figure}

\begin{proof}[Dimostrazione]
    Vale la seguente catena di uguaglianze:
    \[F_f(S,T)\begin{array}[t]{cll}
        = & \sum_{x\in S,y\in T}f(x,y)\\
        = & \sum_{x\in S,y\in v}f(x,y) - \sum_{x\in S,y\in S}f(x,y) & \quad T=V-S\\
        = & \sum_{x\in S,y\in V}f(x,y)-0 & \quad\emph{Antisimmetria}\\
        = & \sum_{x\in S-\{s\}, y\in V}f(x,y)+\sum_{y\in V}f(s,y) & \quad s\in S\\
        = & \sum_{x\in S-\{s\}}\left(\sum_{y\in V}f(x,y)\right)+\sum_{y\in V}f(s,y)\\
        = & \sum_{x\in S-\{s\}} 0+\sum_{y\in V}f(s,y) & \quad\emph{Conservazione del flusso}\\
        = & \sum_{y\in V} f(s,y)=|f| & \quad\emph{\hyperref[def:145]{Definizione di valore del flusso}}
    \end{array}\]
\end{proof}

\begin{definition}[Lemma del limite superiore al flusso massimo]
    Il flusso massimo è limitato superiormente dalla capacità del taglio minimo,
    cioè del taglio con minor capacità.
\end{definition}

\newpage
\begin{figure}[ht!]
\centering
\begin{graph}
    \node[main] (a) [label=above left:{$s$}] {$a$};
    \node[main] (b) [above right of=a, xshift=10mm] {$b$};
    \node[main] (c) [below right of=a, xshift=10mm] {$c$};
    \node[main] (d) [right of=b, xshift=10mm] {$d$};
    \node[main] (e) [right of=c, xshift=10mm] {$e$};
    \node[main] (f) [below right of=d, xshift=10mm, label={above right:{$t$}}] {$f$};

    \path[->]   (a) edge[line width=1.3pt] node[above left] {$2$} (b)
                (a) edge node[below left] {$8$} (c)
                (b) edge node[right] {$2$} (c)
                (b) edge node[above] {$7$} (d)
                (b) edge node[above right] {$3$} (e)
                (c) edge node[below] {$6$} (e)
                (d) edge node[right] {$2$} (e)
                (d) edge node[above right] {$3$} (f)
                (e) edge[line width=1.3pt] node[below right] {$4$} (f);

    \node[] (0) at (a) [shift={(-5mm, 10mm)}] {};
    \draw[-, dashed] (e)+(20mm, 5mm) -- (0);

    \node[] (sets) [right of=f, shift={(5mm, 9mm)}] {$\begin{array}[t]{l}
        S=\{a,c,e\}\\
        T=\{d,d,f\}\\
        \\
        C(S,T)=6\\
        F_{f^*}(S,T)\leq 6
    \end{array}$};
\end{graph}
\caption{Limite superiore al \emph{flusso massimo}}
\end{figure}

\begin{proof}[Dimostrazione]
    Dobbiamo dimostrare che nessun \emph{flusso netto} supera la \emph{capacità}
    del \emph{taglio} attraversato. Ovvero, dobbiamo verificare la seguente
    disuguaglianza:
    \[F_f(S,T)\leq C(S,T)\quad\forall f\wedge\forall (S,T)\]
    Per la proprietà del \emph{vincolo sulla capacità} vale quanto segue:
    \[F_f(S,T)=\sum_{x\in S,y\in T}f((x,y))\leq\sum_{x\in S,y\in T}C(x,y)=C(S,T)\]
    Inoltre, il \emph{flusso} che attraversa un \emph{taglio} è uguale al
    \emph{valore del flusso}:
    \[|f|=F_f(S,T)\quad\forall f\wedge\forall (S,T)\]
    quindi il \emph{valore del flusso} è limitato superiormente dalla capacità
    di tutti i possibili \emph{tagli}:
    \[|f|\leq C(S,T)\quad\forall f\wedge\forall (S,T)\]
\end{proof}

\begin{definition}[Teorema del taglio minimo-flusso massimo]
    Le seguenti tre affermazioni sono equivalenti:
    \begin{enumerate}
        \item $f$ è un flusso massimo;
        \item Non esiste alcun cammino aumentante nella rete $G_f$;
        \item Esiste una taglio minimo $(S,T)$ tale per cui $C(S,T)=|f|$;
    \end{enumerate}
\end{definition}

\begin{proof}[Dimostrazione]
    Procediamo dimostrando circolarmente che ogni punto implica gli altri.
    
    \paragraph{\bm{$(1)\Rightarrow(2)$}}
    Se esistesse un \emph{cammino aumentante}, il \emph{flusso} potrebbe essere
    aumentato e quindi non sarebbe massimo, generando un assurdo.

    \paragraph{\bm{$(2)\Rightarrow(3)$}}
    Poiché non esiste alcun \emph{cammino aumentante} nella \emph{rete residua}
    $G_f$, non esiste un \emph{cammino} da $s$ a $t$. Se $S$ è l'insieme dei
    \emph{nodi} raggiungibili da $s$ e $T=V-S$, $(S,T)$ è un \emph{taglio}
    perché $s\in S$ e $t\in T$, ma siccome $t$ non è raggiungibile da $s$,
    tutti gli \emph{archi} $(x,y)$ con $x\in S$ e $y\in T$ sono saturi, ovvero
    $f(x,y)=c(x,y)$. Per il \emph{\nameref{def:154}} vale:
    \[|f|=\sum_{x\in S,y\in T}f(x,y)\]
    Ne segue che:
    \[|f|=\sum_{x\in S,y\in T}f(x,y)=\sum_{x\in S, y\in T}c(S,T)=C(S,T)\]
    Poiché $|f|=C(S,T)$ e grazie al \emph{\nameref{def:155}} sappiamo che
    $|f|\leq C(S',T')$ per ogni \emph{taglio} $(S',T')$, possiamo affermare che
    $(S,T)$ è un \emph{taglio} di \emph{capacità} minima.

    \paragraph{\bm{$(3)\Rightarrow(1)$}}
    Poiché per ogni \emph{flusso} $f$ e ogni \emph{taglio} $(S,T)$ vale la
    relazione $|f|\leq C(S,T)$, il \emph{flusso} che soddisfa $|f|=C(S,T)$
    deve essere massimo.
\end{proof}

\subsection{Complessità}
\paragraph{Limite superiore nell'Algoritmo di Ford-Fulkerson}
Nell'\emph{Algoritmo di Ford-Fulkerson} la ricerca del \emph{flusso massimo}
parte da un \emph{flusso nullo} che viene di volta in volta aumentato fino ad
arrivare al \emph{flusso massimo}. Poiché ogni incremento può essere unitario,
il numero di incrementi è $O(|f^*|)$. Ogni incremento richiede una \emph{visita
del grafo}, la \emph{somma tra due flussi} e la determinazione della \emph{rete
residua}, e ognuna di queste operazioni costa $O(V+E)$, quindi la
\emph{complessità} finale del caso peggiore è $O((V+E)|f^*|)$.

\begin{note}
    Dato che $E=\Omega(V)=O(V^2)$, possiamo esprimere la \emph{complessità} come
    $O(E|f^*|)$ o $O(V^2|f^*|)$.
\end{note}

\paragraph{Limite superiore nell'Algoritmo di Edmonds-Karp}
Edmonds e Karp suggerirono di usare una \emph{visita in ampiezza} invece di una
\emph{visita generica}. Gli \emph{aumenti di flusso} vengono eseguiti $O(VE)$
volte e per ognuno di essi si paga il costo $O(V+E)$ di una \emph{visita in
ampiezza}. La \emph{complessità} finale è quindi $O(VE(V+E))$, ma siccome
$E=\Omega(V)$ perché ogni \emph{nodo}, ad eccezione di $s$ e $t$, ha almeno
un \emph{arco entrante} e un \emph{arco uscente}, possiamo semplificare
scrivendo che la \emph{complessità} è $O(VE^2)$ o $O(V^5)$.

\paragraph{Dimostrazione}
\begin{definition}[Monotonia]
    Siano $\delta_f(s,x)$ la distanza minima da $s$ a $x$ in una rete residua
    $G_f$ e $f'=f+g$ una flusso nella rete iniziale con $g$ non nullo e
    derivante da un cammino aumentante. Allora, $\delta_{f'}(s,x)\geq\delta_f(s,x)$.
\end{definition}
\begin{proof}[Dimostrazione]
    Questo comportamento è facile da dimostrare in quanto ad ogni aumento di
    \emph{flusso} alcuni \emph{archi} si \q{spengono} perché la loro
    \emph{capacità residua} scende a $0$. Poiché quegli \emph{archi} erano
    utilizzati nei \emph{cammini minimi}, \emph{visite} successive non potranno
    individuare \emph{cammini} più corti.
\end{proof}

\noindent
Dimostrata la \emph{monotonia}, possiamo dimostrare formalmente che il numero
di aumenti di \emph{flusso} è limitato superiormente da $O(VE)$.

\begin{proof}[Dimostrazione]
    Siano $G_f$ una \emph{rete residua} e $C$ un \emph{cammino aumentante} di
    $G_f$. $(x,y)$ è un \emph{arco critico}, o un \emph{collo di bottiglia},
    in $C$ se:
    \[c_f(x,y)=\min_{(u,v)\in C}\{c_f(u,v)\}\]
    Ogni \emph{cammino} contiene almeno un \emph{arco critico} e ogni volta che
    il \emph{flusso} associato a $C$ viene aggiunto, tutti gli \emph{archi
    critici} scompaiono dalla \emph{rete residua}.

    \noindent
    Poiché i \emph{cammini aumentanti} sono \emph{minimi}, abbiamo che:
    \[\delta_f(s,y)=\delta_f(s,x)+1\]
    Una volta che l'\emph{arco} $(x,y)$ sarà stato \emph{saturato}, potrà essere
    riutilizzato soltanto se il \emph{flusso} attraverso quell'\emph{arco}
    diminuisce. Ovvero, solo se verrà utilizzato l'\emph{arco} $(y,x)$.

    Se $g$ è il \emph{flusso} quando ciò accade, vale:
    \[\delta_g(s,y)=\delta_g(s,x)+1\]
    Per la \emph{monotonia} sappiamo anche che $\delta_g(s,y)\geq\delta_f(s,y)$,
    quindi vale quanto segue:
    \[\delta_g(s,y)\begin{array}[t]{cl}
        = & \delta_g(s,x)+1\\
        \geq & \delta_f(s,y)+1\\
        = & \delta_f(s,x)+1+1
    \end{array}\]
    Questo significa che dal momento in cui un \emph{arco critico} viene
    \emph{saturato} a quando torna \emph{critico}, il \emph{cammino minimo}
    passante per esso si è allungato di almeno due passi. Di conseguenza, la
    lunghezza \emph{massima} del \emph{cammino} fino a $x$, considerando che
    poi si deve attraversare l'\emph{arco} $(x,y)$ è al massimo $V-2$. Ne
    consegue che un \emph{arco} può diventare \emph{critico} un massimo di
    $\frac{V-2}{2}=\frac{V}{2}-1$ volte e, poiché ci sono $O(E)$ \emph{archi}
    che possono diventare \emph{critici} $O(V)$ volte, il numero massimo di
    \emph{flussi aumentanti} è $O(VE)$.
\end{proof}

\paragraph{Riassunto delle complessità}
\mbox{}\\
\begin{table}[h!]
\centering
\renewcommand{\arraystretch}{1.2}
\begin{tabular}{|c|c|p{0.285\textwidth}|}
    \hline
    \textbf{Nome} & \textbf{Complessità} & \textbf{Note}\\
    \hline
    Ford-Fulkerson & $O(E|f^*|)$ & Converge con valori razionali\\
    \hline
    Edmonds-Karp & $O(VE^2)$ & Specializzazione basata su BFS\\
    \hline
    Dinic, blocking flow & $O(V^2E)$ & In alcune reti particolari
    $O(\min(V^{2/3},E^{1/2})E)$\\
    \hline
    MPN & $O(V^3)$ & Solo su DAG\\
    \hline
    Dinic & $O(VE\log V)$ & Usa Dynamic tree\\
    \hline
    Goldberg e Rao & $O(\min(V^{2/3}E^{1/2})E\log(\frac{V^2}{E}+2)\log C)$ &
    $C=\max_{(x,y)\in E}c(x,y)$\\
    \hline
    \makecell[t]{Orlin, King\\Rao, Tarjan} & $O(VE)$ & Pubblicato nel 2013\\
    \hline
\end{tabular}
\end{table}