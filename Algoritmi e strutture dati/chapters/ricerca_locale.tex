\chapter{Ricerca locale}
\section{Introduzione}
La \emph{ricerca locale} è una tecnica risolutiva che, partendo da una soluzione
ammissibile qualunque, ricerca una soluzione migliore nelle \q{vicinanze} di
quella nota. Questo processo continua fino ad arrivare a una soluzione localmente
ottima. La particolarità di questa tecnica è che la soluzione trovata potrebbe
non essere la migliore possibile.

\begin{figure}[h!]
\centering
\begin{tikzpicture}
\begin{axis}[xmin = 0, xmax = 12, ymin = -1.5, ymax = 2.0,
        axis lines = left,
        xlabel = $x$,
        ylabel = {\emph{Costo}},
        ylabel style={rotate=-90},
        every axis x label/.style={
            at={(ticklabel* cs:1)},
            anchor=west,
        },
        every axis y label/.style={
            at={(ticklabel* cs:1)},
            anchor=south,
        },
        ticks=none
    ]
    \addplot[
        domain=0:12,
        samples = 200,
        smooth
    ] {exp(-x/10)*( cos(deg(x)) + sin(deg(x))/10 )};

    \node[label={below:{$A$}}, circle, fill, inner sep=1.51pt]
        at (axis cs:pi,{-exp(-pi/10)}) {};
    \node[label={below:{$B$}}, circle, fill, inner sep=1.5pt]
        at (axis cs:{3*pi},{-exp(-3*pi/10)}) {};
\end{axis}
\end{tikzpicture}
\caption{Possibili soluzioni locali}
\end{figure}

\noindent
Nel grafico di cui sopra, se la ricerca partisse da una soluzione \q{vicina} a
$B$, un algoritmo basato su \emph{ricerca locale} potrebbe convergere sul punto
$B$, sebbene esso non rappresenti una soluzione ottima globale.

\begin{minicode}{Implementazione generica di un algoritmo basato su ricerca locale}
\ind\bc{SOL} ricercaLocale()\\
    \bc{SOL} sol = \{ Una qualsiasi soluzione ammissibile \}\\
    \indf while ($\exists$S $\in$ I(Sol) migliore di sol) do\\
        sol = S\\
    \indf return sol
\end{minicode}

\section{Problema della ricerca del flusso massimo}
\begin{definition}[Rete di flusso]
    Una rete di flusso $G=(V,E,s,t,c)$ è data da un grafo orientato $G=(V,E)$,
    da due nodi $s,t\in V$, detti rispettivamente sorgente e pozzo, e da una
    funzione di capacità $c:V\times V\to\mathbb{R}^+$ tale per cui $c(x,y)=0$
    $\forall (x,y)\notin E$.
\end{definition}
\begin{note}
    Assumiamo che per ogni \emph{nodo} $x\in V$ esista un \emph{cammino} da $s$
    in $t$ passante per esso, e ignoriamo qualsiasi \emph{nodo} che non goda di
    questa proprietà.
\end{note}
\begin{definition}[Flusso]
    Un flusso in $G$ è una funzione $f:V\times V\to\mathbb{R}$ che soddisfa le
    seguenti proprietà:
    \begin{enumerate}
        \item Vincolo sulla capacità: $f(x,y)\leq c(x,y)$ $\forall x,y\in V$;
        \item Antisimmetria: $f(x,y)=-f(x,y)$ $\forall x,y\in V$;
        \item Conservazione del flusso: $sum_{y\in V}f(x,y)=0$ $\forall x\in V-\{s,t\}$;
    \end{enumerate}
\end{definition}

\noindent
La prima proprietà stabilisce che un \emph{flusso} non può mai eccedere la
capacità di un \emph{arco}. In particolare, se la capacità di un \emph{arco}
tra due \emph{nodi} $A$ e $B$ è 7, e $f(A,B)=5$, nei \emph{grafi} useremo la
seguente notazione:

\begin{figure}[h!]
    \centering
    \begin{graph}
        \node[main] (a) {$A$};
        \node[main] (b) [right of=a, xshift=15mm] {$B$};

        \path[->]   (a) edge node[above] {$5/7$} (b);
    \end{graph}
    \caption{\emph{Vincolo sulla capacità}}
\end{figure}

\noindent
La proprietà di \emph{antisimmetria} ci dice invece che il \emph{flusso}
che attraversa un \emph{arco} in direzione $(x,y)$ è l'opposto del \emph{flusso}
che attraverso lo stesso \emph{arco} in direzione $(y,x)$.

\begin{figure}[h!]
    \centering
    \begin{graph}
        \node[main] (a) {$A$};
        \node[main] (b) [right of=a, xshift=15mm] {$B$};

        \path[->]   (a) edge[bend left=25] node[above] {$5$} (b)
                    (b) edge[bend left=25, dashed] node[above] {$-5$} (a);
    \end{graph}
    \caption{\emph{Antisimmetria}}
\end{figure}

\noindent
Infine, la proprietà di \emph{conservazione del flusso} ci dice che per ogni
\emph{nodo} diverso dalla \emph{sorgente} e dal \emph{pozzo}, la somma dei
\emph{flussi} entranti ed uscenti è sempre pari a $0$.

\begin{figure}[h!]
    \centering
    \begin{graph}
        \node[main] (a) {$A$};
        \node[main] (c) [below right of=a, xshift=10mm] {$C$};
        \node[main] (b) [below left of=c, xshift=-10mm] {$B$};
        \node[main] (d) [above right of=c, xshift=10mm] {$D$};
        \node[main] (e) [below right of=c, xshift=10mm] {$E$};

        \path[->]   (a) edge[bend left=25] node[above] {$2$} (c)
                    (c) edge[bend left=25, dashed] node[above] {$-2$} (a)
                    (c) edge[bend left=25] node[above] {$1$} (d)
                    (d) edge[bend left=25, dashed] node[above] {$-1 $} (c)
                    (c) edge[bend left=25] node[above] {$4$} (e)
                    (e) edge[bend left=25, dashed] node[above] {$-4$} (c)
                    (b) edge[bend left=25] node[above] {$3$} (c)
                    (c) edge[bend left=25, dashed] node[above] {$-3$} (b);
    \end{graph}
    \caption{\emph{Conservazione del flusso}}
\end{figure}

\begin{definition}[Valore del flusso]
    Il valore di un flusso $f$ è definito come:
    \[|f|=\sum_{(s,x)\in E}f(s,x)\]
    ovvero come la quantità di flusso uscente da $s$.
\end{definition}

\noindent
Date tutte queste definizioni possiamo finalmente enunciare il problema in
questione.

\begin{problem}[Problema della ricerca del flusso massimo]
    Data una rete $G=(V,E,s,t,c)$, trovare un flusso che abbia valore massimo
    fra tutti i flussi associabili alla rete. In particolare, ricercare il flusso
    $f^*$ tale per cui $|f^*|=\max\{|f|\}$.
\end{problem}

\subsection{Metodo delle reti residue}
L'algoritmo che intendiamo realizzare è detto \emph{metodo delle reti residue}
perché è basato su un approccio additivo. In particolare, partendo da un
\emph{flusso} $f$ inizialmente nullo, si ripetono le seguenti operazioni:
\begin{enumerate}
    \item \q{Sottrazione} del \emph{flusso} $f$ alla rete attuale, ottenendo
    una \emph{rete residua};
    \item Ricerca di un secondo \emph{flusso} $g$ all'interno della \emph{rete
    residua};
    \item Somma di $g$ ad $f$;
\end{enumerate}
fino a quando $g$ non diventa un \emph{flusso nullo}.

\begin{note}
    Questo particolare algoritmo permette sempre di ottenere il \emph{flusso
    massimo}, quindi non soffre del problema al quale avevamo accennato
    inizialmente.
\end{note}

\noindent
Diamo ora delle altre definizioni:
\begin{definition}[Flusso nullo]
    È definito flusso nullo la funzione $f_0:V\times V\to\mathbb{R^+}$ tale che:
    \[f(x,y)=0\qquad\forall x,y\in V\]
\end{definition}

\begin{figure}[h!]
    \centering
    \begin{graph}
        \node[main] (a) [label=above left:{$s$}] {$A$};
        \node[main] (b) [above right of=a, xshift=10mm] {$B$};
        \node[main] (c) [below right of=a, xshift=10mm] {$C$};
        \node[main] (d) [below right of=b, xshift=10mm,
            label=above right:{$t$}] {$D$};

        \node[] (e) [left of=a, xshift=-10mm] {$|f|=0$};

        \path[->]   (a) edge node[above left] {$0/2$} (b)
                    (b) edge node[above right] {$0/3$} (d)
                    (a) edge node[below left] {$0/2$} (c)
                    (c) edge node[right] {$0/1$} (b)
                    (c) edge node[below right] {$0/1$} (d);
    \end{graph}
    \caption{\emph{Flusso nullo}}
\end{figure}

\newpage
\begin{definition}[Somma di flussi]
    Per ogni coppia di flussi $f_1$ e $f_2$ in $G$, definiamo il flusso somma
    $g=f_1+f_2$ come un flusso tale che:
    \[g(x,y)=f_1(x,y)+f_2(x,y)\qquad\forall x,y\in V\]
\end{definition}

\begin{figure}[h!]
    \centering
    \begin{graph}
        \node[main] (a1) [label=above left:{$s$}] {$A$};
        \node[main] (b1) [above right of=a1, xshift=10mm] {$B$};
        \node[main] (c1) [below right of=a1, xshift=10mm] {$C$};
        \node[main] (d1) [below right of=b1, xshift=10mm,
            label=above right:{$t$}] {$D$};

        \path[->]   (a1) edge node[above left] {$1/2$} (b1)
                    (b1) edge node[above right] {$1/3$} (d1)
                    (a1) edge node[below left] {$1/2$} (c1)
                    (c1) edge node[right] {$0/1$} (b1)
                    (c1) edge node[below right] {$1/1$} (d1);

        \node[] (plus) [below of=c1, yshift=10mm] {$+$};

        \node[main] (b2) [below of=plus, yshift=10mm] {$B$};
        \node[main] (a2) [below left of=b2, xshift=-10mm,
            label=above left:{$s$}] {$A$};
        \node[main] (c2) [below right of=a2, xshift=10mm] {$C$};
        \node[main] (d2) [below right of=b2, xshift=10mm,
            label=above right:{$t$}] {$D$};

        \path[->]   (a2) edge node[above left] {$1/2$} (b2)
                    (b2) edge node[above right] {$2/3$} (d2)
                    (a2) edge node[below left] {$1/2$} (c2)
                    (c2) edge node[right] {$1/1$} (b2)
                    (c2) edge node[below right] {$0/1$} (d2);

        \node[] (equals) [right of=plus, xshift=20mm] {$=$};

        \node[main] (a3) [right of=equals, label=above left:{$s$}] {$A$};
        \node[main] (b3) [above right of=a3, xshift=10mm] {$B$};
        \node[main] (c3) [below right of=a3, xshift=10mm] {$C$};
        \node[main] (d3) [below right of=b3, xshift=10mm,
            label=above right:{$t$}] {$D$};

        \path[->]   (a3) edge node[above left] {$2/2$} (b3)
                    (b3) edge node[above right] {$3/3$} (d3)
                    (a3) edge node[below left] {$1/2$} (c3)
                    (c3) edge node[right] {$1/1$} (b3)
                    (c3) edge node[below right] {$1/1$} (d3);
    \end{graph}
    \caption{\emph{Somma di flussi}}
\end{figure}

\begin{definition}[Capacità residua]
    È definita capacità residua di un flusso $f$ in una rete $G=(V,E,s,t,c)$ una
    funziona $c_f:V\times V\to\mathbb{R^+}$ tale che:
    \[c_f(x,y)=c(x,y)-f(x,y)\qquad\forall x,y\in V\]
\end{definition}

\begin{figure}[h!]
    \centering
    \begin{graph}
        \node[main] (a) {$A$};
        \node[main] (b) [right of=a, xshift=15mm] {$B$};

        \path[->]   (a) edge[bend left=25] node[above] {$5+\underline{2}/7$} (b)
                    (b) edge[bend left=25, dashed] node[above] {$-5+\underline{5}/0$} (a);
    \end{graph}
    \caption{\emph{Capacità residua}}
\end{figure}

\noindent
Possiamo notare che gli \emph{archi all'indietro} che avevamo visto enunciando
la proprietà di \emph{antisimmetria} dei \emph{flussi}, si originano dalla
definizione di \emph{capacità residua}. Infatti, poiché l'\emph{arco} $(B,A)$
non esiste, $c(B,A)=0$ e quindi, vale la seguente catena di uguaglianze:
\[c_f(B,A)\begin{array}[t]{cll}
    = & c(B,A)-f(B,A) & \quad\text{Dalla definizione di \emph{capacità residua}}\\
    = & c(B,A)-(-f(A,B)) & \quad\text{Per la proprietà di \hyperref[def:144]
        {\emph{antisimmetria}}}\\
    = & 0 - (-5) = 5
\end{array}\]

\newpage
\begin{definition}[Rete residua]
    Data una rete di flusso $G=(V,E,s,t,c)$ e un flusso $f$ su $G$, una rete
    $G_f=(V,E_f,s,t,c_f)$ è una rete residua di $G$ rispetto ad $f$, se è tale
    per cui:
    \[c_f(x,y)>0\qquad\forall x,y\in V\::\:(x,y)\in E_f\]
\end{definition}

\begin{figure}[h!]
\centering
\begin{graph}
    \node[main] (a) [label=above left:{$s$}] {$A$};
    \node[main] (b) [above right of=a, xshift=40mm, yshift=10mm] {$B$};
    \node[main] (c) [below right of=a, xshift=40mm, yshift=-10mm] {$C$};
    \node[main] (d) [below right of=b, xshift=40mm, yshift=-10mm,
        label=above right:{$t$}] {$D$};

    \path[->]   (a) edge[bend left=25, dotted] node[above left]
                    {$2+\underline{0}/2$} (b)
                (b) edge[bend left=25, dashed] node[above left]
                    {$-2+\underline{2}/0$} (a)
                (b) edge[bend left=25] node[above right]
                    {$2+\underline{1}/3$} (d)
                (d) edge[bend left=25, dashed] node[above right]
                    {$-2+\underline{2}/0$} (b)
                (a) edge[bend right=25] node[below left]
                    {$1+\underline{1}/2$} (c)
                (c) edge[bend right=25, dashed] node[below left]
                    {$-1+\underline{1}/0$} (a)
                (c) edge[bend left=25] node[left]
                    {$0+\underline{1}/1$} (b)
                (b) edge[bend left=25, dashed] node[right]
                    {$-0+\underline{0}/0$} (c)
                (c) edge[bend right=25, dotted] node[below right]
                    {$1+\underline{0}/1$} (d)
                (d) edge[bend right=25, dashed] node[below right]
                    {$-1+\underline{1}/0$} (c);
\end{graph}
\caption{Rete residua}
\end{figure}

\noindent
Gli \emph{archi} $(A,B)$ e $(C,D)$ sono stati saturati, ovvero $f(A,B)=c(A,B)$
e $f(C,D)=c(C,D)$. Di conseguenza, la loro \emph{capacità residua} è $0$ e
quindi non fanno parte della \emph{rete residua}.

\begin{minicode}{Schema generale dell'algoritmo}
\ind\bc{int}[][] maxFlow(\bc{GRAPH} G, \bc{NODE} s, \bc{NODE} t, \bc{int}[][] c)\\
    \bc{FLOW} f = f$_0$\hfill\com{Inizializza $f$ a un \emph{flusso nullo}}
    \bc{int}[][] c$_f$ = c\hfill\com{Inizialmente la \emph{capacità residua} è
    pari a quella iniziale}
    \bc{FLOW} g\\
    \indf do\\
        g = \{ trova un \emph{flusso} in c$_f$ tale che $|g|>0$, altrimenti f$_0$ \}\\
        f = f + g\\
        c$_f$ = \{ \emph{capacità} della \emph{rete residua} di $G$ rispetto ad $f$ \}\\
    \indf while (g $\neq$ f$_0$)\\
    \indf return f
\end{minicode}

\paragraph{Dimostrazione}
Per dimostrare la correttezza dell'algoritmo dobbiamo dimostrare il seguente
lemma:

\begin{definition}[Lemma della somma di flussi]
    Se $f$ è un flusso in $G$ e $g$ è un flusso in $G_f$, allora anche $f+g$ è
    un flusso in $G$.
\end{definition}
\begin{proof}[Dimostrazione]
    Dobbiamo dimostrare che $f+g$ rispetta tutte e tre le proprietà dei
    \emph{flussi}.

    \paragraph{Vincolo sulla capacità}
    Per ogni coppia di \emph{nodi} $x,y\in V$ vale la seguente catena di
    disuguaglianze:
    \[\begin{array}{rcll}
        g(x,y) & \leq & c_f(x,y) & \quad\text{È vero perché $g$ è un
            \emph{flusso} in $G_f$}\\
        f(x,y)+g(x,y) & \leq & c_f(x,y)+f(x,y)\\
        (f+g)(x,y) &  \leq & c(x,y)-f(x,y)+f(x,y) & \quad\text{Definizione di
            \hyperref[def:148]{\emph{capacità residua}}}\\
        (f+g)(x,y) & \leq & c(x,y) & \quad\text{Semplificazione algebrica}
    \end{array}\]

    \paragraph{Antisimmetria}
    Per ogni coppia di \emph{nodi} $x,y\in V$ vale la seguente catena di
    uguaglianze:
    \[\begin{array}{rcll}
        f(x,y)+g(x,y) & = & -f(y,x)-g(x,y)\\
        f(x,y)+g(x,y) & = & -(f(x,y)+g(x,y))\\
        (f+g)(x,y) & = & -(f+g)(x,y)
    \end{array}\]

    \paragraph{Conservazione del flusso}
    Per ogni coppia di \emph{nodi} $x,y\in V$ vale la seguente catena di
    uguaglianze:
    \[\sum_{y\in V}(f+g)(x,y)\begin{array}[t]{cl}
        = & \sum_{y\in V}(f(x,y)+g(x,y))\\
        = & \sum_{y\in V}f(x,y)+\sum_{y\in V}g(x,y)\\
        = & 0 + 0\\
        = & 0
    \end{array}\]
\end{proof}

\subsection{Metodo dei cammini aumentanti}
Quindi l'algoritmo che abbiamo descritto è corretto, tuttavia rimane un problema:
come facciamo a trovare il \emph{flusso} $g$?

Una soluzione è quella proposta da Ford e Fulkerson nel 1956 che utilizza i
cosiddetti \emph{cammini aumentanti}. In particolare, viene identificato nella
\emph{rete residua} $G_f$ un \emph{cammino} $C=(v_0,\dots,v_n)$ tale per cui
$v_0=s$ e $v_n=t$. Quindi, si definisce la \emph{capacità} del \emph{cammino}
come la minore tra le \emph{capacità} degli \emph{archi} attraversati, ovvero:
\[c_f(C)=\min_{i\in\{2,\dots,n\}}c_f(v_{i-1},v_i)\]

\begin{note}
    La \emph{capacità} di un \emph{cammino} è definibile anche come la
    \emph{capacità} del suo \q{collo di bottiglia}.
\end{note}

\noindent
Infine, il \emph{flusso} addizionale $g$ è definito in modo che $g(v_{i-1},v_i)=
c_f(C)$, $g(v_i,v_{i-1})=-c_f(C)$ per la proprietà \emph{antisimmetrica} e
$g(x,y)=0$ per tutte le coppie $x,y\notin C$.

\begin{figure}[h!]
\centering
\resizebox*{0.48\textwidth}{!}{\subfloat[\emph{Grafo} iniziale]{\begin{graph}
    \node[main] (a) [label=above left:{$s$}] {$a$};
    \node[main] (b) [above right of=a, xshift=10mm] {$b$};
    \node[main] (c) [below right of=a, xshift=10mm] {$c$};
    \node[main] (d) [right of=b, xshift=10mm] {$d$};
    \node[main] (e) [right of=c, xshift=10mm] {$e$};
    \node[main] (f) [below right of=d, xshift=10mm, label={above right:{$t$}}] {$f$};

    \path[->]   (a) edge node[above left] {$2$} (b)
                (a) edge node[below left] {$8$} (c)
                (b) edge node[right] {$2$} (c)
                (b) edge node[above] {$7$} (d)
                (b) edge node[above right] {$3$} (e)
                (c) edge node[below] {$6$} (e)
                (d) edge node[right] {$2$} (e)
                (d) edge node[above right] {$3$} (f)
                (e) edge node[below right] {$4$} (f);
\end{graph}}}\hfill
\resizebox*{0.48\textwidth}{!}{\subfloat[Scelta di un \emph{cammino}]{\begin{graph}
    \node[main] (a) [label=above left:{$s$}] {$a$};
    \node[main] (b) [above right of=a, xshift=10mm] {$b$};
    \node[main] (c) [below right of=a, xshift=10mm] {$c$};
    \node[main] (d) [right of=b, xshift=10mm] {$d$};
    \node[main] (e) [right of=c, xshift=10mm] {$e$};
    \node[main] (f) [below right of=d, xshift=10mm, label={above right:{$t$}}] {$f$};

    \path[->]   (a) edge[line width=1.3pt] node[above left] {$2$} (b)
                (a) edge node[below left] {$8$} (c)
                (b) edge node[right] {$2$} (c)
                (b) edge node[above] {$7$} (d)
                (b) edge[line width=1.3pt] node[above right] {$3$} (e)
                (c) edge node[below] {$6$} (e)
                (d) edge node[right] {$2$} (e)
                (d) edge node[above right] {$3$} (f)
                (e) edge[line width=1.3pt] node[below right] {$4$} (f);
\end{graph}}}\\
\resizebox*{0.48\textwidth}{!}{\subfloat[Definizione del \emph{flusso} $g$]{\begin{graph}
    \node[main] (a) [label=above left:{$s$}] {$a$};
    \node[main] (b) [above right of=a, xshift=10mm] {$b$};
    \node[main] (c) [below right of=a, xshift=10mm] {$c$};
    \node[main] (d) [right of=b, xshift=10mm] {$d$};
    \node[main] (e) [right of=c, xshift=10mm] {$e$};
    \node[main] (f) [below right of=d, xshift=10mm, label={above right:{$t$}}] {$f$};

    \path[->]   (a) edge[bend left=25] node[above left] {$2/2$} (b)
                (b) edge[bend left=25, dashed] node[above, xshift=-2mm] {$-2/0$} (a)
                (a) edge node[below left] {$8$} (c)
                (b) edge node[right] {$2$} (c)
                (b) edge node[above] {$7$} (d)
                (b) edge[bend left=25] node[above right, yshift=-2mm] {$2/3$} (e)
                (e) edge[bend left=25, dashed] node[above, xshift=3mm] {$-2/3$} (b)
                (c) edge node[below] {$6$} (e)
                (d) edge node[right] {$2$} (e)
                (d) edge node[above right] {$3$} (f)
                (e) edge[bend right=25] node[below right] {$2/4$} (f)
                (f) edge[bend right=25, dashed] node[below, xshift=2mm] {$-2/4$} (e);
\end{graph}}}
\caption{\emph{Cammini aumentanti} in generale}
\end{figure}

\begin{minicode}{Schema generale dell'algoritmo}
\ind\bc{int}[][] maxFlow(\bc{GRAPH} G, \bc{NODE} s, \bc{NODE} t, \bc{int}[][] c)\\
    \bc{int}[][] f = new \bc{int}[1\dots G.size()][1\dots G.size()]\hfill\com{\emph{Flusso} parziale}
    \bc{int}[][] g = new \bc{int}[1\dots G.size()][1\dots G.size()]\hfill
        \com{\emph{Flusso}\,da\;\emph{cammino\,aumentante}}
    \bc{int}[][] c$_f$ = new \bc{int}[1\dots G.size()][1\dots G.size()]\hfill\com{\emph{Capacità residua}}
    \indf foreach (x,y $\in$ G.V()) do\\
        f[x][y] = 0\hfill\com{Inizializza un \emph{flusso nullo}}
        c$_f$[x][y] = c[x][y]\\
    \indf do\\
        g = \{ \emph{flusso} associato ad un \emph{cammino aumentante} in $c_f$, oppure $f_0$ \}\\
        \indff foreach (x,y $\in$ G.V()) do\\
            f[x][y] = f[x][y] + g[x][y]\hfill\com{\emph{Somma dei flussi}:\:$f=f+g$}
            c$_f$ = c[x][y] - f[x][y]\hfill\com{Aggiorna la \emph{capacità residua}}
    \indf while (g $\neq$ f$_0$)\\
    \indf return f
\end{minicode}