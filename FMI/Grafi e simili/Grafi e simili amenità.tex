\documentclass[12pt, a4paper]{report}

\usepackage[utf8]{inputenc}
\usepackage{geometry}
 \geometry{
 a4paper,
 total={170mm,257mm},
 left=20mm,
 top=20mm,
}

\usepackage{titlesec}
\titleformat
{\chapter}
[display]{\bfseries\Large\itshape}
{Capitolo Nr.\thechapter}
{0.5ex}
{
    \rule{\textwidth}{1pt}
    \vspace{1ex}
	\centering
}
[\vspace{-0.5ex}\rule{\textwidth}{0.3pt}]

\renewcommand{\contentsname}{Indice}
\newcommand{\N}{\mathbb{N}}

\usepackage{amsthm} % Fornisce il comando \newtheorem
\usepackage{thmtools} % fornisce il comando \declaretheorem
\usepackage{amsmath}
\usepackage{amsfonts}
\usepackage{amssymb}
\usepackage{enumitem} % Permette di usare la numerazione romana nelle liste
\usepackage{latexsym}
\usepackage{mathtools}
\usepackage{dsfont}

\theoremstyle{definition}
\newtheorem{definition}{Definizione}[section]
\newtheorem{theorem}{Teorema}[section]
\newtheorem{corollary}{Corollario}[section]
\newtheorem{lemma}{Lemma}[section]
\newtheorem{observation}{Oss}[section]
\declaretheorem[name=Dim, qed=$\blacksquare$, numbered=no]{demonstration}
\newtheorem*{proposition}{Prop}
\newtheorem*{property}{Proprietà}
\newtheorem*{note}{NB}

\title{Grafi e simili amenità}
\author{Leonardo De Faveri}
\date{}

\begin{document}
\maketitle
\tableofcontents

\chapter{Definizioni generali}
\section{Grafi e sottografi}
Dato un insieme $V$, indichiamo con $\binom{V}{2}$, l'insieme i cui elementi sono
tutti i sottoinsiemi di $V$ composti da 2 elementi, ovvero:
\[\binom{V}{2}\coloneqq\{A\in 2^V\ :\ |A|=2\}\]

Vale inoltre, la formula seguente:
\[\left|\binom{V}{2}\right|=\begin{cases}
    \binom{|V|}{2}=\frac{|V|}{2!(|V|-2)!}=\frac{|V|(|V|-1)}{2}
    & \text{se } |V|\geq 2\\
    0 & \text{se } |V|<2
\end{cases}\]

\begin{definition}(Concetto di grafo)
    Un \emph{grafo} $G$ è una coppia $(V,E)$, dove $V$ è un insieme non vuoto
    detto \emph{insieme dei vertici}, mentre $E\subseteq\binom{V}{2}$ ed è detto
    \emph{insieme dei lati} di $G$.

    Se $G=(V,E)$ è un \emph{grafo} ed $e=\{v_1, v_2\}\in E$, cioè è un lato di $G$,
    allora diciamo che $v_1$ e $v_2$ sono gli \emph{estremi} di $e$ e anche che
    $e$ \emph{congiunge} $v_1$ e $v_2$.

    Se $G$ è un \emph{grafo}, allora $V(G)$ e $E(G)$ indicano rispettivamente
    l'\emph{insieme dei vertici} e l'\emph{insieme dei lati} di $G$.
\end{definition}

\begin{definition}(Concetto di sottografo)
    Siano $G=(V,E)$ e $G'=(V',E')$ due \emph{grafi}. Diciamo che $G'$ è un
    \emph{sottografo} se $V'\subset V$ e $E'\subset E$.
    
    Se $G'$ è un \emph{sottografo} di $G$ e vale:
    \[E'=\{e\in E\ |\ e=\{v_1, v_2\}\text{ con }\ v_1,v_2\in V'\}\]
    allora $G'$ si dice \emph{sottografo di $G$ indotto da $V'$} e si indica con
    il simbolo $G[V']$.
\end{definition}

\begin{definition}[Concetto di grafo finito]
    Un \emph{grafo} $G$ è detto \emph{grafo finito} se ha un numero finito di
    \emph{vertici.}
\end{definition}

\begin{observation}
    Un \emph{grafo finito} ha anche un numero finito di \emph{lati}.
\end{observation}
\begin{observation}
    Un \emph{grafo} con un numero finito di \emph{lati} non è necessariamente
    \emph{finito}.
\end{observation}

\subsection{Grado e score}

\begin{definition}
    Sia $G=(V,E)$ un \emph{grafo finito} e sia $v\in V$. Definiamo il \emph{grado}
    di $v$ in $G$ come:
    \[deg_G(v)\coloneqq|\{e\in E\ :\ v\in e\}|\]
\end{definition}

\begin{proposition}[Relazione fondamentale tra i gradi dei vertici e il numero di
    lati di un grafo finito]
    Se $G=(V,E)$ è un \emph{grafo finito}, vale la seguente:
    \[\sum_{v\in V}deg_G(v)=2|E|\]
\end{proposition}

\begin{definition}[Concetto di score]
    Sia $G=(V,E)$ un \emph{grado finito} con $n$ \emph{vertici}. Definiamo lo
    \emph{score} di $G$ come la n-upla composta dai \emph{gradi} dei \emph{vertici}
    di $G$ vista a meno di riordinamento.

    Lo \emph{score} di $G$ è detto essere in \emph{forma canonica} se è ordinato in
    modo non decrescente.
\end{definition}

\begin{observation}
    La \emph{forma canonica} dello \emph{score} di un \emph{grafo} è unica.
\end{observation}

\begin{observation}
    Siano $G$ un \emph{grafo finito} e $score(G)=(d_1,\dots,d_n)$ il suo \emph{score}.
    La conoscenza di $score(G)$ implica quella del numero di \emph{vertici} e di
    \emph{lati}. Il numero di \emph{vertici} è ricavabile dal numero di componenti
    dello \emph{score} e il numero di \emph{lati} si ricava grazie alla \emph{Relazione
    fondamentale tra gradi dei vertici e numero dei lati di un grafo finito}.
\end{observation}

\section{Morfismi e isomorfismi}
\begin{definition}(Concetto di morfismo)
    Siano $G=(V,E)$ e $G'=(V',E')$ due \emph{grafi}. Una funzione \(f:V\to V'\)
    iniettiva si dice \emph{morfismo} da $G$ in $G'$ se vale la seguente:
    \begin{equation} \label{eq:1}
        \forall v_1,v_2\in V, \{v_1,v_2\}\in E\Rightarrow\{f(v_1),f(v_2)\}\in E'
    \end{equation}
    Se $f:V\to V'$ è un \emph{morfismo} da $G$ in $G'$, scriveremo $f:G\to G'$.
\end{definition}

\begin{observation}
    Siano $G=(V,E)$ e $G'=(V',E')$ due \emph{grafi} e sia \(f:V\to V'\) una
    funzione iniettiva. Per ogni $e=\{v_1,v_2\}\in E$, allora:
    \[f(e)=\{f(v_1),f(v_2)\}\in\binom{V}{2}\]
    Se definiamo $f(E)\coloneqq\{f(e)\in\binom{V}{2}\ |\ e\in E\}$, segue che la
    condizione \ref{eq:1} è equivalente alla seguente:
    \begin{equation} \label{eq:2}
        f(E)\subset E'    
    \end{equation}
    Dunque, $f$ è un \emph{morfismo} se e soltanto se $f(E)\subset E'$. 
\end{observation}

\begin{definition}(Concetto di isomorfismo)
    Siano $G$ e $G'$ due \emph{grafi} e sia $f:V\to V'$ una funzione. Diciamo che
    $f$ è un \emph{isomorfismo} se valgono le seguenti:
    \begin{enumerate}[label=(\roman*)]
        \item $f$ è bigettiva
        \item $f$ è un \emph{morfismo} da $G$ in $G'$
        \item $f^{-1}:V(G')\to V(G)$ è un \emph{morfismo} da $G'$ in $G$
    \end{enumerate}
    Se esiste un \emph{isomorfismo} da $G$ in $G'$, allora $G$ si dice
    \emph{isomorfo} a $G'$ e si scrive $G\cong G'$.
\end{definition}

\begin{proposition}
    Siano $G$ e $G'$ due \emph{grafi} e sia $f:V\to V'$ una funzione. Diciamo che
    $f$ è un \emph{isomorfismo} tra $G$ e $G'$ se valgono le seguenti:
    \begin{enumerate}[label=(\roman*)]
        \item $f$ è \emph{bigettiva}
        \item $f(E)=E'$, ovvero $\forall e\in\binom{V}{2},e\in E\Leftrightarrow f(e)\in E'$
    \end{enumerate}
\end{proposition}

\begin{proposition}
    Se $G$ e $G'$ sono due \emph{grafi isomorfi}, allora hanno lo stesso \emph{score}.
\end{proposition}
\begin{observation}
    Non tutti i grafi con lo stesso \emph{score} sono \emph{isomorfi}.
\end{observation}

\section{Passeggiate, cammini e cicli}
\begin{definition}
    Sia $G=(V,E)$. Una successione finita e ordinata $(v_0,v_1,\dots,v_n)$ di
    \emph{vertici} di $G$ si dice:
    \begin{itemize}
        \item \emph{Passeggiata} in $G$ se $n=0$ oppure $n\geq1$ e $\{v_i,v_{i+1}\}
        \in E\ \forall i\in\{0,1,\dots,n-1\}$
        \item \emph{Cammino} in $G$ se è una \emph{passeggiata} in $G$ e $v_i\neq v_j
        \ \forall i,j\in\{0,1,\dots,n\}$ con $i\neq j$
        \item \emph{Ciclo} in $G$ se è una \emph{passeggiata} in $G$ e $n\geq 3$,
        $v_0=v_n$ e $v_i\neq v_j\ \forall i,j\in\{0,1,\dots,n-1\}$ con $i\neq j$
    \end{itemize}
    Se $P=(v_0,v_1,\dots,n)$ è una \emph{passeggiata} in $G$, allora $n$ è detta
    \emph{lunghezza della passeggiata} e scriveremo $n=l(P)$.
\end{definition}

\begin{definition}(Congiungibilità)
    Sia $G=(V,E)$ e siano $v,w\in V$. Diciamo che $v$ e $w$ sono \emph{congiungibili}
    in $G$ con \emph{passeggiata} (risp. con \emph{cammino}) se esiste una
    \emph{passeggiata} (risp. un \emph{cammino}) $(v_0,v_1,\dots,v_n)$ con $v_0=v$
    e $v_n=w$.
\end{definition}

\begin{proposition}
    Sia $G=(V,E)$ e siano $v,w\in V$. Allora $v$ e $w$ sono \emph{congiungibili}
    con \emph{cammino} se e solo se lo sono con \emph{passeggiata}.
\end{proposition}

\begin{observation}
    Dati un \emph{grafo} $G=(V,E)$ e due \emph{vertici} $v,w\in V$, diciamo che
    $v$ e $w$ sono \emph{congiungibili} in $G$ se lo sono per \emph{cammini} o
    equivalentemente per \emph{passeggiata}.
\end{observation}

\section{Connessione}

\begin{definition}(Concetto di componente connessa)
    Sia $G=(V,E)$ e sia $\sim$ la \emph{relazione di congiungibilità} su $V$. Indichiamo
    con $\{V_i\}_{i\in I}$ l'insieme di tutte le $\sim$-classi di equivalenza
    (considerate come sottoinsiemi di $V$). I \emph{sottografi} $\{G[V_i]\}_{i\in I}$
    \emph{indotti} da $G$ su $V_i$ si dicono \emph{componenti connesse} di $G$.
\end{definition}

\begin{definition}(Concetto di grafo connesso)
    Un \emph{grafo} si dice \emph{connesso} se possiede una sola \emph{componente
    connessa}, altrimenti si dice \emph{sconnesso}.
\end{definition}

\begin{observation}
    \mbox{}
    \begin{enumerate}[label=(\roman*)]
        \item Se $G$ un \emph{grafo}, allora $G$ è \emph{connesso} se e solo se
        ogni coppia di \emph{vertici} di $G$ è \emph{congiungibile} in $G$
        \item Ogni \emph{componente connessa} $G'$ di $G$ è un \emph{grafo connesso}
    \end{enumerate}
\end{observation}

\begin{proposition}
    Siano $G$ e $G'$ due \emph{grafi} e sia $f:G\to G'$ un \emph{morfismo}. Valgono
    le seguenti:
    \begin{enumerate}[label=(\roman*)]
        \item Se $v,w\in V(G)$ sono \emph{congiungibili} in $G$, allora anche $f(v)$
        e $f(w)$ lo sono in $G'$
        \item Se $f$ è un \emph{isomorfismo}, allora $v\sim w$ in $G$ se e solo se
        $f(v)\sim f(w)$ in $G'$
    \end{enumerate}
\end{proposition}

\begin{corollary}
    Siano $G$ e $G'$ due \emph{grafi isomorfi}, $\{G_i\}_{i\in I}$ le \emph{componenti
    connesse} di $G$ e $\{G'_j\}_{j\in J}$ le \emph{componenti connesse} di $G'$.
    Allora, $G$ e $G'$ hanno lo stesso numeri di \emph{componenti connesse} e tali
    componenti sono a 2 a 2 \emph{isomorfe}. Più precisamente:
    \[\exists\varphi: I\to J\text{ biettiva } t.c.\ G_i\cong G'_{\varphi(i)}\ 
    \forall i\in I\]
\end{corollary}
\begin{corollary}
    Due \emph{grafi isomorfi} sono entrambi \emph{connessi} o entrambi \emph{sconnessi}.
\end{corollary}

\subsection{Grafi 2-connessi e Hamiltoniani}
\begin{definition}
    Sia $G=(V,E)$ un \emph{grafo} con almeno due \emph{vertici} e sia $v\in V$.
    Definiamo il \emph{grafo} $G-v$, detto \emph{grafo ottenuto da $G$ cancellando
    il vertice $v$}, ponendo:
    \[V(G-v)\coloneqq V\backslash\{v\}\, E(G-v)\coloneqq\{e\in E\ :\ v\notin e\}\]
\end{definition}

\begin{definition}
    Un \emph{grafo} $G$ si dice \emph{2-connesso} se $\forall v\in V(G)$, $G-v$ è
    \emph{connesso}.
\end{definition}
\begin{lemma}
    Ogni \emph{grafo 2-connesso} è anche \emph{connesso}.
\end{lemma}

\begin{definition}
    Sia $G$ un \emph{grafo}. Un \emph{ciclo} in $G$ che attraversa tutti i vertici
    è detto essere un \emph{ciclo Hamiltoniano} in $G$. Se $G$ ammette almeno un
    \emph{ciclo Hamiltoniano}, allora $G$ è detto \emph{grafo Hamiltoniano}.
\end{definition}

\begin{observation}
    Un \emph{grafo Hamiltoniano} è finito e possiede almeno 3 \emph{vertici}.
\end{observation}
\begin{lemma}
    Un \emph{grafo Hamiltoniano} è anche \emph{2-connesso}.
\end{lemma}

\begin{lemma}
    Siano $G$ e $G'$ due \emph{grafi isomorfi}. Valgono le seguenti:
    \begin{enumerate}[label=(\roman*)]
        \item $G$ è \emph{2-connesso} se e solo se lo è anche $G'$
        \item $G$ è \emph{Hamiltoniano} se e solo se lo è anche $G'$
    \end{enumerate}
\end{lemma}

\section{Alberi}
\begin{definition}
    Sia $G=(V,E)$ un \emph{grafo} e sia $v\in V$. Diciamo che $v$ è una \emph{foglia}
    di $G$ se $deg_G(v)=1$.
\end{definition}
\begin{lemma}
    I \emph{grafi 2-connessi} e i \emph{grafi Hamiltoniani} non hanno \emph{foglie}.
\end{lemma}

\begin{definition}[Concetto di albero e foresta]
    Un \emph{grafo} si dice \emph{albero} se è \emph{connesso} e senza \emph{cicli}.
    Una \emph{foresta} è un \emph{grafo} senza \emph{cicli}.
\end{definition}

\begin{theorem}
    Sia $T=(V,E)$ un \emph{albero}. Le seguenti affermazioni sono equivalenti:
    \begin{enumerate}[label=(\roman*)]
        \item $T$ è un \emph{albero}
        \item $\forall v,v'\in V$, esiste un unico cammino in $T$ che congiunge
        $v$ e $v'$
        \item $T$ è \emph{connesso} e per ogn $e\in E$, il grafo $T-e\coloneqq(V,
        E\backslash\{e\})$ è \emph{sconnesso}
        \item $T$ non ha \emph{cicli} e per ogni $e\in\binom{V}{2}\backslash E$, il
        \emph{grafo} $T+e\coloneqq(V,E\cup\{e\})$ ha almeno un ciclo
    \end{enumerate}
\end{theorem}

\begin{lemma}
    Sia $T$ un \emph{albero finito} avente almeno 2 \emph{vertici}. Allora, $T$
    possiede almeno 2 \emph{foglie}.
\end{lemma}
\begin{observation}
    Il precedente Lemma è falso se non si assume che $T$ sia \emph{finito}.
\end{observation}

\begin{theorem}[Teorema di caratterizzazione degli alberi finiti]
    Sia $T=(V,E)$ un \emph{grafo finito}. Le seguenti affermazioni sono equivalenti:
    \begin{enumerate}[label=(\roman*)]
        \item $T$ è un \emph{albero}
        \item $T$ è \emph{connesso} e soddisfa la seguente \emph{formula di Eulero}:
        \[|V|-1=|E|\]
    \end{enumerate}
\end{theorem}
\begin{corollary}
    Sia $n\in\N\backslash\{0\}$ e sia $d=(d_1,\dots,d_n)\in\N^n$. Esiste un
    \emph{albero} con \emph{score} $d$ se e soltanto se vale:
    \[n-1=\frac{1}{2}\sum_{i=1}^nd_i\]
\end{corollary}

\subsection{Alberi di copertura}
\begin{definition}
    Sia $G$ un \emph{grafo}. Un \emph{sottografo} $T$ di $G$ si dice \emph{albero
    di copertura} di $G$ se $T$ è un \emph{albero} e $V(T)=V(G)$.
\end{definition}
\begin{observation}
    Se un \emph{grafo} $G$ ammette almeno un \emph{albero di copertura} $T$, allora
    $G$ è \emph{connesso}. Infatti, per ogni $v,v'\in V(G)=V(T)$ esiste un (unico)
    \emph{cammino} $C$ in $T$ che congiunga $v$ con $v'$. Poiché $T$ è un
    \emph{sottografo} di $G$, $C$ è anche un \emph{cammino} in $G$. Segue che $G$
    è \emph{connesso}. 
\end{observation}

\begin{theorem}
    Ogni \emph{grafo finito e connesso} possiede almeno un \emph{albero di copertura}.
\end{theorem}
\begin{observation}
    Si può dimostrare che anche ogni \emph{grafo infinito e connesso} possiede un
    \emph{albero di copertura}, di conseguenza, un \emph{grafo} è \emph{connesso}
    se e soltanto se possiede un \emph{albero di copertura}.
\end{observation}






\chapter{Teorema dello score}
\begin{lemma}
    Sia $n\in\N\backslash\{0\}$ e sia $d=(d1,\dots,d_n)\in\N^n$ un vettore tale
    che $d_1\leq\dots\leq d_n\leq 2$. Le seguenti affermazioni sono verificate:
    \begin{enumerate}[label=(\roman*)]
        \item Se $d=(\underbrace{0,\dots,0}_{(n-1)-volte},2)$ oppure $d=(
        \underbrace{0,\dots,0}_{(n-2)-volte},2,2)$, allora $d$ non è lo
        \emph{score} di un \emph{grafo}
        \item Se $d=(0,\dots,0)$, allora $d$ è lo \emph{score} di un \emph{grafo}
        avente $n$ \emph{vertici} isolati. Inoltre, se esiste $m\in\N$ tale che
        $n\geq m\geq 3$ e $d=(\underbrace{0,\dots,0}_{(n-m)-volte},
        \underbrace{2,\dots,2}_{m-volte})$, allora $d$ è lo \emph{score} di un
        \emph{grafo} avente $n-m$ \emph{vertici} isolati e un \emph{m-ciclo}
        \item Se $d$ possiede un numero dispari di componenti pari a 1, allora $d$
        non è lo \emph{score} di un \emph{grafo}. Se invece ha un numero pari di
        componenti pari a 1, $d$ è un vettore della seguente forma:
        \[d=(\underbrace{0,\dots,0}_{h-volte}, \underbrace{1,\dots,1}_{(2k+2)-volte},
        \underbrace{2,\dots,2}_{m-volte}) \text{ per qualche } h,k,m\in\N\]
        ed è lo \emph{score} di un \emph{grafo} avente $n$ \emph{vertici} isolati,
        $k$ segmenti (coppie di \emph{vertici} di \emph{grado} 1) e una linea con
        $m+2$ nodi.

        In particolare:
        \begin{itemize}
            \item Se $h=0$, i \emph{vertici} isolati vengono eliminati
            \item Se $k=0$, i segmenti vengono eliminati
            \item Se $m=0$, la linea si restringe fino a diventare un segmento
        \end{itemize}
    \end{enumerate}
\end{lemma}

\begin{corollary}
    Sia $n\in\N\backslash\{0\}$ e sia $d=(d_1,\dots,d_n)\in\N^n$ tale che $d_1\leq
    \dots\leq d_n\leq 2$ e il numero di componenti pari a 1 è pari. Allora $d$
    non è lo \emph{score} di un \emph{grafo} se e soltanto se $d$ non ha una delle
    seguenti forme:
    \[d=(0,\dots,0,2)\text{ oppure }d=(0,\dots,0,2,2)\]
\end{corollary}

\begin{theorem}[Teorema dello score]
    Siano, $n\in\N$ con $n\geq 2$ e $d=(d_1,\dots,d_n)\in\N^n$ tale che $d_1\leq
    \dots\leq d_n\leq n-1$. Definiamo il vettore $d'=(d'_1,\dots,d'_{n-1})\in\N^{n-1}$
    ponendo:
    \[d'_i\coloneqq\begin{cases}
        d_i & \text{se } i<n-d_n\\
        d_i-1 & \text{se } i\geq n-d_n
    \end{cases}\forall i\in\{1,\dots,n-1\}\]
    Allora $d$ è lo \emph{score} di un \emph{grafo} se e solo se lo è $d'$.
\end{theorem}

\chapter{Ostruzioni}
\section{Ostruzioni per grafi}
Le \emph{ostruzioni} sono condizioni che si applicano a delle n-uple di valori
interi. Le n-uple che verificano un'\emph{ostruzione} non possono essere lo
\emph{score} di un \emph{grafo}. Se invece non si verificano, non si può affermare
nulla sulla natura di quell'n-upla.

\subsection{Ostruzione 1}
\begin{lemma}
    Sia $n\in\N$ con $n\geq 1$. Se $G=(V,E)$ è un \emph{grafo} con $n$ \emph{vertici},
    allora vale:
    \[deg_G(v)\leq n-1\ \forall v\in V\]
\end{lemma}
\begin{corollary}[Ostruzione 1]
    Siano $n\in\N$, con $n\geq 1$ e $d=(d_1,d_2,\dots,d_n)\in\N^n$ con $d_1\leq d_2
    \leq\dots\leq d_n$. Se $d_n>n-1$, allora non esiste alcun \emph{grafo} $G$
    avente $d$ come \emph{score}.
\end{corollary}

\begin{observation}
    Siano $n,m\in\N\{0\}$ e sia $d\in\N^{n+m}$ un vettore della seguente forma:
    \[d=(\underbrace{0,\dots,0}_{m-volte},d_1,d_2,\dots,d_n)\]
    dove $0<d_1\leq d_2\leq\dots\leq d_n$. Definiamo $d'\coloneqq(d_1,d_2,\dots,d_n)$.
    Osserviamo che $d$ è lo \emph{score} di un \emph{grafo} se e soltanto se lo è
    $d'$. Infatti, se $d$ è lo \emph{score} di un \emph{grafo} $G$, questi avrà $m$
    \emph{vertici} isolati e di conseguenza, togliendoli si ottiene un \emph{grafo}
    $G'$ con \emph{score} $d'$. Viceversa, se $G'$ è un \emph{grafo} con \emph{score}
    $d'$, aggiungendo $m$ \emph{vertici} isolati si ottiene un \emph{grafo} $G$ con
    \emph{score} $d$.
\end{observation}

\subsection{Ostruzione 2}
\begin{lemma}
    Siano $n,k\in\N\backslash\{0\}$ tali che $k<n$ e sia $h\coloneqq n-k$. Sia
    ancora $d\in\N^n$ un vettore di forma:
    \[d=(d_1,\dots,d_n,\underbrace{d_{n+1},\dots,d_m}_{\text{ultime k componenti}})\]
    con $d_1\leq\dots\leq d_n<n-1=d_{n+1}=\dots=d_m$. Allora, se $d$ è lo \emph{score}
    di un \emph{grafo}, vale $d_1\geq k$.
\end{lemma}
\begin{corollary}[Ostruzione 2]
    Siano $h,k\in\N\backslash\{0\}$. Siano poi, $n\coloneqq h+k$ e $d\in\N^n$ un
    vettore nella forma:
    \[d=(d_1,\dots,d_h,\underbrace{n-1,\dots,n-1}_{k-volte})\]
    con $d_1\leq\dots\leq d_h<n-1$. Se $d_1<k$, allora $d$ non è lo \emph{score} di
    un \emph{grafo}.
\end{corollary}

\subsection{Ostruzione 3}
\begin{lemma}(Ostruzione 3)
    Siano $n\in\N$ con $n\geq 3$ e $d=(d_1,\dots,d_{n-1},d_n)\in\N^n$ con
    $d_1\leq\dots\leq d_{n-1}\leq d_n$. Sia poi $L\in\N$ definito come:
    \[L\coloneqq|\{i\in\{i,\dots,n-2\}\ :\ d_i\geq 2\}|\]
    Se $L<d_{n-1}+d_n-n$ allora $d$ non è lo \emph{score} di un \emph{grafo}.
\end{lemma}

\subsection{Ostruzione 4}
\begin{lemma}[Ostruzione 4: Lemma delle strette di mano]
    Sia $n\in\N\backslash\{0\}$ e sia poi $d=(d_1,\dots,d_n)\in\N^n$. Se $d$
    possiede un numero dispari di componenti dispari, allora $d$ non può essere lo
    \emph{score} di un \emph{grafo}.
\end{lemma}

\section{Ostruzioni per grafi isomorfi}
Siano $G$ e $G'$ due \emph{grafi finiti isomorfi}. Valgono le seguenti proprietà:
\begin{enumerate}
    \item $score(G)=score(G')$
    \item $G$ e $G'$ hanno lo stesso numero di \emph{componenti connesse}. In
    particolare, sono entrambi \emph{connessi} o entrambi \emph{sconnessi}
    \item $G$ e $G'$ sono entrambi \emph{2-connessi} o non lo è nessuno
    \item $G$ e $G'$ sono entrambi \emph{Hamiltoniani} o non lo è nessuno
    \item $G$ e $G'$ hanno lo stesso numero di \emph{sottografi} che sono
    \emph{3-cicli}, \emph{4-cicli}, $\dots$
    \item Siano $f:V(G)\to V(G')$ un \emph{isomorfismo} da $G$ in $G'$, $v\in V(G)$
    un \emph{vertice} di $G$ con $deg_G(v)=k$, e $v_1,\dots,v_k\in V(G)$ suoi
    \emph{vertici} adiacenti, allora $deg_G(v)=deg_G(f(v))$ e
    $deg_G(v_i)=deg_G(f(v_i))\ \forall i\in\{1,\dots,k\}$
\end{enumerate}
Tutti i \emph{grafi} per i quali almeno una di queste proprietà non è verificata
non possono essere \emph{isomorfi}.

\section{Altri risultati utili}
\subsection{Forzatura alla sconnessione}
\begin{lemma}[Forzatura alla sconnessione]
    Sia $n\in\N\backslash\{0\}$ e sia $d=(d_1,\dots,d_n)\in\N^n$. Se vale:
    \begin{equation} \label{eq:F-S}\tag{F-S}
        \frac{1}{2}\sum_{i=1}^nd_i<n-1
    \end{equation}
    allora tutti i \emph{grafi} che hanno \emph{score} uguale a $d$ sono
    \emph{sconnessi}.
\end{lemma}

\subsection{Forzatura alla connessione}
\begin{lemma}[Forzatura alla connessione]
    Sia $n\in\N\backslash\{0\}$ e sia $d=(d_1,\dots,d_n)\in\N^n$ tale che $d_1\leq
    \dots\leq d_n$. Se vale $d_1+d_n\geq n-1$, ovvero:
    \begin{equation}\label{eq:F-C}\tag{F-C}
        d_1\geq n-d_n-1
    \end{equation}
    allora tutti i \emph{grafi} che hanno \emph{score} uguale a $d$ sono
    \emph{connessi}.
\end{lemma}
\begin{observation}
    La precedente condizione (\ref{eq:F-C}) si può riscrivere equivalentemente come:
    \begin{equation}\label{eq:F-C2}\tag{F-C}
        d_1+d_n\geq n-1
    \end{equation}
    Dunque, se $d=(d_1,\dots,d_n)\in\N^n$ con $n\geq 1$ e $d_1\leq\dots\leq d_n$,
    e vala le precedente disuguaglianza, allora ogni \emph{grafo} con \emph{score}
    $d$, se esiste, è \emph{connesso}.
\end{observation}

\end{document}