\documentclass[12pt, a4paper]{report}

\usepackage[utf8]{inputenc}
\usepackage{geometry}
 \geometry{
 a4paper,
 total={170mm,257mm},
 left=20mm,
 top=20mm,
}

\usepackage{titlesec}
\titleformat
{\chapter}
[display]{\bfseries\Large\itshape}
{Capitolo Nr.\thechapter}
{0.5ex}
{
    \rule{\textwidth}{1pt}
    \vspace{1ex}
	\centering
}
[\vspace{-0.5ex}\rule{\textwidth}{0.3pt}]

\renewcommand{\contentsname}{Indice}

\usepackage{amsthm} % Fornisce il comando \newtheorem
\usepackage{thmtools} % fornisce il comando \declaretheorem
\usepackage{amsmath}
\usepackage{amsfonts}
\usepackage{amssymb}
\usepackage{enumitem} % Permette di usare la numerazione romana nelle liste
\usepackage{latexsym}
\usepackage{mathtools}
\usepackage{dsfont}
\usepackage{nccmath}

\theoremstyle{definition}
\newtheorem{definition}{Definizione}[section]
\newtheorem{theorem}{Teorema}[section]
\newtheorem{corollary}{Corollario}[theorem]
\newtheorem{lemma}{Lemma}[theorem]
\newtheorem{observation}{Oss}[section]
\declaretheorem[name=Dim, qed=$\blacksquare$, numbered=no]{demonstration}
\newtheorem*{proposition}{Prop}
\newtheorem*{property}{Proprietà}
\newtheorem*{note}{NB}

\newcommand{\Z}{\mathbb{Z}}
\newcommand{\N}{\mathbb{N}}
\newcommand{\Mod}[1]{\ (\mathrm{mod}\ #1)}
\newcommand{\inv}{(\Z/_{n\Z})^*}

\begin{document}
\paragraph{\emph{1. L’ordinamento dei numeri naturali è un buon ordinamento e seconda
forma del principio d'induzione}}
\paragraph{Teorema 7.4 (Buon ordinamento).}
L'ordinamento dei numeri naturali è un buon ordinamento.

\begin{demonstration}
    Suppongo che $A\subseteq\N$ non abbia minimo e dimostro che $A=\emptyset$.
    Sia $B$ il suo complementare, ovvero $B=\N-A$, e dimostro per induzione che
    \[\forall n\in\N\ \{0,\dots,n\}\subseteq B\]
    $0\notin A$ poiché altrimenti ne sarebbe il minimo, dunque $\{0\}
    \subseteq B$. Suppongo ora che $\{0,\dots,n\}\subseteq B$, quindi $0,\dots n
    \notin A$, ma allora $n+1\notin A$ altrimenti ne sarebbe il minimo, quindi
    $\{0,\dots,n+1\}\subseteq B$. Ma allora, $B=\N$ e $A=\emptyset$.
\end{demonstration}

\paragraph{Teorema 7.5 (Seconda forma dell'induzione).}
Sia $\mathcal{P}(n)$ una famiglia di proposizione indicizzate su $\N$ e si
supponga che valgano le seguenti:
\begin{enumerate}[label=(\roman*)]
    \item $\mathcal{P}(0)$ è vera
    \item $\forall n\in\N\ (\mathcal{P}(k)\text{ vera }\forall k<n)\Rightarrow
    \mathcal{P}(n)$ vera
\end{enumerate}
allora, $\mathcal{P}(n)$ è vera $\forall n\in\N$.

\begin{demonstration}
    Sia $A=\{n\in\N:\mathcal{P}(n)\text{ non è vera}\}$. Suppongo per assurdo
    che $A\neq\emptyset$, dunque per il Teorema di buon ordinamento, $A$ ha un
    minimo: $n\coloneqq\min{A}$. Per l'ipotesi (i), $\mathcal{P}(0)$ è vera,
    dunque $0\notin A$. Inoltre, se $k<n$, $k\notin A$ perché $n$ ne è il minimo.
    Ma allora, $\mathcal{P}(k)$ è vera $\forall k<n$, e quindi per (ii), anche
    $\mathcal{P}(n)$ è vera, di conseguenza $n\notin A$, contraddicendo il fatto
    che $n\in A$.
\end{demonstration}

\paragraph{\emph{2. Esistenza e unicità di quoziente e resto nella divisione euclidea
tra numeri interi}}
\paragraph{Teorema 7.7.} Siano $n,m\in\Z$ con $m\neq 0$. Esistono e sono unici
$q,r\in\Z$ tali che:
\[\begin{cases}
    n=mq+r\\
    0\leq r<|m|
\end{cases}\]

\begin{demonstration}
    Esistenza.
    Ipotizzo che $n,m\in\N$ e procedo per induzione su $n$. Se $n=0$ basta
    porre $q=0$ e $r=0$. Se invece $0<n$ e $n<m$ pongo $q=0$ e $r=n$, altrimenti
    ipotizzo che la tesi sia vera per ogni $k<n$ e pongo $k=n-m$. Poiché $m\neq0$,
    $0\leq k<n$ e per ipotesi induttiva, vale:
    \[\begin{cases}
        k=mq+r\\
        0\leq r<m
    \end{cases}\]
    Ma, $n=k+m=(mq+r)+m=m(q+1)+r$.

    Siano ora, $n<0$ e $m>0$, allora $-n>0$ e quindi per il caso precedente
    \[\begin{cases}
        -n=mq+r\\
        0\leq r<m=|m|
    \end{cases}\]
    e dunque, $n=m(-q)-r$. Se $r=0$ ho finito, se invece $0<r<m=|m|$ vale
    $0<m-r<m=|m|$ e quindi $n=m(-q)-m+(m-r)=m(-q-1)+m-r$.

    Infine, se $m<0$, allora $-m>0$ e per i due casi precedenti $\exists q,r\in\Z$
    tali che $n=(-m)q+r$ con $0\leq r<-m=|m|$.

    Unicità. Sia $n=mq+r=mq'+r'$ con $0\leq r,r'<|m|$. Ipotizzo $r'\geq r$, quindi
    vale $m(q-q')=r'-r$ e passando al modulo ottengo $|m|\cdot|q-q'|=|r'-r|=r'-r<|m|$,
    da cui $0\leq |q-q'|<1$ e quindi $|q-q'|=0$, cioè $q=q'$. A questo punto, da
    $n=mq+r=mq'+r'$ segue che $r=r'$.
\end{demonstration}

\paragraph{\emph{3. Rappresentazione dei numeri naturali in una base arbitraria
maggiore o uguale a 2}}

\paragraph{Teorema 8.4 (Rappresentazione dei natural in base arbitraria).} Sia
$b\in\N$ con $b\geq 2$. Ogni numero $n\in\N$ è rappresentabile in base $b$, cioè
esiste una successione $\{\epsilon_i\}_{i\in\N}$, composta da valori in interi
$0\leq \epsilon_i<b_i$, che sia definitivamente nulla, ovvero per la quale
esiste un valore $k\in\N$ per cui $\epsilon_i=0\ \forall i>k$, e tale che
$n=\sum_{i=0}^{+\infty}\epsilon_ib^i$. Inoltre, se esiste un'altra tale
successione $\{\epsilon'_i\}_{i\in\N}$, vale $\epsilon_i=\epsilon'_i$ per ogni
$i\in\N$.
\begin{demonstration}
    Esistenza. Procedo per induzione su $n$. Se $n=0$ posso prendere $\epsilon_i=0$
    per ogni $i\in\N$. Se $n>0$, ipotizzo che la tesi sia vera $\forall k<n$.
    Considero la divisione euclidea tra $n$ e $b$, ovvero siano $q,r\in\Z$ tali che
    \[\begin{cases}
        n=bq+r\\
        0\leq r<b
    \end{cases}\]
    Se $n<b$, valgono $q=0$ e $r=n$, quindi posso definire una successione
    $\{\epsilon_i\}_{i\in\N}$ tale per cui $\epsilon_0=r=n$ e $\epsilon_i=0\
    \forall i>0$. Se invece $n\geq b$, poiché $b\geq2$, vale $0<q<bq\leq bq+r=n$,
    quindi per ipotesi induttiva esiste una successione definitivamente nulla
    $\{\delta_i\}_{i\in\N}$, costituita da interi $0\leq \delta_i<b$ e tale che
    $q=\sum_{i=0}^{+\infty}\delta_ib^i$. Ma allora:
    \[n=bq+r=b\sum_{i=0}^{+\infty}\delta_ib^i+r=\sum_{i=0}^{+\infty}\delta_ib^{i+i}
    +r=\sum_{i=1}^{+\infty}\delta_{i-1}b^i+r=\sum_{i=0}^{+\infty}\epsilon_ib^i\]
    con $\epsilon_0=r$ e $\epsilon_i=\delta_{i-1}$ per ogni $i\geq1$. La successione
    $\{\epsilon_i\}_{i\in\N}$ è definitivamente nulla perché lo è $\{\delta_i\}_
    {i\in\N}$, inoltre $0\leq\epsilon_i<b\ \forall i\in\N$.

    Unicità. Procedo per induzione su $n$. Se $n=0=\sum_{i=0}^{+\infty}\epsilon_ib^i$,
    poiché ogni termine $\epsilon_ib^i$ è non negativo e dato che $b\geq2$,
    necessariamente $\epsilon_i=0$ per ogni $i\in\N$.
    Se $n>0$, ipotizzo che la tesi sia vera $\forall k<n$. Sia $n=\sum_{i=0}^{+\infty}
    \epsilon_ib^i=\sum_{i=0}^{+\infty}\epsilon'_ib^i$. Posso scrivere:
    \[n=b\sum_{i=1}^{+\infty}\epsilon_ib^i+\epsilon_0=b\sum_{i=i}^{+\infty}
    \epsilon'_ib^i+\epsilon'_0\]
    Per l'unicità di quoziente e resto nella divisione euclidea tra numeri interi,
    $\epsilon_0=\epsilon'_0$ e $q=\sum_{i=1}^{+\infty}\epsilon_ib^i=\sum_{i=1}^
    {+\infty}\epsilon'_ib^i$, quindi poiché $q<n$, per ipotesi induttiva, si ha
    che $\epsilon_i=\epsilon'_i\ \forall i\geq1$.
\end{demonstration}

\newpage
\paragraph{\emph{4. Teorema di esistenza e unicità di M.C.D e m.c.m tra due numeri
interi non entrambi nulli}}
\paragraph{Teorema 9.8.} Dati due numeri $n,m\in\Z$ non entrambi nulli, esiste
il massimo comun divisore tra $n$ e $m$.
\begin{demonstration}
    Sia $S=\{s\in\Z|s>0,\exists x,y:s=nx+my\}$. Poiché $nn+mm>0$ (non sono
    entrambi nulli), $S\neq\emptyset$, dunque per il Teorema di buon ordinamento,
    $S$ ha minimo: $d\coloneqq nx+my=\min{S}$.

    Dimostro che $d$ è il massimo comun divisore. Se $c|n$ e $c|m$, allora $n=ch$
    e $m=ck$, quindi $d=nx+my=chx+xky=c(hx+ky)$, cioè $c|d$.

    Resta da dimostrare che $d|n$ e $d|m$. Se considero la divisione euclidea tra
    $n$ e $d$ ottengo $n=dq+r$ con $0\leq r<|d|$. Se $r>0$ potrei scrivere $r=n-dq
    =n-(nx+my)q=n(1-xq)+(-m)yq$. Quindi $r$ sarebbe un elemento di $S$, ma poiché
    $r<d=\min{S}$ questo è impossibile, di conseguenza $r=0$, ovvero $d|n$.

    Analogamente si può dimostrare che $d|m$.
\end{demonstration}

\paragraph{Proposizione 9.6.} Se $d$ e $d'$ sono due massimi comun divisori di
$n$ e $m$, allora $d'=\pm d$.

\begin{demonstration}
    Se $d$ è un divisore comune di $n$ e $m$, poiché $d'$ ne è il massimo comun
    divisore, si ha che $d|d'$. Invertendo i ruoli di $d$ e $d'$ si ottiene che
    anche $d'|d$. Poiché $d|d'$ e $d'|d$, $d=hd'$ e $d'=kd$, ma allora $d'=hkd'$,
    da cui deriva che o $d'=0$ e quindi $d=0$, oppure $1-hk=0$, ma allora o $h=k=1$
    e quindi $d=d'$, oppure $h=k=-1$ e quindi $d=-d'$. In definitiva, vale che
    $d'=\pm d$.
\end{demonstration}

\paragraph{Teorema 10.4 (Esistenza del m.c.m).}
Siano $n,m\in\Z$ non entrambi nulli. Esiste il minimo comune multiplo tra di essi.
\begin{demonstration}
    Sia $M=\frac{n\cdot m}{(n,m)}=n'm'(n,m)$ con $n=n'(n,m)$ e $m=m'(n,m)$.
    Chiaramente $M=n\cdot m'=n'\cdot m$, quindi $n|M$ e $m|M$.
    
    Se $n|c$ e $m|c$, $(n,m)|c$ e quindi, posto $c=c'(n,m)$, ho che $n'|c'$ e
    $m'|c'$. Poiché $n'=\frac{n}{(n,m)}$ e $m'=\frac{m}{(n,m)}$, $(n',m')=\left(
    \frac{n}{(n,m)},\frac{m}{(n,m)}\right)=1$, $n'm'|c'$ e quindi $M=n'm'(n,m)|c'
    (n,m)=c$.
\end{demonstration}
\noindent
L'unicità si dimostra come per il massimo comun divisore.

\paragraph{\emph{5. Teorema fondamentale dell'aritmetica}}
\paragraph{Teorema 10.5 (Teorema fondamentale dell'aritmetica).}
Per ogni $n\in\Z$ con $n\geq 2$ esistono numeri primi $p_1,\dots,p_k$ positivi
tali che $\prod_{i=1}^kp_i=n$. Se anche $q_i,\dots,q_h$ sono numeri primi
positivi tali che $\prod_{j=1}^{h}q_j=n$, allora esiste una bigezione $\sigma:
\{1,\dots,h\}\to\{1,\dots,k\}$ tale che $q_i=p_{\sigma(i)}$.

In altre parole, ogni numero intero maggiore o uguale a 2 è rappresentabile in
modo unico a meno di riordinamento come prodotto di numeri primi positivi.

\begin{demonstration}
    Esistenza. Procedo per induzione su $n$. Se $n=2$ non devo fare nulla
    perché 2 è un numero primo. Se $n>2$, ipotizzo che la tesi sia vera per
    ogni $d<n$. Se $n$ è primo non c'è nulla da dire, altrimenti esistono
    sicuramente due numeri $d_1,d_2\in\Z$ tali che $1<d_1,d_2<n$. Per ipotesi
    induttiva, $d_1=\prod_{i=1}^kp_i$ e $d_2=\prod_{j=1}^hq_j$, quindi poiché
    $n=d_1\cdot d_2=\left(\prod_{i=1}^{k}p_i\right)\cdot\left(\prod_{j=1}^hq_j
    \right)$, ossia è esprimibile come prodotto di numeri primi.
    
    \newpage
    Unicità. Sia $n=\prod_{i=1}^kp_i=\prod_{j=1}^hq_j$ e ipotizzo $k\leq h$.
    Procedo per induzione su $k$.
    
    Se $k=1$ allora $n=p_1=\prod_{j=1}^hq_j$, quindi $q_j|p_1\ \forall j\in\{1,
    \dots,h\}$. Ma, poiché $p_1$ è un numero primo, o $q_j=1$ o $q_j=p_1$. Siccome,
    per ipotesi, tutti i $q_j$ sono numeri primi positivi, necessariamente $q_j=p_1$.
    A questo punto, se $h>1$ si avrebbe $n=\prod_{j=1}^hq_j\geq q_1\cdot q_2>
    q_1=p_1=n$ e questo è assurdo, per cui $h=1$.

    Se $k>1$, ipotizzo che la tesi sia vera per ogni $d<k$. Sicuramente, $p_k|n$,
    dunque so che esiste un $j$ tale che $q_j|p_k$, ma poiché sia $q_j$ che $p_k$
    sono numeri interi positivi, vale $q_j=p_k$. Ma allora, $p_1\dots p_{k-1}=q_1\dots
    q_{j-1}\cdot q_{j+1}\dots q_h$ e, per ipotesi induttiva, le due fattorizzazioni
    hanno lo stesso numero di elementi, ovvero $k-1=h-1$. Esiste quindi una
    bigezione \mbox{$\delta:\{1,\dots,j-1,j+1,\dots,h\}\to\{1,\dots,k-1\}$} tale
    che $q_i=p_{\delta(i)}$ per ogni $i$. A questo punto, definendo
    $\sigma:\{1,\dots,k\}\to\{1,\dots,k\}$ come:
    \[\sigma(i)=\begin{cases}
        k & \text{se } i=j\\
        \delta(i) & \text{se } i\neq j
    \end{cases}\]
    si ottiene una bigezione tale che $q_i=p_{\sigma(i)}$ per ogni $i$.
\end{demonstration}

\paragraph{\emph{6. Teorema cinese del resto}}
\paragraph{Teorema 12.1 (Teorema cinese del resto).}
Il sistema di congruenze
\[\begin{cases}
    x\equiv a\Mod{n}\\
    x\equiv b\Mod{m}
\end{cases}\]
ha soluzione se e soltanto se $(n,m)|b-a$. Inoltre, le soluzione sono tutti e
soli gli elementi di $[c]_{[n,m]}$.

\begin{demonstration}
    Sia $c$ una soluzione del sistema. Esistono $h,k\in\Z$ tali che $c=a+hn=b+km$
    e quindi $hn-km=b-a$. Siccome $(n,m)|n$ e $(n,m)|m$ si ha che $(n,m)|hn-km=b-a$.
    Viceversa, se ipotizzo che $(n,m)|b-a$, esistono $h,k\in\Z$ tali che $hn+km=b-a$,
    da cui $a+hn=b-km$. Se ora pongo $c=a+hn=b-km$, è evidente che $c$ è una
    soluzione del sistema.

    Sia $S=\{x\in\Z|x \text{ è soluzione del sistema}\}$. Devo dimostrare che se
    $c$ è una soluzione allora $S=[c]_{[n,m]}$.

    Ipotizzo $S\subseteq[c]_{[n,m]}$. Sia $c'\in S$ un'altra soluzione del sistema,
    allora $c=a+hn=b+km$ e $c'=a+h'n=b+k'm$. Se ora calcolo $c-c'$ ottengo:
    \[c-c'=a+hn-(a+h'n)=n(h-h')\Rightarrow n|c-c'\]
    \[c-c'=b+km-(b+k'm)=m(k-k')\Rightarrow m|c-c'\]
    Ma allora, $[n,m]|c-c'$ ossia $c'\equiv c\Mod{[n,m]}$, ovvero $c'\in[c]_{[n,m]}$.

    Infine, suppongo $[c]_{[n,m]}\subseteq S$. Sia $c'\in[c]_{[n,m]}$, ovvero $c'=
    c+h[n,m]$. Poiché $c\equiv a\Mod{n}$ e $h[n,m]\equiv 0\Mod{n}$, $c'=c+h[n,m]
    \equiv a\Mod{n}$. In modo analogo si dimostra che $c'\equiv b\Mod{m}$ e che
    quindi $c'\in S$.
\end{demonstration}

\newpage
\paragraph{\emph{7. Teorema di Fermat-Eulero e crittografia RSA}}
\paragraph{Teorema 13.9.}
Sia $n>0$. Per ogni $\alpha\in\inv$, vale:
\[\alpha^{\Phi(n)}=[1]_n \text{ in } \Z/_{n\Z}\]
o, equivalentemente, $\forall\alpha\in\Z\ t.c.\ (\alpha,n)=1$, vale:
\[\alpha^{\Phi(n)}\equiv 1\Mod{n}\]

\begin{demonstration}
    Sia $\alpha\in\inv$. Considero la seguente funzione:
    \[L_\alpha:\inv\to\inv\]
    definita in modo che $L_\alpha(\beta)\mapsto\alpha\beta\ \forall\beta\in\inv$.
    Poiché l'insieme $\inv$ è finito e coincide sia col dominio che col codominio,
    se riesco a dimostrare che $L_\alpha$ è iniettiva, sarà anche suriettiva.

    Dimostro quindi l'iniettività. Siano $\beta_1,\beta_2\in\inv\ t.c.\ L_\alpha
    (\beta_1)=L_\alpha(\beta_2)$. Provo che $\beta_1=\beta_2$. Vale:
    \[L_\alpha(\beta_1)=L_\alpha(\beta_2)\Leftrightarrow\alpha\beta_1=\alpha\beta_2
    \Rightarrow\alpha^{-1}\alpha\beta_1=\alpha^{-1}\alpha\beta_2\Rightarrow
    [1]_n\beta_1=[1]_n\beta_2\Rightarrow\beta_1=\beta_2\]
    Dunque, $L_\alpha$ è iniettiva e suriettiva, ovvero è una bigezione.
    
    Passo ora alla dimostrazione del teorema. Sia $k\coloneqq\Phi(n)$ e
    $\inv=\{\beta_1,\dots,\beta_k\}$, allora gli elementi $L_\alpha(\beta_1),
    \dots,L_\alpha(\beta_k)$ sono tutti e soli gli elementi di $\inv$, a meno di
    riordinamento. Poiché il prodotto in $\inv$ è commutativo, vale:
    \[\prod_{i=i}^k\beta_i=\prod_{i=1}^kL_\alpha(\beta_i)=\prod_{i=1}^k\alpha\beta_i
    =\alpha^k\prod_{i=1}^k\beta_i\text{ in }\inv\]
    Sia $\inv\ni\gamma\coloneqq\prod_{i=1}^k\beta_i$, segue che:
    \[\gamma=\alpha^k\gamma\Rightarrow\gamma^{-1}\cdot\gamma=\alpha^k\gamma\cdot
    \gamma^{-1}\Rightarrow[1]_n=\alpha^k\]
    Se $\alpha\in\Z$ con $(\alpha,n)=1$, allora $[\alpha]_n\in\inv$ e per la
    precedente segue che:
    \[[\alpha]_n^{\Phi(n)}=[1]_n\Rightarrow[\alpha^{\Phi(n)}]_n=[1]_n\Leftrightarrow
    \alpha^{\Phi(n)}\equiv1\Mod{n}\]
\end{demonstration}

\paragraph{Teorema fondamentale della crittografia RSA.}
Sia $c\in\N\backslash\{0\}\ t.c.\ (c,\Phi(n))=1$ e sia $d\in[c]^{-1}_{\Phi(n)}$ con
$d>0$. Allora, la funzione $P_c$, che eleva il suo argomento alla potenza $c$, è
una funzione invertibile e vale:
\[(P_c)^{-1}=P_d\]
\begin{demonstration}
    Sia $\alpha\in\inv$. Devo dimostrare che $P_d(P_c(\alpha))=\alpha$. Ricordo
    che, poiché $d\in[c]^{-1}_{\Phi(n)}$, vale:
    \[c\cdot d\equiv1\Mod{\Phi(n)}\Leftrightarrow
    \Phi(n)|c\cdot d-1\Leftrightarrow\exists k\in\Z\ t.c.\ c\cdot d-1=k\cdot
    \Phi(n)\Leftrightarrow c\cdot d=1+k\cdot\Phi(n)\]
    Osservo che, $k\cdot\Phi(n)=c\cdot d-1$ e, siccome per ipotesi $c,d\geq1$,
    $c\cdot d-1\geq0$, inoltre $\Phi(n)>0$, dunque anche $k\geq0$. Per il Teorema
    13.9 vale $\alpha^{\Phi(n)}=[1]_n$, quindi segue che:
    \[P_d(P_c(\alpha))=P_d(\alpha^c)=(\alpha^c)^d=\alpha^{c\cdot d}=\alpha^{1+
    k\cdot\Phi(n)}=\alpha^1\cdot\alpha^{k\cdot\Phi(n)}=\alpha\cdot(\alpha^
    {\Phi(n)})^k=\alpha\cdot([1]_n)^k=\alpha\]
\end{demonstration}

\newpage
\paragraph{\emph{8. Teorema di equivalenza tra la congiungibilità con cammini e
la congiungibilità con passeggiate; la relazione di congiungibilità è una
relazione di equivalenza}}
\paragraph{Proposizione 15.8.}
Sia $G$ un grafo e siano $u,v\in V(G)$ suoi vertici, allora $u$ e $v$ sono
congiungibili mediante cammino se e solo se lo sono mediante una passeggiata.

\begin{demonstration}
    Dato che un cammino è anche una passeggiata, se due vertici sono congiungibili
    mediante un cammino lo sono anche mediante una passeggiata.

    Viceversa, suppongo che tra due vertici $u,v\in V(G)$ esista una passeggiata.
    Definisco i seguenti insiemi:
        \[\mathcal{P}=\{P\ |\ P\text{ è una passeggiata tra $u$ e $v$}\}\
        A=\{l(P)|P\in\mathcal{P}\}\]
    Poiché i vertici $u$ e $v$ sono congiungibili per passeggiata, $\mathcal{P}
    \neq\emptyset$ e quindi anche $A\neq\emptyset$. Ma allora, per il Teorema di
    buon ordinamento, $A$ possiede un minimo, ovvero esiste una passeggiata $P_0$
    da $u$ a $v$ che ha lunghezza minima, nel senso che:
    \[l(P_0)\leq l(P)\ \forall P\in\mathcal{P}\]
    Dimostro che $P_0$ è un cammino. Sia $P_0=\{v_0,\dots,v_n\}$. Se per assurdo
    $P_0$ non fosse un cammino, esisterebbero $i,j\in\{0,\dots,n\}$ con $i<j$ tali
    che $v_i=v_j$. Si consideri quindi, $P_1=\{v_0,\dots,v_i,v_{j+1},\dots,v_n\}$.
    $P_1$ è una passeggiata dato che lo è anche $P_0$, cioè vale:
    \[\{v_h,v_{h+1}\}\in E(G)\ \forall 0\leq h<n\]
    e poiché $v_i=v_j$, $\{v_i,v_{i+1}\}=\{v_j,v_{j+1}\}\in E(G)$. Dato che,
    $v_0=u$ e $v_n=v$, $P_1$ congiunge $u$ a $v$, ovvero $P_1\in\mathcal{P}$, ma
    siccome $l(P_1)=l(P_0)-(j-i)<l(P_0)$, ciò contraddice la minimalità di $P_0$.
    È stato quindi assurdo supporre che $P_0$ non fosse un cammino.
\end{demonstration}

\paragraph{Proposizione 15.9.} La relazione di congiungibilità è una relazione
di equivalenza.

\begin{demonstration}
    Sia $G$ un grafo. Indico con $\sim$ la relazione di congiungibilità, ovvero se
    $u,v\in V(G)$, $u\sim v$ se e solo se $u$ è congiungibile a $v$. Devo dimostrare
    che per $\sim$ valgono le proprietà riflessiva, simmetrica e transitiva.
    \begin{enumerate}[label=(\roman*)]
        \item \emph{Riflessività}: se $v\in V(G)$, $\{v\}$ è un cammino che
        congiunge $v$ a $v$ e dunque $v\sim v$
        \item \emph{Simmetria}: se $v,w\in V(G)$ con $v\sim w$, esiste un cammino
        $(v_0,\dots,v_n)$ con $v_0=v$ e $v_n=w$ che congiunge $v$ a $w$. Allora,
        invertendo l'ordine dei vertici si ottiene $(v_n,\dots,v_0)$ che è un
        cammino da $w$ a $v$, ovvero $w\sim v$
        \item \emph{Transitività}: se $v,w,z\in V(G)$ con $v\sim w$ e $w\sim z$,
        esistono due passeggiate $P_1=(v_0,\dots,v_n)$ e $P_2=(w_0,\dots,w_m)$
        con $v_0=v$, $v_n=w_0=w$ e $w_m=z$. Sia $Q=(v_0,\dots,v_n,w_1,\dots,w_m)$.
        Poiché $v_n=w_0$, $\{w_0,w_1\}=\{v_n,w_1\}\in E(G)$ e quindi $Q$ è una
        passeggiata. Dato che $v_0=v$ e $w_m=z$, $Q$ è una passeggiata da $v$ a $z$,
        ovvero $v\sim z$
    \end{enumerate}
\end{demonstration}

\newpage
\paragraph{\emph{9. Relazione fondamentale dei grafi finiti e Lemma delle strette
di mano}}
\paragraph{Proposizione 17.2.}
Se $G=(V,E)$ è un grafo finito, allora:
\[\sum_{v\in V}\deg_G(v)=2|E|\]

\begin{demonstration}
    Siano $V=\{v_1,\dots,v_n\}$ e $E=\{e_1,\dots,e_k\}$. Per ogni $i\in\{1,\dots,n\}$
    e $j\in\{1,\dots,k\}$, definisco il numero $m_{i,j}={0,1}$ come:
    \[m_{i,j}=\begin{cases}
        1 & \text{se}\ v_i\in e_j\\
        0 & \text{se}\ v_i\notin e_j
    \end{cases}\]
    Vale:
    \[\sum_{i=1}^n\left(\sum_{j=1}^km_{i,j}\right)=\sum_{j=1}^k\left(\sum_{i=1}
    ^nm_{i,j}\right)\]
    Per ogni $i\in\{1,\dots,n\}$, vale:
    \begin{equation}\label{eq:1}
        \sum_{j=1}^km_{i,j}=|\{j\in\{1,\dots,k\}|v_i\in e_j\}|=\deg_G(v)
        \Rightarrow
        \sum_{i=1}^n\left(\sum_{j=1}^km_{i,j}\right)=\sum_{i=1}^n\deg_G(v)
    \end{equation}
    Per ogni $j\in\{1,\dots,k\}$, vale:
    \begin{equation}\label{eq:2}
        \sum_{i=1}^nm_{i,j}=|\{i\in\{1,\dots,n\}|v_i\in e_j\}|=2
        \Rightarrow
        \sum_{j=1}^k\left(\sum_{i=1}^nm_{i,j}\right)=\sum_{j=1}^k2=2|E|
    \end{equation}
    Poiché (\ref{eq:1})=(\ref{eq:2}), $\sum_{i=1}^n\deg_G(v)=2|E|$.
\end{demonstration}

\paragraph{Corollario 17.6 (Lemma delle strette di mano).}
In un grafo finito il numero di vertici di grado dispari è sempre pari.

\begin{demonstration}
    Sia $G=(V,E)$ un grafo finito. Definisco $P$ e $D$ come segue:
    \[P=\{v\in V|\deg_G(v) \text{ è pari}\};\ D=\{v\in V|\deg_G(v) \text{ è dispari}\}\]
    Vale:
    \[\sum_{v\in P}\deg_G(v)+\sum_{v\in D}\deg_D(v)=\sum_{v\in V}\deg_G(v)
    \underrel{=}{\text{Relaz. fondamentale}}2|E|\]
    \[\Rightarrow\sum_{v\in D}\deg_G(v)=2|E|-\sum_{v\in P}\deg_G(v)\]
    Poiché al termine destro ci sono solo quantità pari, anche il termine sinistro
    deve esserlo, ma $\sum_{v\in D}\deg_G(v)$ è pari se e soltanto se $|D|$ è pari.
\end{demonstration}

\newpage
\paragraph{\emph{10. Teorema di caratterizzazione degli alberi finiti mediante
la formula di Eulero}}
\paragraph{Teorema 20.6.}
Sia $T=(V,E)$ un grafo finito. Sono fatti equivalenti:
\begin{enumerate}[label=(\roman*)]
    \item $T$ è un albero
    \item $T$ è connesso e vale la seguente formula di Eulero:
    \[|V|-1=|E|\]
\end{enumerate}
\begin{demonstration}
    $(i)\Rightarrow(ii)$. Procedo per induzione su $|V(T)|$. Se $|V(T)|=1$ la tesi
    è vera. Suppongo $|V(T)|\geq2$ e sia $v\in V(T)$ una foglia. Ora, $T-v$ è
    un albero e $|V(T-v)|=|V(T)|-1$. Per ipotesi induttiva, vale:
    \[|V(T)|-1-1=|V(T-v)|-1=|E(T-v)|\]
    Dato che $\deg_T(v)=1$, $|E(T-v)|=|E(T)|-1$ e quindi la tesi è verificata.

    $(ii)\Rightarrow(i)$. Devo dimostrare che $T$ non ha cicli. Procedo per
    induzione su $|V(T)|$. Se $|V(T)|=1$ la tesi è vera. Suppongo $|V(T)|\geq2$.
    Dimostro che $T$ ha una foglia. Dalla Relazione fondamentale dei grafi finiti,
    ottengo:
    \[2|V(T)|-2=2|E(T)|=\sum_{v\in V}\deg_T(v)\]
    Dato che $T$ è connesso ed ha almeno due lati, non possono esistere vertici
    di grado 0, dunque, se non esistessero foglie, ogni $v\in V(T)$ dovrebbe avere
    $\deg_T(v)\geq2$, ma questo genererebbe un assurdo perché varrebbe $2|V(T)|-
    2\geq2|V(T)|$. Pertanto, almeno un vertice deve essere di grado 1.
    Sia quindi $v\in V(T)$ una foglia e si consideri il grafo $T-v$.
    
    Dato che $T$ è connesso e $\deg_T(v)=1$, anche $T-v$ è connesso. Inoltre,
    poiché $|V(T-v)|=|V(T)|-1$ e $|E(T-v)|=|E(T)|-1$, si ha che $|V(T-v)|-1=
    |E(T-v)|$. Per ipotesi induttiva $T-v$ è un albero, ma allora $T$ non ha
    cicli in quanto i vertici di un ciclo hanno tutti grado almeno 2 e quindi
    un ciclo in $T$ non potrebbe passare per $v$, ossia sarebbe contenuto in
    $T-v$ contraddicendo il fatto che $T-v$ è un albero.
\end{demonstration}

\paragraph{\emph{11. Teorema di esistenza dell’albero di copertura per i grafi
connessi finiti}}
\paragraph{Teorema 21.3.}
Si $G$ un grafo connesso finito. Allora, $G$ ha un albero di copertura.

\begin{demonstration}[Prima dimostrazione]
    Si consideri l'insieme
    \[\mathcal{T}=\{T|T\text{ è un sottografo di $G$ e $T$ è un albero}\}\]
    $\mathcal{T}\neq\emptyset$, poiché se $v\in V(G)$, $\{\{v\},\emptyset\}\in\mathcal
    {T}$. Dato che $G$ è finito, esiste $\bar{T}\in\mathcal{T}$ con massimo numero
    di vertici, ossia tale che:
    \[|V(T)|\leq|V(\bar{T})|\ \forall T\in\mathcal{T}\]
    Devo dimostrare che $|V(\bar{T})|=|V(G)|$. Suppongo che esista $v\in V(G)
    \backslash V(\bar{T})$, allora, sfruttando la connessione di $G$, posso
    determinare due vertici $w\in V(G)\backslash V(\bar{T})$ e $u\in V(\bar{T})$
    tali che $\{u,w\}\in E(G)$. Ma allora, $T'=(V(\bar{T})\cup\{w\}, E(\bar{T})\cup
    \{u,w\})$ è sia un sottografo di $G$ che un albero. Quindi $T'\in\mathcal{T}$,
    ma poiché $|V(T')|=|V(\bar{T})|+1$, viene contraddetta la massimalità di
    $\bar{T}$.
\end{demonstration}

\newpage
\begin{demonstration}[Seconda dimostrazione]
    Si consideri l'insieme
    \[\mathcal{C}=\{C|C\text{ è un sottografo connesso di $G$ e }V(C)=V(G)\}\]
    $\mathcal{C}\neq\emptyset$ dato che $G\in\mathcal{C}$. Poiché $G$ è finito,
    esiste un grafo $\bar{C}\in\mathcal{C}$ con il minor numero di lati, ovvero:
    \[|E(\bar{C})|\leq|E(C)|\ \forall C\in\mathbb{C}\]
    Devo dimostrare che $\bar{C}$ è un albero. Se non lo fosse, per la proprietà
    (3) del Teorema di caratterizzazione degli alberi finiti, esisterebbe un lato
    $e\in E(\bar{C})$ tale che $\bar{C}-e$ è connesso. Ma, $V(\bar{C}-e)=V(C)=V(G)$,
    quindi $C\in\mathcal{C}$ e poiché, $|E(\bar{C}-e)|=|E(\bar{C})-1|=E(C)$, la
    minimalità di $\bar{C}$ viene contraddetta.
\end{demonstration}

\end{document}